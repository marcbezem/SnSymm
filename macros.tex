% macros are -*-latex-*- ones

%% macros about using categories
\DeclareMathOperator{\oboperator}{Ob}
\newcommand{\ob}[1]{\oboperator{#1}}
\DeclareMathOperator{\moroperator}{Mor}
\newcommand{\mor}[1]{\moroperator\left(#1\right)}
\newcommand{\op}[1]{{#1}^{\mathrm{op}}}
\newcommand{\comma}[2]{\left( #1 \mathbin{\downarrow} #2 \right)}
%\newcommand{\slice}[2]{\comma{#1}{#2}}
\newcommand{\slice}[2]{{#1}\kern-.5pt/\kern-1pt{#2}}
%\newcommand{\coslice}[2]{\comma{#2}{#1}}
\newcommand{\coslice}[2]{#2\kern-.5pt\backslash\kern-1pt#1}
\newcommand{\localize}[2]{{#2}^{-1}{#1}}
%\newcommand{\id}[1]{\mathrm{id}_{#1}}
\newcommand{\final}{1}
\newcommand{\initial}{0}
\newcommand{\psh}[1]{\widehat{#1}}
\DeclareMathOperator{\Pshoperator}{Psh}
\newcommand{\Psh}[1]{\Pshoperator(#1)}
\DeclareMathOperator{\Shoperator}{Sh}
\newcommand{\Sh}[1]{\Shoperator(#1)}
%\newcommand{\fiber}[2]{{#1}_{#2}}
\newcommand{\fiberprod}[3]{{#1}\times_{#3}{#2}}
\newcommand{\indout}[1]{\left\langle{#1}\right\rangle}
\newcommand{\indin}[1]{\left({#1}\right)}
\newcommand{\elemcat}[2][\null]{\int_{#1}{#2}}
\DeclareMathOperator*{\limoperator}{lim}
\renewcommand{\lim}[2][\null]{\limoperator_{#1}\left( #2 \right)}
\DeclareMathOperator*{\colimoperator}{colim}
\newcommand{\colim}[2][\null]{\colimoperator_{#1}\left( #2 \right)}
\newcommand{\limend}[1]{\int_{#1}}
\newcommand{\colimend}[1]{\int^{#1}}
\newcommand{\tens}{\mathbin \odot}
\newcommand{\cotens}{\mathbin \pitchfork}
%\newcommand{\adjoint}{\dashv}
\newcommand{\leftadjto}{\dashv}
\newcommand{\rightadjto}{\vdash}
\newcommand{\adjarrows}{\rightleftarrows}
\newcommand{\adjointarrows}{\rightleftarrows}
\newcommand{\ladjointmap}[2][]{{#2}_{#1}^\natural}
\newcommand{\radjointmap}[2][]{{#2}_{#1}^\flat}
\DeclareMathOperator{\spanoperator}{Span}
\newcommand{\catspan}[3][\null]{\spanoperator_{#1}\left(#2,#3\right)}
\DeclareMathOperator{\homoperator}{Hom}
% \renewcommand{\hom}[3][\null]{%
%         \ifx#1\null%
%                 \homoperator\left(#2,#3\right)%
%         \else #1\left(#2,#3\right) \fi}
\DeclareMathOperator{\Homoperator}{\underline{Hom}}
%\newcommand{\Hom}[3][\null]{\Homoperator_{#1}\left(#2,#3\right)}
\newcommand{\Arr}[1]{\operatorname{Arr}(#1)}
\DeclareMathOperator{\functorcatoperator}{Fun}
\DeclareMathOperator{\pseudofunctorcatoperator}{PFun}
\newcommand{\functorcat}[3][\null]{%
  \ifx#1\null%
  \functorcatoperator\left({#2},{#3}\right)%
  \else\Homoperator(#2,#3)%
  \fi%
}
\newcommand{\pseudofunctorcat}[3][\null]{%
  \ifx#1\null%
  \pseudofunctorcatoperator\left({#2},{#3}\right)%
  \else\Homoperator(#2,#3)%
  \fi%
}
\DeclareMathOperator{\opfiboperator}{OpFib}
\newcommand{\opfib}[1]{%
  \opfiboperator\left({#1}\right)%
}
\DeclareMathOperator{\bifiboperator}{BiFib}
\newcommand{\bifib}[1]{%
  \bifiboperator\left({#1}\right)%
}
\newcommand{\mono}{\hookrightarrow}
\newcommand{\epi}{\twoheadrightarrow}
\newcommand{\dom}{\mathrm{dom}}
\newcommand{\cod}{\mathrm{cod}}
\newcommand{\yoneda}[2][\null]{\mathfrak h^{#1}_{#2}}
\newcommand{\fromsum}[2]{\langle #1,#2 \rangle}
\newcommand{\toprod}[2]{\langle #1,#2 \rangle}
\newcommand{\inv}[1]{{#1}^{-1}}
\newcommand{\walkingcospan}{%
  \tikz[anchor=base,baseline,scale=.2]{%
    \draw[thick] (0,0) -| (1,1);%
  }%
}
\DeclareMathOperator{\nerveoperator}{N}
\newcommand{\nerve}[1]{\nerveoperator{#1}}

%% macros about bifibrational stuff
\DeclareMathOperator{\cartoperator}{Cart}
\newcommand{\cart}[3][\null]{\rho^{#1}_{#2,#3}}
\newcommand{\cartz}{\rho}
% \newcommand{\cart}[3][\null]{\cartoperator_{#1}\left(#2,#3\right)}
\DeclareMathOperator{\cocartoperator}{Cocart}
\newcommand{\cocart}[3][\null]{\lambda^{#1}_{#2,#3}}
\newcommand{\cocartz}{\lambda}
% \newcommand{\cocart}[3][\null]{\cocartoperator_{#1}\left(#2,#3\right)}
\newcommand{\push}[2]{{#1}_!#2}
\newcommand{\pull}[2]{{#1}^\ast#2}
\newcommand{\pushfact}[1]{{#1}_{\triangleright}}
\newcommand{\pullfact}[1]{{#1}^{\triangleleft}}
\newcommand{\middlefact}[3]{\vphantom{#1}_{#2}{#1}^{#3}}
\newcommand{\bigluing}[1]{\mathcal G\kern -.1em\ell\left(#1\right)}
\newcommand{\pseudo}[1]{\tilde{#1}}
\newcommand{\tensor}{\otimes}
\newcommand{\groth}[1]{\operatorname{\mathfrak{G}}({#1})}
\newcommand{\comp}[1]{\left\{#1\right\}}
\newcommand{\codtribe}[1]{\mathfrak p_{#1}}
\DeclareMathOperator{\subsetoperator}{Sub}
\newcommand{\Sub}[1]{\subsetoperator(#1)}

%% macros about Kan extensions
\newcommand{\restr}[1]{{#1}^\ast}
\newcommand{\lkan}[1]{{#1}_!}
\newcommand{\rkan}[1]{{#1}_\ast}


%% macros about formatting categories
\newcommand{\cat}[1]{\mathscr{#1}}
\newcommand{\concrete}[1]{\mymathsf{#1}}
% \newcommand{\Set}{\concrete{Set}}
% \newcommand{\Clan}{\concrete{Clan}}
% \newcommand{\RelClan}{\concrete{RelClan}}
% \newcommand{\Top}{\concrete{Top}}
% \newcommand{\Grpd}{\concrete{Grpd}}
% \newcommand{\Grp}{\concrete{Grp}}
% \newcommand{\Cat}{\concrete{Cat}}
% \newcommand{\Adj}{\concrete{Adj}}
% \newcommand{\Quil}{\concrete{Quil}}
% \newcommand{\Mod}{\concrete{Mod}}
% \newcommand{\CAT}{\concrete{CAT}}
% \newcommand{\Comp}{\concrete{Comp}}
% \newcommand{\sSet}{\boldsymbol{\cat S}}
\newcommand{\lincat}[1]{\mathbf{#1}}
\newcommand{\abgrp}[1]{{#1}_{\mathrm{ab}}}
\DeclareMathOperator{\lawthmodoperator}{Mod}
\newcommand{\lawthmod}[2][\null]{%
  \lawthmodoperator_{#2}
  \ifx#1\null%
  \null%
  \else%
  {\left(#1\right)}%
  \fi%
}
\newcommand{\finset}{\aleph_0}

%% simplicial stuff
\newcommand{\simpcat}{\boldsymbol\Delta}
\newcommand{\standsimp}[1]{\simpcat[#1]}
\newcommand{\simplicial}[1]{\boldsymbol{s}#1}
\newcommand{\face}[3][\null]{%
  \ifx#1\null%
  \partial_{#2}^{#3}%
  \else%
  \lexponent{\mathrm{d}_{#2}^{#3}}{#1}%
  \fi%
}
\newcommand{\degen}[3][\null]{%
  \ifx#1\null%
  \sigma_{#2}^{#3}%
  \else%
  \lexponent{\mathrm{s}_{#2}^{#3}}{#1}%
  \fi%
}

%% macros about model categories
\newcommand{\Ho}[2][\null]{\mathbf{Ho}_{\rm #1}\left( #2 \right)}
\newcommand{\class}[1]{\mathfrak{#1}}
\newcommand{\fib}{\mathrm{Fib}}
\newcommand{\cof}{\mathrm{Cof}}
%\newcommand{\weq}{\mathrm{W}}
\newcommand{\worth}[1][\null]{\mathbin{%
    \tikz[scale=.2,anchor=base,baseline]{%
      \draw (0,0) -- (1,0) -- (1,1) -- (0,1) -- (0,0) -- (1,1);%
    }_{#1}%
  }%
}%
\newcommand{\sorth}[1][\null]{
  \mathbin{\perp_{#1}}
}%
\newcommand{\bifibrant}[1]{{#1^{\rm cf}}}
\newcommand{\fibrant}[1]{{#1^{\rm f}}}
\newcommand{\cofibrant}[1]{{#1^{\rm c}}}
\newcommand{\fibrantrep}[1]{\mathfrak j_{#1}}
\newcommand{\cofibrantrep}[1]{\mathfrak q_{#1}}
\newcommand{\fibrantrepz}{\fibrantrep\null}
\newcommand{\cofibrantrepz}{\cofibrantrep\null}
\DeclareMathOperator*{\lderivoperator}{\mathbf L}
\DeclareMathOperator*{\rderivoperator}{\mathbf R}
\newcommand{\lderiv}[1]{\lderivoperator#1}
\newcommand{\rderiv}[1]{\rderivoperator#1}
\newcommand{\htpyclass}[1]{\left[ #1 \right]}
\newcommand{\htpyhom}[3][\null]{\pi\left(#2,#3\right)_{#1}}
\newcommand{\htpyquotient}[1]{\pi\,{#1}}
% \newcommand{\htpyhom}[3][\null]{\left[ #2,#3 \right]_{#1}}
\DeclareMathOperator*{\derivfwoperator}{\mathbf D}
\newcommand{\derivfw}[1]{\derivfwoperator{#1}}
\newcommand{\disc}[1]{#1_{\rm d}}
%\newcommand{\triv}[1]{#1_{\rm triv}}
\newcommand{\fiberwise}[1]{#1_{\rm fw}}
\newcommand{\lexponent}[2]{\vphantom{#1}^{#2}{#1}}
\newcommand{\llp}[1]{\lexponent{#1}{\worth}}
\newcommand{\rlp}[1]{{#1}^{\worth}}
\newcommand{\wllp}[2][\null]{\lexponent{#2}{\worth[#1]}}
\newcommand{\wrlp}[2][\null]{{#2}^{\worth[#1]}}
\newcommand{\sllp}[2][\null]{\lexponent{#2}{\sorth[#1]}}
\newcommand{\srlp}[2][\null]{{#2}^{\sorth[#1]}}
\newcommand{\lhmtp}{\mathrel{\sim_\ell}}
\newcommand{\rhmtp}{\mathrel{\sim_r}}
\newcommand{\hmtp}{\mathrel{\sim}}
\newcommand{\iso}{\cong}
\newcommand{\htpyeq}{\simeq}

%% macros dealing with Reedy stuff
\newcommand{\deginf}[2]{#1_{#2}}
\DeclareMathOperator{\latchingoperator}{L}
\newcommand{\latch}{\latchingoperator}
\newcommand{\latching}[1]{\latchingoperator_{#1}}
\DeclareMathOperator{\matchingoperator}{M}
\newcommand{\match}{\matchingoperator}
\newcommand{\matching}[1]{\matchingoperator_{#1}}

%% macros about set-theoretic stuff
\newcommand{\universe}[1]{\mathbb{#1}}
\newcommand{\powerset}[1]{\operatorname{\mathcal P}\left(#1\right)}

%% macros about type theoretic stuff
\newcommand{\theory}[1]{\mathbb{#1}}
\newcommand{\Var}{\mymathsf{Var}}
\newcommand{\entails}{\mathrel{\vdash}}
\newcommand{\defequal}{\mathrel{\equiv}}
\newcommand{\type}[1]{{#1}\ \mymathsf{type}}
\newcommand{\ctxt}[1]{{#1}\ \mymathsf{context}}
\newcommand{\subs}[3]{{#1}\left[#2\!\gets\!#3\right]}
\newcommand{\freevar}[1]{\mymathsf{fv}\left(#1\right)}
\newcommand{\alphabet}[1]{\mathcal{#1}}
\newcommand{\shift}[2][\null]{\mymathsf{pr}_{#2}^{#1}}
\newcommand{\var}{\mathrm{var}}
\newcommand{\idsymbol}{\mathrm{Id}}
\newcommand{\eqsymbol}{\mathrm{Eq}}
\newcommand{\sumsymbol}{\Sigma}
\newcommand{\prodsymbol}{\Pi}
\newcommand{\sumtype}[2]{\sumsymbol_{#1}#2}
\newcommand{\prodtype}[2]{\prodsymbol_{#1}#2}
\newcommand{\locprod}[2]{\prodsymbol_{#1}{#2}}
\newcommand{\idtype}[3][\null]{\operatorname{\idsymbol}_{#1}\left(#2,#3\right)}
\newcommand{\eqtype}[3][\null]{\operatorname{\eqsymbol}_{#1}\left(#2,#3\right)}
%\newcommand{\refl}[1]{\mymathsf{refl}_{#1}}
\newcommand{\jidoperator}{\mymathsf{j}}
\newcommand{\jid}[4]{\jidoperator(#1,#2,#3,#4)}
\newcommand{\transid}[2]{\mymathsf{trans}_{#1}\ifx#2\null\null\else(#2)\fi}
\newcommand{\name}[1]{\ulcorner{#1}\urcorner}
\newcommand{\recnat}[1]{\ifx#1\null\mymathsf{rec}\else\mymathsf{rec}(#1)\fi}
\newcommand{\sing}[1]{\mymathsf{isContr}(#1)}
\newcommand{\hequiv}[1]{\mymathsf{isEquiv}(#1)}
\newcommand{\idisequiv}[1]{\mymathsf{idIsEquiv}_{#1}}
\newcommand{\idtoeq}{\mymathsf{IdtoEq}}
\newcommand{\rewrite}[1][\null]{\mathrel{\rightsquigarrow}_{#1}}
%% semantics
\newcommand{\sem}[1]{\llbracket#1\rrbracket}
\newcommand{\domsem}[1]{\sem{#1}_0}
\newcommand{\codsem}[1]{\sem{#1}_1}
\newcommand{\loctribe}[2]{{#1}\,(#2)}
\newcommand{\locreltribe}[3]{\mathrm P_{#3}{#1}\,(#2)}

%% macros about general math
\DeclareMathOperator{\autoperator}{Aut}
%\newcommand{\aut}[1]{\autoperator\left(#1\right)}
\newcommand{\quotient}[2]{\left. {#1} \middle / {#2} \right.}
\newcommand*\from{:}
\newcommand*\cocolon{%
  \nobreak
  \mskip6mu plus1mu
  \mathpunct{}%
  \nonscript
  \mkern-\thinmuskip
  {:}%
  \mskip2mu
  \relax
}
\newcommand*\cofrom{:}
\DeclareMathOperator{\im}{im}
\newcommand{\blank}{\mathord{\color{lightgray}-}}
\newcommand{\Naturals}{\mathbb{N}}
%\newcommand{\doteq}[1][]{\mathrel{\cdot=_{#1}}}

%% macros about general typesetting
\newcommand{\define}[1]{\emph{#1}}
\newcommand{\showcase}[1]{\textbf{#1}}
\newcommand{\nb}[1]{\nobreakdash#1}

\newcommand{\mltt}{{\sc mltt}}%
\newcommand{\wfs}{\textsc{wfs}}

%% theorems typesetting
\newtheorem{theorem}{Theorem}[section]
\newtheorem*{theorem*}{Theorem}
\newtheorem*{theorem_french*}{Théorème}
\newtheorem{proposition}[theorem]{Proposition}
\newtheorem*{proposition*}{Proposition}
\newtheorem{lemma}[theorem]{Lemma}
\newtheorem{corollary}[theorem]{Corollary}
\newtheorem{claim}[theorem]{Claim}
\newtheorem*{claim*}{Claim}
\theoremstyle{definition}
\newtheorem{definition}[theorem]{Definition}
\newtheorem*{definition*}{Definition}
\newtheorem*{definition_french*}{Définition}
\newtheorem{defprop}[theorem]{Definition-Proposition}
\theoremstyle{remark}
\newtheorem{remark}[theorem]{\sc Remark}
\newtheorem{notation}[theorem]{\sc Notation}
\newtheorem{vocabulary}[theorem]{\sc Terminology}
\newtheorem{example}[theorem]{\sc Example(s)}
\newtheorem{cexample}[theorem]{\sc Counterexample(s)}
