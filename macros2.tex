%% cleveref setup
\crefname{lem}{lemma}{lemmas}
\crefname{prop}{proposition}{propositions}
\crefname{thm}{theorem}{theorems}
\crefname{cor}{corollary}{corollaries}
\crefname{rem}{remark}{remarks}
\crefname{exa}{example}{examples}

%% Macros
\let\oldequiv\equiv

\newcommand*{\from}{:}
\newcommand*{\blank}{\mathord{\color{lightgray}-}}
\newcommand*{\inv}[1]{{#1}^{-1}}

\newcommand*{\set}[1]{\{#1\}}
\newcommand*{\refl}[1]{\mathop{{\mathit{refl}}_{#1}}}
\newcommand*{\refloi}[1]{\mathop{\mathit{refl}^{-o}_{#1}}}
\newcommand*{\inl}{\mathop{\mathsf{inl}}}
\newcommand*{\inr}{\mathop{\mathsf{inr}}}
\newcommand*{\trp}[2][]{\mathop{\mathrm{trp}^{#1}_{#2}}}
\newcommand*{\fst}{\mathop{\mathit{fst}}}
\newcommand*{\snd}{\mathop{\mathit{snd}}}
\newcommand*{\funext}{\mathop{\mathrm{funext}}}
\newcommand*{\ptw}{\mathop{\mathrm{ptw}}}
\newcommand*{\ap}[1]{\left[{#1}\right]}
\newcommand*{\ptdto}{\to_\ast}%
\newcommand*{\ptdweq}{\weq_\ast}%
\newcommand*{\loopspace}[1]{\mathop{\Omega^{#1}}}%

\newcommand*{\nat}{\mathord{\mathrm{nat}}}
\newcommand*{\ltr}{\mathord{\mathrm{ltr}}}
\newcommand*{\rtr}{\mathord{\mathrm{rtr}}}

\newcommand*{\id}{\mathord{\mathrm{id}}}
\newcommand*{\pt}{{\mathord{\mathrm{pt}}}}
\newcommand*{\Type}{\mathord{\mathrm{Type}}}
\newcommand*{\Prop}{\mathord{\mathrm{Prop}}}
\newcommand*{\Set}{\mathord{\mathrm{Set}}}
\newcommand*{\cat}[1]{\mathscr{#1}}

\DeclareMathOperator\Bop{B}         % with extra space
\newcommand*{\B}{\mathrm B}          % without extra space
\newcommand*{\N}{\mathrm N}
\newcommand*{\weq}{\simeq}
\newcommand*{\QQ}{\mathbb{Q}}
\newcommand*{\ZZ}{\mathbb{Z}}
\newcommand*{\NN}{\mathbb{N}}
\newcommand*{\CC}{\mathbb{C}}
\newcommand*{\RR}{\mathbb{R}}
\newcommand*{\isom}{\cong}
\newcommand*{\ct}{*}
\newcommand*{\cto}{*_{\mathrm{o}}}
\newcommand*{\rrfl}{\mathit{rrfl}}
\newcommand*{\cp}[1]{\mathit{cp}_{#1}}
\newcommand*{\dblslash}{\mathbin{/\kern-3pt/}}

%\renewcommand{\equiv}{\simeq}
%\newcommand*{\liff}{\equiv}
\newcommand*{\jdeq}{\equiv}
%\newcommand*{\defeq}{\mathrel{\hbox{:}\mkern-5mu\equiv}}
\newcommand*{\defeq}{\vcentcolon\jdeq}
\newcommand*{\defequi}{\defeq}%definitionally equal}
\newcommand*{\defis}{\vcentcolon=}
% \newcommand*{\defis}{\mathrel{\hbox{:}\mkern-5mu=}}
\newcommand*{\leftadjto}{\dashv}

\DeclareMathOperator\im{im}
\DeclareMathOperator\Aut{Aut}
\DeclareMathOperator\Out{Out}
\DeclareMathOperator\Inn{Inn}
\DeclareMathOperator\Ker{Ker}

\DeclarePairedDelimiter\Trunc{\lVert}{\rVert}
\DeclarePairedDelimiter\trunc{\lvert}{\rvert} % truncation
\DeclarePairedDelimiter\angled{\langle}{\rangle}
\DeclarePairedDelimiterX\setof[2]\lbrace\rbrace{#1 \mid #2}

\newcommand*{\nonempty}[1]{\Trunc{#1}}
\newcommand*{\setTrunc}[1]{\Trunc{#1}_0}
\newcommand*{\settrunc}[1]{\trunc{#1}_0}

\DeclarePairedDelimiterXPP\higherTrunc[2]{}\lVert\rVert{_{#1}}{#2}
\DeclarePairedDelimiterXPP\highertrunc[2]{}\lvert\rvert{_{#1}}{#2}

\newcommand*{\merely}[1]{\higherTrunc{-1}{#1}}

%%%%%%%%%%%%%%%%%%%%%%%%%%%%%%%%%%%%%%%%%%%%%%%%%%%%%%%%%%%%%%%%%%%%%%%%%%%%
\newcommand*{\ie}{{\em i.e.,\xspace}}%\xspace}fixlater
\newcommand*{\eg}{{\em e.g.,\xspace}}%\xspace}fixlater
\newcommand*{\ev}{\mathrm{ev}}
\newcommand*{\ve}{\mathrm{ve}}
\newcommand*{\el}{\mathrm{elim}}
\newcommand*{\we}{\overset\sim\to}

\newcommand*{\iscontr}{\mathrm{isContr}}
\newcommand*{\isprop}{\mathrm{isProp}}
\newcommand*{\isset}{\mathrm{isSet}}
\newcommand*{\isgrpd}{\mathrm{isGrpd}}
\newcommand*{\isEq}{\mathrm{isEquiv}}
\newcommand*{\isonetype}{\mathrm{1Type}}
\newcommand*{\isconn}{\mathrm{isConn}}

\newcommand*{\conn}{\mathrm{conn}}
\newcommand*{\conncomp}[2]{{#1}_{\left(#2\right)}}
\newcommand*{\aut}{\mathrm{Aut}}
\newcommand*{\Hom}{\mathrm{Hom}}
\newcommand*{\setgroup}[1]{||#1||}

\newcommand*{\UU}{\mathcal{U}}
\newcommand*{\UUp}{\UU_*}
\newcommand*{\UUptd}{\UUp}
\newcommand*{\pttype}{\UUp}

\newcommand*{\circled}[1]{\textrm{\textcircled{\small #1}}}

\newcommand*{\hopf}{{\mathcal H}}
\newcommand*{\Sn}[1]{{\mathbb S^{#1}}}% general sphere
\newcommand*{\Sc}{{\Sn 1}}%the circle
\newcommand*{\Sp}{{\Sn 2}}%the 2-sphere
\newcommand*{\sbt}{\tikz[baseline]{\node[scale=.7,inner
    sep=0, outer sep=0, circle, anchor=base, yshift=.05ex]%
    {$\bullet$};}}%
\newcommand*{\base}{\mathord{\sbt}}%point in circle
\newcommand*{\uc}[1]{{I_{#1}}}%universal set bundle
\newcommand*{\Sloop}{\mathord{\circlearrowleft}}%loop in circle
\newcommand*{\bn}[1]{\mathbf{#1}}
\newcommand*{\Eq}{\mathrm{Eq}}
\newcommand*{\emptytype}{\emptyset}
\newcommand*{\istrunc}[1]{\mathsf{istrunc}_{#1}}
\newcommand*{\ind}{\operatorname{\mathsf{ind}}}
\newcommand*{\susp}{\operatorname{\Sigma}}
\newcommand*{\mrd}{\operatorname{\mathsf{mrd}}}
\newcommand*{\glue}{\operatorname{\mathsf{glue}}}
\newcommand*{\hcomp}{\mathbin{\cdot_h}}

\newcommand*{\etop}[1]{\bar {#1}}
\newcommand*{\ptoe}[1]{\tilde {#1}}
\newcommand*{\cast}{\mathrm{cast}}
\newcommand*{\ua}{\mathrm{ua}}%univalence inverse

%% fundamental group
\newcommand*{\hgr}[1]{\uppi_{#1}}
\newcommand*{\fgr}{\hgr 1}

%% paths over paths

%% \newcommand*{\pathover}[4]{#1 \overset{#2}{\underset{#3}=} #4}
\newcommand*{\pathoverdisplay}[4]{{#1} \overset{#2}{\underset{#3}=} #4}
\newcommand*{\pathover}[4]{#1 =^{#2}_{#3} #4}

\newcommand*{\pair}{\mathop{\mathrm{pair}}}
\newcommand*{\rec}{\mathop{\mathrm{rec}}}
\newcommand*{\pathpair}[2]{\overline{({#1},{#2})}}

%% global tikz styles
\tikzset{cell/.style={%
    shorten <=1em,%
    shorten >=1em,
    /tikz/commutative diagrams/Rightarrow
  }%
}%
\tikzset{
tikzshortarrow/.style={
    shorten >=0.2cm,
    shorten <=0.2cm,
    thick,
  }
}
\tikzset{pushout/.style={%
    commutative diagrams/.cd,dr,phantom,"\ulcorner", very near end
  }%
}
\tikzset{
    rotninety/.style={anchor=south, rotate=90, inner sep=2pt}
}
\usetikzlibrary{arrows}
