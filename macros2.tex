%% to get the correct cleveref functionality
%% we redefine the thm enviroments from lmcs.cls
\newcommand{\renewtheorem}[1]{%
  \expandafter\let\csname #1\endcsname\relax
  \expandafter\let\csname c@#1\endcsname\relax
  \expandafter\let\csname end#1\endcsname\relax
  \newtheorem{#1}%
}

\theoremstyle{plain}

\renewtheorem{thm}{Theorem}[section]
\crefname{thm}{Theorem}{Theorems}
\renewtheorem{cor}[thm]{Corollary}
\crefname{cor}{Corollary}{Corollaries}
\renewtheorem{lem}[thm]{Lemma}
\crefname{lem}{Lemma}{Lemmas}
\renewtheorem{prop}[thm]{Proposition}
\crefname{prop}{Proposition}{Propositions}

\theoremstyle{definition}

\renewtheorem{rem}[thm]{Remark}
\crefname{rem}{Remark}{Remarks}
\renewtheorem{exa}[thm]{Example}
\crefname{exa}{Example}{Examples}
\renewtheorem{defi}[thm]{Definition}
\crefname{defi}{Definition}{Definitions}

% Meta-macros for:
\newcommand*{\constant}[1]{\mathrm{#1}}% defined constants : roman
\newcommand*{\constructor}[1]{\mathrm{#1}} % constructors : italic
\newcommand*{\typeformer}[1]{\mathrm{#1}} % (uppercase) typeformers : roman

%% Macros
\let\oldequiv\equiv

\renewcommand*{\th}{\textsuperscript{th}}
\newcommand*{\from}{:}
\newcommand*{\blank}{\mathord{{-}}}%{\color{lightgray}-}}
\newcommand*{\inv}[1]{#1^{-1}}

\newcommand*{\set}[1]{\{#1\}}
\newcommand*{\refl}[1]{\mathop{{\constructor{refl}}_{#1}}}
\newcommand*{\refloi}[1]{\mathop{\constructor{refl}^{-o}_{#1}}}
\newcommand*{\inl}{\mathop{\constructor{inl}}}
\newcommand*{\inr}{\mathop{\constructor{inr}}}
\newcommand*{\mrd}{\operatorname{\constructor{mrd}}}
\newcommand*{\glue}{\operatorname{\constructor{glue}}}
\newcommand*{\north}{\constructor{N}}
\newcommand*{\south}{\constructor{S}}
\newcommand*{\trp}[2][]{\mathop{\constant{trp}^{#1}_{#2}}}
\newcommand*{\fst}{\mathop{\constant{fst}}}
\newcommand*{\snd}{\mathop{\constant{snd}}}
\newcommand*{\funext}{\mathop{\constant{funext}}}
\newcommand*{\ptw}{\mathop{\constant{ptw}}}
\newcommand*{\ap}[1]{\mathop{\left[{#1}\right]}}
\newcommand*{\ptdto}{\to_\ast}%
\newcommand*{\ptdweq}{\weq_\ast}%
\newcommand*{\loopspace}[1]{\mathop{\Omega^{#1}}}%

\newcommand*{\nat}{\mathord{\constant{nat}}}
\newcommand*{\ltr}{\mathord{\constant{ltr}}}
\newcommand*{\rtr}{\mathord{\constant{rtr}}}

\newcommand*{\Id}{\mathord{\constant{Id}}}
\newcommand*{\id}{\mathord{\constant{id}}}
\newcommand*{\cat}[1]{\mathscr{#1}}

\newcommand*{\weq}{\simeq}
\newcommand*{\QQ}{\mathbb{Q}}
\newcommand*{\ZZ}{\mathbb{Z}}
\newcommand*{\NN}{\mathbb{N}}
\newcommand*{\CC}{\mathbb{C}}
\newcommand*{\RR}{\mathbb{R}}
\newcommand*{\isom}{\cong}
\newcommand*{\ct}{*}
\newcommand*{\cto}{*_{\mathrm{o}}}

%\renewcommand{\equiv}{\simeq}
%\newcommand*{\liff}{\equiv}
\newcommand*{\jdeq}{\equiv}
%\newcommand*{\defeq}{\mathrel{\hbox{:}\mkern-5mu\equiv}}
\newcommand*{\defeq}{\vcentcolon\jdeq}
\newcommand*{\defequi}{\defeq}%definitionally equal}
\newcommand*{\defis}{\vcentcolon=}
% \newcommand*{\defis}{\mathrel{\hbox{:}\mkern-5mu=}}
\newcommand*{\leftadjto}{\dashv}

\DeclareMathOperator\im{im}
\DeclareMathOperator\Aut{Aut}

\DeclarePairedDelimiter\Trunc{\lVert}{\rVert}
\DeclarePairedDelimiter\trunc{\lvert}{\rvert} % truncation
\DeclarePairedDelimiter\angled{\langle}{\rangle}
\DeclarePairedDelimiterX\setof[2]\lbrace\rbrace{#1 \mid #2}

\newcommand*{\nonempty}[1]{\Trunc{#1}}
\newcommand*{\setTrunc}[1]{\Trunc{#1}_0}
\newcommand*{\settrunc}[1]{\trunc{#1}_0}

\DeclarePairedDelimiterXPP\higherTrunc[2]{}\lVert\rVert{_{#1}}{#2}
\DeclarePairedDelimiterXPP\highertrunc[2]{}\lvert\rvert{_{#1}}{#2}

\newcommand*{\merely}[1]{\higherTrunc{-1}{#1}}

%%%%%%%%%%%%%%%%%%%%%%%%%%%%%%%%%%%%%%%%%%%%%%%%%%%%%%%%%%%%%%%%%%%%%%%%%%%%
\newcommand*{\ev}{\constant{ev}}
\newcommand*{\ve}{\constant{ve}}
\newcommand*{\el}{\constant{elim}}

\newcommand*{\iscontr}{\constant{isContr}}
\newcommand*{\isprop}{\constant{isProp}}
\newcommand*{\isset}{\constant{isSet}}
\newcommand*{\isgrpd}{\constant{isGrpd}}
\newcommand*{\isequiv}{\constant{isEquiv}}
\newcommand*{\isonetype}{\constant{1Type}}
\newcommand*{\isconn}{\constant{isConn}}
\newcommand*{\istrunc}[1]{\constant{isTrunc}_{#1}}

\newcommand*{\conncomp}[2]{{#1}_{\left(#2\right)}}

\newcommand*{\UU}{\mathcal{U}}
\newcommand*{\UUp}{\UU_*}
\newcommand*{\UUptd}{\UUp}
\newcommand*{\pttype}{\UUp}

\newcommand*{\circled}[1]{\textrm{\textcircled{\small #1}}}

\newcommand*{\hopf}{{\mathcal H}}
\newcommand*{\Sn}[1]{{\mathbb S^{#1}}}% general sphere
\newcommand*{\Sc}{{\Sn 1}}%the circle
\newcommand*{\Sp}{{\Sn 2}}%the 2-sphere
\newcommand*{\sbt}{\tikz[baseline]{\node[scale=.7,inner
    sep=0, outer sep=0, circle, anchor=base, yshift=.05ex]%
    {$\bullet$};}}%
\newcommand*{\base}{\mathord{\sbt}}%point in circle
\newcommand*{\uc}[1]{{I_{#1}}}%universal set bundle
\newcommand*{\Sloop}{\mathord{\circlearrowleft}}%loop in circle
\newcommand*{\bn}[1]{\mathbf{#1}}
\newcommand*{\emptytype}{\emptyset}
\newcommand*{\ind}{\operatorname{\constant{ind}}}
\newcommand*{\susp}{\operatorname{\Sigma}}
\newcommand*{\hcomp}{\mathbin{\cdot}} % was : \cdot_h

\newcommand*{\etop}[1]{\bar {#1}}
\newcommand*{\ptoe}[1]{\tilde {#1}}

%% fundamental group
\newcommand*{\hgr}[1]{\uppi_{#1}}
\newcommand*{\fgr}{\hgr 1}

%% paths over paths

%% \newcommand*{\pathover}[4]{#1 \overset{#2}{\underset{#3}=} #4}
\newcommand*{\pathoverdisplay}[4]{{#1} \overset{#2}{\underset{#3}=} #4}
\newcommand*{\pathover}[4]{#1 =^{#2}_{#3} #4}

%% global tikz styles
\usetikzlibrary{arrows}

\tikzset{cell/.style={%
    shorten <=1em,%
    shorten >=1em,
    /tikz/commutative diagrams/Rightarrow
  }%
}%
\tikzset{
tikzshortarrow/.style={
    shorten >=0.2cm,
    shorten <=0.2cm,
    thick,
  }
}
\tikzset{pushout/.style={%
    commutative diagrams/.cd,dr,phantom,"\ulcorner", very near end
  }%
}
\tikzset{
    rotninety/.style={anchor=south, rotate=90, inner sep=2pt}
}
\tikzset{
    rottwoseventy/.style={anchor=north, rotate=90, inner sep=2pt}
}

\definecolor{darkgreen}{rgb}{0,0.4,0}
\definecolor{darkblue}{rgb}{0,0,1}
\definecolor{darkred}{rgb}{.7,0,0}

\tikzset{
	tikzforeground/.style={
		->,
%		shorten >=0.2cm,
%		shorten <=0.2cm,
		line width = 0.8pt,
		preaction={draw, -, line width=5pt, white},
	},
	tikzbackground/.style={
		->,
%		shorten >=0.2cm,
%		shorten <=0.2cm,
		line width = 0.4pt,
	},
%	tikzmiddle/.style={
%		->,
%		shorten >=0.2cm,
%		shorten <=0.2cm,
%		line width = .8pt,
%		preaction={draw, -, line width=3pt, white},
%	},
	tikzequal/.style={
		-,
		double,
		shorten >=0.2cm,
		shorten <=0.2cm,
		line width = 0.6pt,
		preaction={draw, -, line width=3pt, white},
	},
}
