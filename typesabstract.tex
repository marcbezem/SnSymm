\documentclass{easychair}

%\usepackage{fontspec}
\usepackage[utf8]{inputenc}
\usepackage{amsmath,amsthm,amssymb}
%\usepackage{unicode-math}
%\usepackage{polyglossia}
\usepackage{enumitem}
\usepackage{tikz,tikz-cd}
\usepackage{hyperref}
\usepackage{microtype}
%\usepackage[explicit]{titlesec}
\usepackage[mathscr]{euscript}
\usepackage{mathtools}
\usepackage{titling}

\input macros%
\input macros2%
% more macros
\newcommand{\Sp}{{S^2}}%
\newcommand{\Sn}{{S^n}}%
\newcommand{\mrd}{\mathit{merid}}%
\newcommand{\susp}[1]{\Sigma{#1}}%
\newcommand{\Truncset}[1]{{\Trunc{#1}}_0}%
\newcommand{\truncset}[1]{{\trunc{#1}}_0}%
\newcommand{\Truncprop}[1]{\Trunc{#1}}%
\newcommand{\truncprop}[1]{\trunc{#1}}%
\newcommand{\ptdto}{\to_\ast}%
\newcommand{\flip}{\operatorname{flip}}%
\newcommand{\conncomp}[2]{{#1}_{(#2)}}%
%

%% REMOVE FOR NORMAL BIBLIO
\makeatletter
\renewenvironment{thebibliography}[1]{%
  \list{\@biblabel{\@arabic\c@enumiv}}%
  {\settowidth\labelwidth{\@biblabel{#1}}%
    \leftmargin\labelwidth
    \advance\leftmargin\labelsep
    \@openbib@code
    \usecounter{enumiv}%
    \let\p@enumiv\@empty
    \renewcommand\theenumiv{\@arabic\c@enumiv}}%
  \sloppy
  \clubpenalty4000
  \@clubpenalty \clubpenalty
  \widowpenalty4000%
  \sfcode`\.\@m}
{\def\@noitemerr
  {\@latex@warning{Empty `thebibliography' environment}}%
  \endlist}
\makeatother
%%%% 


\hypersetup{
    colorlinks,
    linkcolor={red!50!black},
    citecolor={blue!50!black},
    urlcolor={blue!80!black}
}

\setlist[enumerate]{label=(\roman*)}

\title{On Symmetries of Spheres in HoTT-UF}%
\author{Pierre Cagne\thanks{Universitetet i Bergen} \and Nicolai
  Kraus\thanks{University of Birmingham} \and Marc
  Bezem\footnotemark[1]{}}%
\institute{
  % University of Birmingham
  % \and
  % University of Bergen
}
\titlerunning{\thetitle}
\authorrunning{\theauthor}


% \date{\normalsize Last updated on \today}%

%% global tikz styles
\tikzset{cell/.style={%
    shorten <=1em,%
    shorten >=1em,
    /tikz/commutative diagrams/Rightarrow
  }%
}%

\def\U{\universe U}%
\def\blank{\_}%
\def\hopffam{\mathcal H}

\begin{document}

\maketitle

The goal of this talk is to give insights in the symmetries of the
$n$-sphere in synthetic homotopy theory, i.e.\ the type $S^n = S^n$.

We work in intuitionistic Martin-Löf's type theory with $\Sigma$-,
$\Pi$- and $\mathrm{Id}$-types and with a cumulative hierarchy of
universes, simply written $\U$, for which Voevodsky's univalence axiom
hold.  A good reference for our setting is the HoTT Book \cite{HoTT},
to which we will refer frequently. We specifically use the following
higher inductive types (HITs): the propositional truncation
$\Trunc{A}$ of an arbitrary type $A$ (\cite[Ch.~3.7]{HoTT}); the set
truncation $\Trunc{A}_0$ of an arbitrary type $A$
(\cite[Ch.~6.9]{HoTT}); the circle $\Sc$ (\cite[Ch.~6.4]{HoTT}); the
suspension $\susp A$ of an arbitrary type $A$ (\cite[Ch.~6.5]{HoTT}).

% Any HIT comes with its induction principle capturing the essence of the HIT,
% and computation rules explaining how to simplify certain expressions.
% The essence of the propositional truncation is a faithful way of turning
% a type $A$ into a proposition (a type with all elements equal).
% The essence of the set truncation is a faithful way of turning
% a type $A$ into a set (all types $a=_Aa'$ propositions). 

We give some more details of the circle and the suspension, as they
are crucial for our presentation. The circle is a higher inductive
type with one point constructor and one path constructor:
\begin{align*}
  \base &: \Sc,  \quad\text{the base point}\\
  \Sloop &: \base =_{\Sc} \base,  \quad\text{the loop}
\end{align*}
The circle comes with an elimination rule such that functions
$\Sc \to A$ correspond to pairs of an element $a:A$ and a path
$\ell: a=a$ for any type $A$. The first part of the talk will be
dedicated to showing that
\begin{equation}
  \label{eq:s1-result}%
  (\Sc = \Sc) \weq (\Sc+\Sc).
\end{equation}

For $n\geq 2$, the $n$-sphere $S^n$ is inductively defined as the
suspension $\susp {(S^{n-1})}$. The suspension $\susp A$ of a type $A$
is a higher inductive type with two point constructors and path
constructors indexed by $A$:
\begin{align*}
  \label{eq:suspension}
  N &: \susp A, \quad\text{the `North pole'}\\
  S &: \susp A, \quad\text{the `South pole'}\\
  \mrd &: A \to (N =_{\susp A} S), \quad\text{the `meridians'}
\end{align*}
The elimination principle for $\susp A$ gives again a correspondence
between the type $\susp A \to B$ and the type of triplets consisting
of $b_N: B$, $b_S: B$ and $m: A \to b_N=b_S$.

One cannot expect (\ref{eq:s1-result}) to generalise to higher
dimensions. However, (\ref{eq:s1-result}) implies that $\Sc = \Sc$
consists of two equivalent connected components. Modulo univalence,
one component contains $\id_\Sc$ and the other $-\id_\Sc$. The latter
is the function $\Sc \to \Sc$ corresponding to the pair $\base:\Sc$
together with the path $\inv\Sloop:\base=\base$. This weaker statement
does in fact generalise to higher spheres in the homotopy theory of
topological spaces. In the talk, we will elaborate a proof in HoTT-UF
for the case $n=2$, following the following lines.

The first step is to define $-\id_{\susp A}$ for any type $A$ as the
function corresponding to the triplet $S:\susp A$, $N:\susp A$ and
$\inv{\mrd(\blank)} : (A \to S=N)$. In other words, $-\id_{\susp A}$
flips the poles and reverses each meridian. Notice that
$-\id_{\susp A}$ is an equivalence, as it is its own
pseudo-inverse. The function
\begin{displaymath}
  \flip_A \defequi \blank \circ -\id_{\susp A} : (\susp A \to \susp A) \to (\susp A \to \susp A)
\end{displaymath}
is then an equivalence, hence establishing an equivalence from the
connected component at $\id_{\susp A}$ to the connected component at
$-\id_{\susp A}$. Notice that $-\id_{\susp A}$ is not necessarily
distinct from $\id_{\susp A}$ (take for example $A\defequi 1$ for
which $\susp A$ is contractible). In particular, it is non-trivial to
prove that $-\id_{\Sp} \neq \id_{\Sp}$.

As the sphere $\Sp$ is connected, in proving the proposition
$\Trunc{\varphi=\id_\Sp}+\Trunc{\varphi=-\id_\Sp}$ for an equivalence
$\varphi:\Sp \weq \Sp$, one can as well suppose that $\varphi$ is a
pointed map by a path $\varphi_0:N=\varphi(N)$. It is worth computing
the degree of such a $\varphi$. Recall that the degree $d(f,f_0)$ of a
pointed function $(f,f_0):\Sp \ptdto\Sp$ is defined as the integer
$\bar \pi_2(f,f_0)(1)$ where $\bar\pi_2(f,f_0)$ is the group morphism
$\pi_2(f,f_0) : \pi_2(\Sp) \to \pi_2(\Sp)$ viewed through the
equivalence $\pi_2(\Sp) \weq \ZZ$.

For example, for each $k:\ZZ$, the following map
$\delta_k : \Sp \ptdto \Sp$ has degree $k$: first define
$c_k:\Sc\to\Sc$ as the map that corresponds to the pair $\base:\Sc$
together with the path $\Sloop^k:\base=\base$; then define $\delta_k$
as the map corresponding to the triplet $N:\Sp$, $S:\Sp$ and
$\mrd{}\circ c_k: \Sc \to N=S$, obviously pointed by the path
$\refl N:N=\delta_k(N)$.  It is easy to see that
$\delta_1 = \id_{\Sp}$ and one can prove that
$\delta_{-1} = -\id_{\Sp}$ also.  Proving that $\delta_k$ has indeed
degree $k$ is non-trivial, and we shall exhibit a proof using the Hopf
fibration.

Using the functoriality of $\pi_2$, one gets
\begin{displaymath}
  d((g,g_0)\circ (f,f_0)) = \bar\pi_2(g,g_0)\left( \bar\pi_2(f,f_0)(1)\right)
  = \bar\pi_2(g,g_0)(1)\times\bar\pi_2(f,f_0)(1).
\end{displaymath}
The last identity comes from the fact that $\bar\pi_2(g,g_0)$ is a
group morphism $\ZZ\to\ZZ$. In other words, $d$ is a morphism of
monoids, and as such, it maps equivalences to invertible elements of
$\ZZ$. Hence, $d(\varphi,\varphi_0) = \pm 1$. The last step is to
prove that having the same degree means precisely being in the same
connected component of $\Sp \ptdto \Sp$. In order to do so, we shall
give an alternate description of the degree, based on the Hopf
fibration, and on the proof that $\pi_2(\Sp) \weq \ZZ$ (cf.\
\cite[Ch.~8.6]{HoTT}).

To generalize this proof to the case $n>2$, one can rely on the
Freudenthal suspension theorem (\cite[Ch.~8.6]{HoTT}). If time
permits, we will sketch how it allows to prove that $\Sn \ptdto \Sn$
has exactly two connected components, and we shall then detail the
work in progress in order to get the full property.

Future works include formalizing this proof in cubical type theory
(CTT) and experimenting with actual computation of the degree of
selected symmetries. This is one of the motivations for this
work. Indeed, the univalence axiom is necessary for the definition of
the Hopf fibration and for the Freudenthal suspension theorem (and
hence for the definition of the degree), and an implementation in CTT
will display the computational content of univalence at
work. Hopefully, this would pose as a reasonable computational
challenge, simpler than the current problems tackled through CTT such
as the computation of Brunerie's number (cf. \cite{brunerie:thesis}).

\bibliographystyle{alpha}
\bibliography{bib}

\end{document}

\section{The circle}
\label{sec:circle-case}%
In this document, we shall establish an equivalence
between the type of equalities of $\Sc$ with itself and the sum type
of $\Sc$ with itself. 
This type enjoys a inductive property: given a type family
$T : \Sc \to \U$, an element $t:T(\base)$ and a path
$p:t=_{T(\base)}t$, the $\Sc$-induction property provides a dependent
function $f : \prod_{x:\Sc}T(x)$ such that $f(\base) \jdeq t$ and an
element $q: f(\Sloop) = p$. In particular, $\Sc$-induction provides a
map
\begin{equation}
  \label{eq:2}
  \left(\sum_{t:T(\base)} \pathover t T \Sloop t\right) \to \prod_{x:\Sc}T(x)
\end{equation}
that is easily shown to be an equivalence (through $\Sc$-induction
again). In the case $T$ is the family constant in $A:\U$, it gives the
following universal property of the circle:
\begin{equation}
  \label{eq:3}
  \left(\sum_{t:A}t=t\right) \weq \left( \Sc \to A \right)
\end{equation}

Using univalence, one gets that $(\Sc = \Sc)$ is
equivalent to $(\Sc \weq \Sc)$ and we shall then concentrate on
proving
\begin{equation}
  (\Sc \weq \Sc) \weq (\Sc+\Sc)
\end{equation}
As usual, we shall denote $f:\Sc\weq\Sc$ for a map $f:\Sc\to \Sc $
such that the proposition $\isEq(f)$ is non-empty instead of the more
honest $(f,!):\Sc \weq \Sc$. 
This is harmless as the first projection is an equivalence
\begin{equation}
  ((f,!) =_{\Sc \weq \Sc} (g,!)) \simeq (f=_{\Sc \to \Sc}g).
\end{equation}

Recall that for each $x:\Sc$ one has an equivalence:
\begin{equation}
  \label{eq:8}
  f_x : (\base = \base) \to (x=x)
\end{equation}
Indeed, such a map $x\mapsto f_x$ is defined by $\Sc$-induction by
setting $f_{\base} \defequi \id$ and $f_{\Sloop}$ to be the proof that
postcomposing with the conjugation $\Sloop \blank \inv\Sloop$ is the
identity. In particular one can easily prove that
$f_x(\Sloop^k) = f_x(\Sloop)^k$ for any $k:\zet$. Now define $\iota_1$
and $\iota_2$ on each $x:\Sc$ as follows:
\begin{align*}
  \iota_1(x) (\base) &\defequi x & \iota_2(x) (\base) &\defequi x,
  \\
  \iota_1(x)(\Sloop) &\defis f_x(\Sloop) &  \iota_2(x)(\Sloop) &\defis f_x(\Sloop)^{-1}
\end{align*}
Each $\iota_j(x)$ ($j\in\{1,2\}$) is indeed an equivalence: $\Sc$
being connected, the proposition $\isEq(\iota_j(x))$ has to be proved
for only one point, say $\base$; but $\iota_1(\base)$ is equal to
$\id_\Sc$ (hence an equivalence) and $\iota_2(\base)$, that we call
$-\id_\Sc$ in the following, is its own pseudo inverse.

The maps $\iota_1$ and $\iota_2$ induce a maps
$\iota \defequi \langle\iota_1,\iota_2\rangle : \Sc + \Sc \to (\Sc\weq
\Sc)$.
\begin{theorem*}
  The map $\iota$ is an equivalence.
\end{theorem*}
%
Before proving the theorem, here is a lemma that already explains that
the two components of $\Sc+\Sc$ are mapped to distinct parts of
$(\Sc \weq \Sc)$.
\begin{lemma}
  \label{lemma:id-not-equal-oppid}%
  The type $(\id_\Sc = -\id_\Sc)$ is empty.
\end{lemma}
\begin{proof}
  By function extensionality and $\Sc$-induction, an element
  $p : \id_\Sc = -\id_\Sc$ is given by the data of
  $p(\base) : \id_\Sc(\base) = -\id_\Sc(base)$, that is
  $p(\base):\base=\base$, together with a pathover
  $p(\Sloop): \pathover {p(\base)} {} \Sloop {p(\base)}$ in the type
  family $x\mapsto \id_\Sc (x)=-\id_\Sc (x)$. The transport over
  $\Sloop$ in this type family is given by
  $\inv\Sloop\blank\inv\Sloop$, so that $p(\Sloop)$ is equivalently
  given by an element of $\inv\Sloop p(\base) \inv\Sloop =
  p(\base)$. By expliciting $p(\base)$ as $\Sloop^k$ for some
  $k:\zet$, the last equation becomes
  $\inv \Sloop \Sloop^k \inv \Sloop = \Sloop^k$, leading to the
  contradiction $2=0$ in $\zet$.
\end{proof}

\begin{proof}[Proof of the theorem]
  Take an element $\phi : \Sc \weq \Sc$ and consider the fiber of
  $\iota$ at $\phi$:
  \begin{equation}
    \label{eq:fiber-iota-two-components}%
    \inv\iota(\phi) \jdeq \sum_{x:\Sc+\Sc} \phi = \iota(x)
    \weq \left(\sum_{x:\Sc}\phi = \iota_1(x)\right) +
    \left(\sum_{x:\Sc}\phi = \iota_2(x)\right)
  \end{equation}
  One wants to prove the proposition $\iscontr(\inv\iota(\phi))$,
  hence one may assume a path $q:\phi(\base) = \base$ instead of
  merely $\Trunc{\phi(\base) = \base}$. By the description of the
  symmetries of $\base:\Sc$, there is $k:\zet$ such that
  $q\phi(\Sloop)\inv q = \Sloop^k$. Denoting $\psi$ for a pseudo
  inverse of the equivalence $\phi$, one also gets
  $q':\psi(\base) = \base$ and subsequently $\ell:\zet$ such that
  $q'\psi(\Sloop)\inv{q'} = \Sloop^\ell$. Doing some path algebra it
  follows that:
  \begin{equation}
    \label{eq:4}
    \psi(\phi(\Sloop)) = \inv{(q' \psi(q))}\Sloop^{\ell k} (q' \psi(q))
  \end{equation}
  By hypothesis, one has also $\tau : \prod_{x:\Sc}\psi\phi(x) = x$,
  so there is a pathover of type
  $\pathover {\tau(\base)} {} \Sloop {\tau(\base)}$ in the type family
  $x \mapsto \psi(\phi(x)) = x$. The transport over $\Sloop$ in this
  family is given by $\Sloop \blank \inv{(\psi\phi(\Sloop))}$, so that
  \begin{equation}
    \label{eq:5}
    \Sloop \tau(\base) \inv{(\psi\phi(\Sloop))} = \tau(\base)
  \end{equation}
  Putting equations~(\ref{eq:5}) and~(\ref{eq:4}) together, one find
  \begin{equation}
    \label{eq:6}
    \inv{(q' \psi(q) \inv{\tau(\base)})} \Sloop^{\ell k} (q' \psi(q) \inv{\tau(\base)}) = \Sloop
  \end{equation}
  However, conjugation in $\base = \base$ is equal to the identity, so
  that in the end $\Sloop^{\ell k} = \Sloop$, or otherwise put
  $\ell k = 1$ in $\zet$. Using the decidability of equality on
  $\zet$, one can show that the only invertible elements of $\zet$ are
  $1$ and $-1$. So $k = 1$ or $k = -1$. We proceed by case:
  \begin{enumerate}
  \item If $k=1$, then $Q \defequi (q, !)$ is element of
    $\phi = \id_\Sc$, where the term $!$ of the proposition
    $q=\Sloop q \inv{\phi(\Sloop)}$ is given by path algebra from the
    hypothesis $q\phi(\Sloop)\inv q=\Sloop^k$ ($k$ being $1$ here). We
    can then only show $\iscontr(\inv\iota(\id_\Sc))$ to conclude (by
    transport over $\inv Q$) that $\iscontr(\inv\iota(\phi))$. 

    One can immediately see that the second component of the right
    hand-side in equation~(\ref{eq:fiber-iota-two-components}) is
    empty when instantiated with $\id_\Sc$ for $\phi$: indeed,
    $\false$ being a proposition and $\Sc$ being connected, one only
    need to check that $\id_\Sc = \iota_2(\base)$ is contradictory;
    and because $\iota_2(\base) \jdeq -\id_\Sc$, it follows from
    lemma~\ref{lemma:id-not-equal-oppid}.

    Hence it remains to prove that $\sum_{x:\Sc}\id_\Sc=\iota_1(x)$ is
    contractible. For $x:\Sc$, the type $\id_\Sc = \iota_1(x)$ is
    equivalent (through first function extensionality and then
    $\Sc$-induction) to
    $\sum_{p:\base=\iota_1(x)(\base)}\pathover p T \Sloop p$ where $T$
    is the type family $y\mapsto y=\iota_1(x)(y)$. Now
    $\iota_1(x)(\base)\jdeq x$ and the transport over the loop
    $\Sloop$ in the type family $T$ is given by
    $\iota_1(x)(\Sloop) \blank \inv \Sloop$. In other words,
    \begin{equation}
      \label{eq:10}
      (\id_\Sc = \iota_1(x)) \weq \sum_{p:\base = x}f_x(\Sloop)p = p {\Sloop}
    \end{equation}
    By induction on $p:\base=x$, one can see that the proposition
    $f_x(\Sloop)p=p\Sloop$ is always satisfied. In the end, one has
    \begin{equation}
      \label{eq:11}
      \left(\sum_{x:\Sc}\id_\Sc=\iota_1(x)\right)
      \weq
      \left(\sum_{x:\Sc}\base=x\right)
    \end{equation}
    The type on the right is a singleton, hence contractible. This
    concludes the case $k=1$.
      \item The case $k=-1$ is completely similar to the case $k=1$. We
    now have a proof that $\phi = -\id_\Sc$, hence we shall prove that
    $\iscontr(\inv\iota(-\id_\Sc))$. From there one can determine that
    the first component of in (\ref{eq:fiber-iota-two-components}) is
    empty and concentrate on proving that
    $\sum_{x:\Sc}-\id_\Sc = \iota_2(x)$. The type
    $-\id_\Sc = \iota_2(x)$ is equivalent to
    $\sum_{p:\base = x}\inv{f_x(\Sloop)}p=p\inv\Sloop$ which is again
    equivalent to $\base=x$. We conclude the proof in the same way by
    recognizing a singleton.
  \end{enumerate}
\end{proof}

\section{The sphere}

We define the sphere $\Sp$ as the suspension $\susp\Sc$ of the circle.

\subsection{Roadmap}
\label{sec:roadmap}

For the sphere $\Sp$, topological intuitions make very unlikely to
construct a proof of $(\Sp + \Sp) \weq (\Sp = \Sp)$. We shall target a
lesser goal: prove that $(\Sp = \Sp)$ has two equivalent connected
components, and investigate some of the elements of both these
components. Through univalence, it is enough to do so for the type
$(\Sp \weq \Sp)$. The first step is to provide a dependent function
\begin{equation}
  \label{eq:14}
  \prod_{f:\Sp \weq \Sp} \Truncprop{e_0=f} + \Truncprop{e_1=f}
\end{equation}
where $e_0$ and $e_1$ are two equivalences to be defined afterwards
that embody the role of $\id_\Sc$ and $-\id_\Sc$ in the case of
$\Sc$. One of the key property is that $e_0 \neq e_1$, so that for
each $f: \Sp \weq \Sp$, the type
$P(f) \defequi \Truncprop{e_0=f} + \Truncprop{e_1=f}$ is a
proposition. Hence, for a given $f$, in proving $P(f)$ one might as
well suppose a path $f_0: N = f(N)$ instead of the mere fact that
$\Truncprop{N=f(N)}$. In other words, one can suppose that $f$ is a
pointed equivalence. We shall show how to produce a map\footnote{This
  map is {\tt \color{red} (should be)} related to the usual degree of
  pointed endomaps of $\Sp$ as follows: for $f:\Sp \ptdto \Sp$, under
  the equivalence $\pi_2(\Sp) \weq \zet \weq (\base = \base)$, the
  induced group morphism $\pi_2(f)$ identifies with $\ap{\phi(f)}$.}
\begin{equation}
  \label{eq:18}
  \phi: (\Sp \ptdto \Sp) \to (\Sc \to \Sc)
\end{equation}
such that $\phi(\id_{\Sp},\refl N) = \id_\Sc$ and such that
$\phi(g\circ f) = \phi(g) \circ \phi(f)$. (In other words, $\phi$ is a
morphism of monoids.) In particular $\phi$ sends pointed equivalences
to equivalences $\Sc \weq \Sc$, which we already know to be merely
equal either to $\id_\Sc$ or $-\id_\Sc$. From the definition of $e_0$
and $e_1$, it will be obvious that $\phi(e_0) = \id_\Sc$ and
$\phi(e_1) = -\id_\Sc$. The proposition (\ref{eq:14}) will be
eventually proved once it is shown that
$\Truncprop{\phi(f) = \phi(g)} \implies \Truncprop{f=g}$.

The second step is to prove that the component
$\sum_{f:\Sp \weq \Sp}\Truncprop{e_0=f}$ is equivalent to
$\sum_{f:\Sp \weq \Sp}\Truncprop{e_1=f}$. For that purpose, we shall
construct a map $\flip : (\Sp \to \Sp) \to (\Sp \to \Sp)$ where $\flip(f)$
is defined by induction as follows:
\begin{equation}
  \label{eq:19}
  \begin{aligned}
    \flip(f)(N) &\defequi f(N)
    \\
    \flip(f)(S) &\defequi f(S)
    \\
    \flip(f)(\mrd(\base)) &\defis f(\mrd(\base))
    \\
    \flip(f)(\mrd(\Sloop)) &\defis \inv{f(\mrd(\Sloop))}
  \end{aligned}
\end{equation}
This is an equivalence, as it is its own pseudoinverse. In addition,
it is easily seen that $\flip(e_0) = \flip(e_1)$. Hence, $\flip$ is an
equivalence that sends the connected component at $e_0$ into the
connected component at $e_1$. The equivalence from
$\conncomp{(\Sp\weq\Sp)}{e_0}$ to $\conncomp{(\Sp\weq\Sp)}{e_1}$ is
obtained by restraining $\flip$ to $\conncomp{(\Sp\weq\Sp)}{e_0}$, and
co-restraining it to $\conncomp{(\Sp\weq\Sp)}{e_1}$.

\subsection{Degree}
\label{sec:winding-numbers}
\def\hopffam{\mathcal H}%

In associating a degree in $\ZZ$ to every pointed map $\Sp \to \Sp$,
the first step is usually to define a map
$\loopspace 2 \Sp \to \loopspace 1 \Sc$. The construction of such a
map in HoTT is known material that can be tracked back at least to
Brunerie's thesis. It goes through the construction of a
type-theoretical version of the Hopf fibration that we shall follow
carefully.

Consider the type family $\hopffam : \Sp \to \U$ defined by induction
as follows:
\begin{equation}
  \label{eq:7}
  \begin{aligned}
    \hopffam (N) &\defequi \Sc
    \\
    \hopffam (S) &\defequi \Sc
    \\
    \hopffam (\mrd(x)) &\defis \iota_1(x) \quad\text{defined above}
  \end{aligned}
\end{equation}
Denote $X$ for the total space of the type family $\hopffam$. One can
show that $X$ is equivalent to the 3-sphere $\susp \Sp$, but this
shall not prove useful here. Hence we keep the neutral notation $X$ to
avoid confusion. By analogy with the topological situation, the
projection $\fst : X\to \Sp$ is referred as the Hopf fibration in the
literature.

If one equip $\Sc$ with the point $\base$, $\Sp$ with the point $N$
and $X$ with the point $(N,\base)$, then $\hopffam$ becomes a pointed
type family and the Hopf fibration a pointed map whose fiber at the
selected point $N$ is equivalent to $\Sc$.
\begin{definition}
  A {\em fiber sequence} is a sequence of pointed maps
  $F \overset i \to B \overset f \to A$ together with an equivalence
  $\varphi : F \weq \inv f(\pt_A)$ such that $i = \fst \varphi$.
\end{definition}
Given that definition, the Hopf fibration gives a fiber sequence as
follows:
\begin{equation}
  \label{eq:9}
  \begin{tikzcd}
    \Sc \rar["i"] & X \rar["\fst"] & \Sp
  \end{tikzcd}
\end{equation}
The first map $i:\Sc \to X$ is simply the inclusion $z\mapsto
(N,z)$.

The usual argument is to apply the following lemma:
\begin{lemma}
  \label{lemma:fiber-seq-omega}%
  If $F\overset i \to B \overset f \to A$ is a fiber sequence, then
  for every $n:\NN$, the following is a fiber sequence:
  \begin{equation}
    \label{eq:20}
    \begin{tikzcd}
      \loopspace n F \rar["\loopspace n i"] & \loopspace n B
      \rar["\loopspace n \fst"] & \loopspace n A
    \end{tikzcd}    
  \end{equation}
  And the fiber of $\loopspace n i$ identifies with
  $\loopspace {n+1} A$.
\end{lemma}
Applying the lemma for the Hopf fiber sequence and $n=1$ will provide
the wanted map $\loopspace 2 \Sp \to \loopspace 1 \Sc$.

If we study the proof-term of lemma~\ref{lemma:fiber-seq-omega}
closely in our instance of the Hopf fibration, the induced fiber
sequence at the level of path spaces:
\begin{equation}
  \label{eq:12}
  \begin{tikzcd}
    \loopspace 1 \Sc \rar["\loopspace 1 i"] & \loopspace 1 X \rar["\loopspace 1 \fst"] & \loopspace 1 \Sp
  \end{tikzcd}
\end{equation}
is obtained through the following equivalences:
\begin{align}
  \inv {(\loopspace 1 \fst)} {(\refl N)}
  &\jdeq \sum_{(q,p):\loopspace 1 X}\refl N = (\loopspace 1 \fst)(q,p)
  \\
  &\weq \sum_{q:N=N}\sum_{p:\pathover \base {\hopffam} q \base}\refl N = q
  \\
  &\weq \sum_{q:N=N}(\refl N = q)\times(\trp[\hopffam]{q} (\base) =\base)
  \\
  &\weq (\base = \base) \jdeq \loopspace 1 \Sc
\end{align}
Following the equivalence from $\loopspace 1 \Sc$ back to the fiber is
simply the function $p \mapsto ((\refl N,p),\refl{\refl N})$. Hence,
by postcomposing with the first projection, one gets
$p \mapsto (\refl N, p)$ which is indeed
$\loopspace 1 i: \loopspace 1 \Sc \to \loopspace 1 X$. The type we are
most interested in the fiber of $\loopspace 1 i$, for which the
proof-term of lemma~\ref{lemma:fiber-seq-omega} is computed as follow:
\begin{align}
  \inv {(\loopspace 1 i)} {(\refl N,\refl\base)}
  &\weq \sum_{p:\base = \base}(\refl N,\refl\base) = (\refl N,p)
  \\
  &\weq \sum_{\alpha:\refl N = \refl N}\sum_{p:\base=\base}\trp [T]{\alpha}(\refl \base)
 = p \label{eq:type-fam-trp}
  \\
  &\weq (\refl N = \refl N) \jdeq \loopspace 2 \Sp
\end{align}
where the type family $T:\loopspace 1 \Sp \to \U$ in
(\ref{eq:type-fam-trp}) is given by
$e\mapsto {(\pathover \base \hopffam e \base)}$. The equivalence from
$\loopspace 2 \Sp$ back to the actual fiber is given by:
\begin{equation}
  \begin{aligned}
    \loopspace 2 \Sp &\to \sum_{p:\base = \base}(\refl N,\refl\base) =
    (\refl N,p)
    \\
    \alpha &\mapsto \left(\trp[T]{\alpha}(\refl\base), \left(\alpha,
        \refl{\trp[T]{\alpha}(\refl\base)}\right)\right)
  \end{aligned}
\end{equation}
The map of interest is the composition of this equivalence with the
first projection to $\loopspace 1 \Sc$, which is then simply
$\alpha \mapsto \trp[T]{\alpha}(\refl \base)$. All that remains to do
is to determine the transport in the type family $T$, so that one can
make explicit further the map $\eta:\loopspace 2 \Sp \to \loopspace 1 \Sc$
described above.

For any paths $e,e':N=N$ and a path $\alpha: e=e'$ between them, the
transport
$\trp[T]{\alpha}:(\trp[\hopffam]{e}(\base) = \base) \to (\trp[\hopffam]{e'}(\base) =
\base)$ identifies with the following function:
\begin{equation}
  \begin{aligned}
    {t_\alpha}: (\trp[\hopffam]{e}(\base) = \base) &\to (\trp[\hopffam]{e'}(\base) = \base)
    \\
    p &\mapsto p\cdot \inv{\left(\ap{\trp[\hopffam]{\blank}(\base)}(\alpha)\right)}
  \end{aligned}
\end{equation}
This is shown by induction on $\alpha$: indeed,
$\ap{\trp[\hopffam]{\blank}(\base)}(\refl{e}) \jdeq \refl{\trp[\hopffam]{e}(\base)}$; hence
path algebra provides a proof of $t_{\refl{e}}(p) = p$. So in
particular when $\alpha:\refl N = \refl N$, one gets
\begin{equation}\label{eq:trp-Ta(refl)}
  \trp[T]{\alpha}(\refl \base) = \refl \base \cdot \inv{\left(\ap{\trp[\hopffam]{\blank}(\base)}(\alpha)\right)}
  = \inv{\left(\ap{\trp[\hopffam]{\blank}(\base)}(\alpha)\right)}
\end{equation}
In the end, the map $\eta\from \loopspace 2 \Sp \to \loopspace 1 \Sc$
is $\inv{(\blank)} \circ \ap \theta$ where $\theta : (N=N) \to \Sc$ is the function defined by
\begin{equation}
  \label{eq:17}
  \theta(p) \defequi \trp[\hopffam]{p}(\base)
\end{equation}
For practical reason, we prefer to work directly with
$\ap\theta\from \loopspace 2 \Sp \to \loopspace 1 \Sc$. This does not
matter much as $\inv{(\blank)}$ is an equivalence
$\loopspace 1 \Sc \weq \loopspace 1 \Sc$, hence every property that
$\ap\theta$ will enjoy, $\eta$ will also.

Recall now the adjoint pair $\susp{}\dashv\loopspace{}$, which
provides an equivalence
\begin{equation}
  \label{eq:24}
  \begin{tikzcd}[%
    /tikz/column 1/.append style={anchor=base east},%
    /tikz/column 2/.append style={anchor=base west}]
    \blank^\natural : (\Sp \ptdto \Sp) \rar["\weq"] & (\Sc \ptdto
    \loopspace 1 \Sp)
    \\
    (f,f_0) \rar[mapsto] & (\varphi_{f,f_0}, \pi_{f,f_0})
  \end{tikzcd}
\end{equation}
where in the right hand side
$\varphi_{f,f_0} \defequi\inv{f_0}\cdot \inv{f(\mrd(\base))}\cdot
f(\mrd(\blank))\cdot f_0$ and the path $\pi_{f,f_0}$ is defined as the
composition of the following elementary path algebra steps:
\begin{equation}
  \label{eq:16}
  \refl N = \inv{f_0}f_0 = \inv{f_0}\refl {f(N)} f_0 =  \inv{f_0}\inv{f(\mrd(\base))}f(\mrd(\base)) f_0
\end{equation}
Remark that for two pointed functions $(f,f_0)$ and $(g,g_0)$, path
algebra and functoriality of $g$ yields a path, call it
$c_{f,f_0,g,g_0}$, in:
\begin{equation}
  \begin{aligned}
    \varphi_{gf,g(f_0)g_0} %
    &= \inv{(g(f_0)g_0)}\inv{gf(\mrd(\base))} gf(\mrd(\blank))
    g(f_0)g_0
    \\
    &= \inv{g_0}\inv{g(f_0)}\inv{gf(\mrd(\base))} gf(\mrd(\blank))
    g(f_0)g_0
    \\
    &= \inv{g_0} \cdot \ap g (\varphi_{f,f_0}(\blank)) \cdot g_0
    \\
    &\jdeq \loopspace 1 (g,g_0) \circ \varphi_{f,f_0}
  \end{aligned}\label{eq:26}
\end{equation}
Now, the transport of $\pi_{gf,g(f_0)g_0}$ along $c_{f,f_0,g,g_0}$ in
the type family $h\mapsto (\refl N = h(\base))$ is simply given by the
composition $c_{f,f_0,g,g_0}\pi_{gf,g(f_0)g_0}$. Compare it to the
path $\ap{\loopspace 1{(g,g_0)}}(\pi_{f,f_0})\varpi$ where $\varpi$ is
the path $\refl N = \inv{g_0}g_0 = \inv{g_0}\refl{g(N)}g_0$ coming
from path algebra. Both are equal from compatibility of path algebra
with functoriality of function. In other words:
\begin{equation}
  \label{eq:27}
  {\left((g,g_0)\circ (f,f_0)\right)}^\natural = \loopspace 1 (g,g_0) \circ (f,f_0)^\natural 
\end{equation}
And in particular for any pointed map $(f,f_0)$ one has an element of
\begin{displaymath}
  (f,f_0)^\natural = \loopspace 1 (f,f_0) \circ
  (\id_{\Sp},\refl N)^\natural
\end{displaymath}

Now recall that the universal property of the circle establishes fro
any type $A$ an equivalence:
\begin{equation}
  \label{eq:25}
  \begin{tikzcd}[%
    /tikz/column 1/.append style={anchor=base east},%
    /tikz/column 2/.append style={anchor=base west}]
    \lambda_{A}: (\Sc \ptdto A) \rar["\weq"] & \loopspace 1 A
    \\
    (h,h_0) \rar[mapsto] & \inv{h_0} \cdot h(\Sloop)\cdot h_0
  \end{tikzcd}
\end{equation}
An inverse equivalence for $\lambda_A$ is simply mapping a loop
$\ell : \pt_A = \pt_A$ to the map $\Sc \to A$ defined by induction as
$\base \mapsto \pt_A$ and $\Sloop \mapsto \ell$ pointed by
$\refl{\pt_A}$. Remark also that for any pointed maps
$(f,f_0):\Sc \ptdto A$ and $(g,g_0):A \ptdto A$,
\begin{equation}
  \label{eq:32}
  \lambda_A((g,g_0)\circ (f,f_0)) = \inv{g_0} \inv{g(f_0)} g(f(\Sloop)) g(f_0) g_0
  = \loopspace 1 (g,g_0) ( \lambda_A(f,f_0) )
\end{equation}
In particular, when $A$ is $\loopspace 1 \Sp$, one can compose
$\lambda_{\loopspace 1 \Sp}$ with $\blank^\natural$ to find an
equivalence from $(\Sp \ptdto \Sp)$ to $\loopspace 2 \Sp$, which,
through equations~(\ref{eq:27}) and~(\ref{eq:32}) maps a composition
of the form $(g,g_0)\circ (f,f_0)$ to
$\loopspace 2 (g,g_0)(\lambda_{\loopspace 1 \Sp}(f,f_0)^\natural)$. In
particular, it follows that
\begin{equation}
  \label{eq:34}
  \lambda_{\loopspace 1 \Sp}\left((f,f_0)^\natural\right) = \loopspace 2 {(f,f_0)} (\ap{\mrd(\base)\cdot \mrd(\blank)}(\Sloop))
\end{equation}

The HoTT-book proves that the map
$\ap \theta:\loopspace 2 \Sp \to \loopspace 1 \Sc$ is (equivalent to)
the set-truncation map of $\loopspace 2 \Sp$, or equivalently that
$\Truncset {\ap \theta}$ is an equivalence. Denote $\Phi$ for the
equivalence $\loopspace 1 \Sc \weq \ZZ$, and define the degree of a
pointed function $(f,f_0)$ to be
\begin{equation}
  \label{eq:28}
  d(f,f_0) \defequi \Phi\left(\ap\theta\left(
      \lambda_{\loopspace 1 \Sp}\left(((f,f_0)^\natural\right))
    \right)\right)
\end{equation}
Hence $\ZZ$ is equivalent to the set-truncation of $(\Sp \ptdto \Sp)$
and the function $d: (\Sp \ptdto \Sp) \to \ZZ$ is the truncation of
element under this equivalence. In other words, it means that two
pointed functions are in the same connected component of
$\Sp \ptdto \Sp$ if and only if they have the same degree.
% To conclude that equivalences are merely equal to either $\id_\Sp$ or
% $-\id_\Sp$, one still need to prove that equivalences have degree
% either $1$ or $-1$. This will be achieved if we manage to prove that
% $d$ maps compositions to products in $\ZZ$. For that we shall 

Now remark that that there is a kind of naturality for the equivalence
defined in (\ref{eq:25}): namely if $\ell:A \ptdto B$, then it defines
a map $\ell^\ast : (\Sc\ptdto A) \to (\Sc \ptdto B)$ by
postcomposition and
$\ap \ell \circ \lambda_A = \lambda_B \circ \ell^\ast$ is
inhabited. Define now $\bar d: (\Sp \ptdto \Sp) \to (\Sc \ptdto \Sc)$
as follows:
\begin{equation}
  \label{eq:29}
  \bar d(f,f_0) \defequi \theta\circ(f,f_0)^\natural
\end{equation}
The situation is better summarized in the following commutative
diagram:
\begin{equation}
  \label{eq:30}
  \begin{tikzcd}
    {} & (\Sc \ptdto \Sc) \rar["\lambda_{\Sc}","\weq"swap] &
    \loopspace 1 \Sc \rar["\Phi","\weq"swap] & \ZZ
    \\
    (\Sp \ptdto \Sp) \rar["\blank^\natural","\weq"swap] \ar[ur,bend
    left,"\bar d"] \ar[urrr, bend
    right=90,"d"swap] & (\Sc \ptdto \loopspace 1 \Sp)
    \rar["\lambda_{\loopspace 1 \Sp}","\weq"swap] \uar["\theta^\ast"]
    & \loopspace 2 \Sp \uar["\ap\theta"swap] & 
  \end{tikzcd}
\end{equation}

Because $d$ is a set-truncation map of $\Sp\ptdto\Sp$, then so is
$\bar d$. Recall from section~\ref{sec:circle-case} that pointed
equivalences $\Sc \ptdto \Sc$ are equal to either $\id_\Sc$ or
$-\id_\Sc$.\footnote{\color{red}This only appears in the proof of the
  main result and not as an independent lemma. It should be
  corrected.}  And because $\bar d(\id_\Sp) = \id_\Sc$ and
$\bar d(-\id_\Sp) = -\id_\Sc$ (to be shown in full details later), it
is enough to prove that $\bar d$ maps equivalences to equivalences to
conclude. Let us shall that by proving $\bar d$ preserves composition
of pointed maps. Any pointed map $(f,f_0): \Sp \ptdto \Sp$ induces a
map
\begin{equation}
  \label{eq:33}
  \pi_2(f,f_0)\defequi \Truncset {\loopspace 2 (f,f_0)}: \Truncset{\loopspace 2 \Sp} \to \Truncset{\loopspace 2 \Sp}
\end{equation}
which, by definition, has the property that the following diagram
commutes:
\begin{equation}
  \label{eq:35}
  \begin{tikzcd}
    \loopspace 2 \Sp \rar["{\loopspace 2 (f,f_0)}"]
    \dar["\truncset\blank"swap] & \loopspace 2 \Sp
    \dar["\truncset\blank"]
    \\
    \Truncset{\loopspace 2 \Sp} \rar["{\pi_2 (f,f_0)}"swap] & \Truncset{\loopspace 2 \Sp}
  \end{tikzcd}
\end{equation}
In particular, one has that
$\ap \theta \circ \loopspace 2 (f,f_0) = \Truncset{\ap\theta} \circ
\pi_2(f,f_0) \circ \truncset\blank$. Now by definition
$\Truncset{\ap\theta} \circ \truncset\blank \jdeq \ap \theta$, and
$\Truncset{\ap\theta}$ being an equivalence, it follows that
$\truncset \blank = \inv{\Truncset{\ap\theta}}\circ \ap\theta$. Hence,
if we denote $\bar \pi_2(f,f_0): \loopspace 1 \Sc \to \loopspace 1 \Sc$ for the
composition
$\Truncset{\ap\theta} \circ \pi_2(f,f_0) \circ
\inv{\Truncset{\ap\theta}}$, then through equation~(\ref{eq:34}),
\begin{equation}
  \label{eq:36}%
  (\inv\Phi \circ d) (f,f_0) = \bar\pi_2(f,f_0)\left(
    \ap\theta \left( \ap {\mrd(\base) \cdot \mrd(\blank)} (\Sloop) \right)
  \right)
\end{equation}
Remember that $\theta\circ (\mrd(\base)\cdot\mrd(\blank)) =
\id_\Sc$. Then it follows that
\begin{equation}
  \label{eq:36}%
  (\inv\Phi \circ d) (f,f_0) = \bar\pi_2(f,f_0)\left( \Sloop \right)
\end{equation}
Through the commutativity of diagram~(\ref{eq:30}), it holds that
\begin{equation}
  \bar d(f,f_0) = \inv{\lambda_\Sc}(\bar \pi_2(f,f_0)(\Sloop))
\end{equation}
In particular, if $(f,f_0)$ and $(g,g_0)$ are pointed maps
$\Sp \ptdto \Sp$, then one has:
\begin{equation}
  \label{eq:37}
  \begin{aligned}
    \bar d ((g,g_0) \circ (f,f_0))
    &= \inv{\lambda_\Sc} \left(
      \bar\pi_2((g,g_0) \circ (f,f_0))\left( \Sloop \right)
    \right)
    \\
    &= \inv{\lambda_\Sc} \left(
      \bar\pi_2(g,g_0)\left( \bar\pi_2(f,f_0)\left(  \Sloop \right) \right)
    \right)
  \end{aligned}
\end{equation}
It is now easy to prove that the latter map is equal to
$\inv{\lambda_\Sc}(\bar \pi_2(g,g_0)(\Sloop)) \circ
\inv{\lambda_\Sc}(\bar \pi_2(f,f_0)(\Sloop))$ by circle induction: as
both maps $\base$ definitionally to $\base$, one only need to check
that their action on the loop is the same. By definition,
$\inv{\lambda_\Sc}(\bar \pi_2(g,g_0)(\bar\pi_2(f,f_0)(\Sloop)))$ acts
on the loop $\Sloop$ as $\bar
\pi_2(g,g_0)(\bar\pi_2(f,f_0)(\Sloop))$. While the action of
$\inv{\lambda_\Sc}(\bar \pi_2(g,g_0)(\Sloop)) \circ
\inv{\lambda_\Sc}(\bar \pi_2(f,f_0)(\Sloop))$ on the loop leads to the
application of
$\loopspace 1 \left(\inv{\lambda_\Sc}(\bar
  \pi_2(g,g_0)(\Sloop))\right)$ to $\bar\pi_2(f,f_0)(\Sloop)$. Hence,
it suffices to prove that
$\varpi_{g,g_0} \defequi\loopspace 1 \left(\inv{\lambda_\Sc}(\bar
  \pi_2(g,g_0)(\Sloop))\right)$ is equal to $\bar \pi_2(g,g_0)$. They
both are group morphisms that have the same image on $\Sloop$: for
each
$k:\ZZ$, one has
\begin{equation}
  \label{eq:39}
  \varpi_{g,g_0} (\Sloop^k) = \varpi_{g,g_0} (\Sloop)^k
  = \bar\pi_2(g,g_0)(\Sloop)^k = \bar\pi_2(g,g_0)(\Sloop^k)
\end{equation}
This concludes that $\bar d$ maps composition to composition. We have
also seen that $\bar d (\id_\Sp,\refl N) = (\id_\Sc,\refl\base)$,
hence $\bar d$ maps pointed equivalences to pointed equivalences.

To conclude: given any equivalence $f:\Sp\simeq \Sp$, in order to
prove the proposition
$P(f)\defequi \Truncprop{\id_\Sp = f} + \Truncprop{\-id_\Sp = f}$, one
can suppose that $f$ is pointed by a path $f_0$; then $\bar d(f,f_0)$
is a pointed equivalence $\S \ptdto \Sc$ which is already known to be
equal to either $(\id_\Sc,\refl\base)$ or
$(-id_\Sc,\refl\base)$. Moreover $\bar d$ has the property of a
set-truncation map, meaning that
$\Truncprop{x=y} \weq (\bar d x = \bar d y)$: so
$\bar d(f,f_0) = (\id_\Sc,\refl \base) \to \Truncprop{(f,f_0) =
  (\id_\Sp,\refl N)}$ and
$\bar d(f,f_0) = (-\id_\Sc,\refl \base) \to \Truncprop{(f,f_0) =
  (-\id_\Sp,\refl N)}$. It remains to project on the first component
and to use the universal property of a sum to find $P(f)$.

%% -----------------------
%% OLD STUFF, MAYBE USEFUL
%% -----------------------
%
% Now remark that under the equivalence $\Phi\lambda_\Sc$,
% multiplication in $\ZZ$ corresponds to composition in
% $\Sc \ptdto \Sc$, as proved in the first section. {\tt \color{red}
%   (Maybe we should make that a lemma somewhere.)} %
% Hence, $d$ maps composition to product if and only if $\bar d$
% preserves composition. Taking equation~(\ref{eq:27}) into account, one
% needs to prove that for pointed functions $(f,f_0)$ and $(g,g_0)$, the
% following type is inhabited:
=======
% \section{The circle}
% \label{sec:circle-case}%
% In this document, we shall establish an equivalence
% between the type of equalities of $\Sc$ with itself and the sum type
% of $\Sc$ with itself. 
% This type enjoys a inductive property: given a type family
% $T : \Sc \to \U$, an element $t:T(\base)$ and a path
% $p:t=_{T(\base)}t$, the $\Sc$-induction property provides a dependent
% function $f : \prod_{x:\Sc}T(x)$ such that $f(\base) \jdeq t$ and an
% element $q: f(\Sloop) = p$. In particular, $\Sc$-induction provides a
% map
>>>>>>> c6c88b18464b4f42b38c637acfd281cb6c10bade
% \begin{equation}
%   \label{eq:2}
%   \left(\sum_{t:T(\base)} \pathover t T \Sloop t\right) \to \prod_{x:\Sc}T(x)
% \end{equation}
% that is easily shown to be an equivalence (through $\Sc$-induction
% again). In the case $T$ is the family constant in $A:\U$, it gives the
% following universal property of the circle:
% \begin{equation}
%   \label{eq:3}
%   \left(\sum_{t:A}t=t\right) \weq \left( \Sc \to A \right)
% \end{equation}

% Using univalence, one gets that $(\Sc = \Sc)$ is
% equivalent to $(\Sc \weq \Sc)$ and we shall then concentrate on
% proving
% \begin{equation}
%   (\Sc \weq \Sc) \weq (\Sc+\Sc)
% \end{equation}
% As usual, we shall denote $f:\Sc\weq\Sc$ for a map $f:\Sc\to \Sc $
% such that the proposition $\isEq(f)$ is non-empty instead of the more
% honest $(f,!):\Sc \weq \Sc$. 
% This is harmless as the first projection is an equivalence
% \begin{equation}
%   ((f,!) =_{\Sc \weq \Sc} (g,!)) \simeq (f=_{\Sc \to \Sc}g).
% \end{equation}

% Recall that for each $x:\Sc$ one has an equivalence:
% \begin{equation}
%   \label{eq:8}
%   f_x : (\base = \base) \to (x=x)
% \end{equation}
% Indeed, such a map $x\mapsto f_x$ is defined by $\Sc$-induction by
% setting $f_{\base} \defequi \id$ and $f_{\Sloop}$ to be the proof that
% postcomposing with the conjugation $\Sloop \blank \inv\Sloop$ is the
% identity. In particular one can easily prove that
% $f_x(\Sloop^k) = f_x(\Sloop)^k$ for any $k:\zet$. Now define $\iota_1$
% and $\iota_2$ on each $x:\Sc$ as follows:
% \begin{align*}
%   \iota_1(x) (\base) &\defequi x & \iota_2(x) (\base) &\defequi x,
%   \\
%   \iota_1(x)(\Sloop) &\defis f_x(\Sloop) &  \iota_2(x)(\Sloop) &\defis f_x(\Sloop)^{-1}
% \end{align*}
% Each $\iota_j(x)$ ($j\in\{1,2\}$) is indeed an equivalence: $\Sc$
% being connected, the proposition $\isEq(\iota_j(x))$ has to be proved
% for only one point, say $\base$; but $\iota_1(\base)$ is equal to
% $\id_\Sc$ (hence an equivalence) and $\iota_2(\base)$, that we call
% $-\id_\Sc$ in the following, is its own pseudo inverse.

% The maps $\iota_1$ and $\iota_2$ induce a maps
% $\iota \defequi \langle\iota_1,\iota_2\rangle : \Sc + \Sc \to (\Sc\weq
% \Sc)$.
% \begin{theorem*}
%   The map $\iota$ is an equivalence.
% \end{theorem*}
% %
% Before proving the theorem, here is a lemma that already explains that
% the two components of $\Sc+\Sc$ are mapped to distinct parts of
% $(\Sc \weq \Sc)$.
% \begin{lemma}
%   \label{lemma:id-not-equal-oppid}%
%   The type $(\id_\Sc = -\id_\Sc)$ is empty.
% \end{lemma}
% \begin{proof}
%   By function extensionality and $\Sc$-induction, an element
%   $p : \id_\Sc = -\id_\Sc$ is given by the data of
%   $p(\base) : \id_\Sc(\base) = -\id_\Sc(base)$, that is
%   $p(\base):\base=\base$, together with a pathover
%   $p(\Sloop): \pathover {p(\base)} {} \Sloop {p(\base)}$ in the type
%   family $x\mapsto \id_\Sc (x)=-\id_\Sc (x)$. The transport over
%   $\Sloop$ in this type family is given by
%   $\inv\Sloop\blank\inv\Sloop$, so that $p(\Sloop)$ is equivalently
%   given by an element of $\inv\Sloop p(\base) \inv\Sloop =
%   p(\base)$. By expliciting $p(\base)$ as $\Sloop^k$ for some
%   $k:\zet$, the last equation becomes
%   $\inv \Sloop \Sloop^k \inv \Sloop = \Sloop^k$, leading to the
%   contradiction $2=0$ in $\zet$.
% \end{proof}

% \begin{proof}[Proof of the theorem]
%   Take an element $\phi : \Sc \weq \Sc$ and consider the fiber of
%   $\iota$ at $\phi$:
%   \begin{equation}
%     \label{eq:fiber-iota-two-components}%
%     \inv\iota(\phi) \jdeq \sum_{x:\Sc+\Sc} \phi = \iota(x)
%     \weq \left(\sum_{x:\Sc}\phi = \iota_1(x)\right) +
%     \left(\sum_{x:\Sc}\phi = \iota_2(x)\right)
%   \end{equation}
%   One wants to prove the proposition $\iscontr(\inv\iota(\phi))$,
%   hence one may assume a path $q:\phi(\base) = \base$ instead of
%   merely $\Trunc{\phi(\base) = \base}$. By the description of the
%   symmetries of $\base:\Sc$, there is $k:\zet$ such that
%   $q\phi(\Sloop)\inv q = \Sloop^k$. Denoting $\psi$ for a pseudo
%   inverse of the equivalence $\phi$, one also gets
%   $q':\psi(\base) = \base$ and subsequently $\ell:\zet$ such that
%   $q'\psi(\Sloop)\inv{q'} = \Sloop^\ell$. Doing some path algebra it
%   follows that:
%   \begin{equation}
%     \label{eq:4}
%     \psi(\phi(\Sloop)) = \inv{(q' \psi(q))}\Sloop^{\ell k} (q' \psi(q))
%   \end{equation}
%   By hypothesis, one has also $\tau : \prod_{x:\Sc}\psi\phi(x) = x$,
%   so there is a pathover of type
%   $\pathover {\tau(\base)} {} \Sloop {\tau(\base)}$ in the type family
%   $x \mapsto \psi(\phi(x)) = x$. The transport over $\Sloop$ in this
%   family is given by $\Sloop \blank \inv{(\psi\phi(\Sloop))}$, so that
%   \begin{equation}
%     \label{eq:5}
%     \Sloop \tau(\base) \inv{(\psi\phi(\Sloop))} = \tau(\base)
%   \end{equation}
%   Putting equations~(\ref{eq:5}) and~(\ref{eq:4}) together, one find
%   \begin{equation}
%     \label{eq:6}
%     \inv{(q' \psi(q) \inv{\tau(\base)})} \Sloop^{\ell k} (q' \psi(q) \inv{\tau(\base)}) = \Sloop
%   \end{equation}
%   However, conjugation in $\base = \base$ is equal to the identity, so
%   that in the end $\Sloop^{\ell k} = \Sloop$, or otherwise put
%   $\ell k = 1$ in $\zet$. Using the decidability of equality on
%   $\zet$, one can show that the only invertible elements of $\zet$ are
%   $1$ and $-1$. So $k = 1$ or $k = -1$. We proceed by case:
%   \begin{enumerate}
%   \item If $k=1$, then $Q \defequi (q, !)$ is element of
%     $\phi = \id_\Sc$, where the term $!$ of the proposition
%     $q=\Sloop q \inv{\phi(\Sloop)}$ is given by path algebra from the
%     hypothesis $q\phi(\Sloop)\inv q=\Sloop^k$ ($k$ being $1$ here). We
%     can then only show $\iscontr(\inv\iota(\id_\Sc))$ to conclude (by
%     transport over $\inv Q$) that $\iscontr(\inv\iota(\phi))$. 

%     One can immediately see that the second component of the right
%     hand-side in equation~(\ref{eq:fiber-iota-two-components}) is
%     empty when instantiated with $\id_\Sc$ for $\phi$: indeed,
%     $\false$ being a proposition and $\Sc$ being connected, one only
%     need to check that $\id_\Sc = \iota_2(\base)$ is contradictory;
%     and because $\iota_2(\base) \jdeq -\id_\Sc$, it follows from
%     lemma~\ref{lemma:id-not-equal-oppid}.

%     Hence it remains to prove that $\sum_{x:\Sc}\id_\Sc=\iota_1(x)$ is
%     contractible. For $x:\Sc$, the type $\id_\Sc = \iota_1(x)$ is
%     equivalent (through first function extensionality and then
%     $\Sc$-induction) to
%     $\sum_{p:\base=\iota_1(x)(\base)}\pathover p T \Sloop p$ where $T$
%     is the type family $y\mapsto y=\iota_1(x)(y)$. Now
%     $\iota_1(x)(\base)\jdeq x$ and the transport over the loop
%     $\Sloop$ in the type family $T$ is given by
%     $\iota_1(x)(\Sloop) \blank \inv \Sloop$. In other words,
%     \begin{equation}
%       \label{eq:10}
%       (\id_\Sc = \iota_1(x)) \weq \sum_{p:\base = x}f_x(\Sloop)p = p {\Sloop}
%     \end{equation}
%     By induction on $p:\base=x$, one can see that the proposition
%     $f_x(\Sloop)p=p\Sloop$ is always satisfied. In the end, one has
%     \begin{equation}
%       \label{eq:11}
%       \left(\sum_{x:\Sc}\id_\Sc=\iota_1(x)\right)
%       \weq
%       \left(\sum_{x:\Sc}\base=x\right)
%     \end{equation}
%     The type on the right is a singleton, hence contractible. This
%     concludes the case $k=1$.
%       \item The case $k=-1$ is completely similar to the case $k=1$. We
%     now have a proof that $\phi = -\id_\Sc$, hence we shall prove that
%     $\iscontr(\inv\iota(-\id_\Sc))$. From there one can determine that
%     the first component of in (\ref{eq:fiber-iota-two-components}) is
%     empty and concentrate on proving that
%     $\sum_{x:\Sc}-\id_\Sc = \iota_2(x)$. The type
%     $-\id_\Sc = \iota_2(x)$ is equivalent to
%     $\sum_{p:\base = x}\inv{f_x(\Sloop)}p=p\inv\Sloop$ which is again
%     equivalent to $\base=x$. We conclude the proof in the same way by
%     recognizing a singleton.
%   \end{enumerate}
% \end{proof}

% \section{The sphere}

% We define the sphere $\Sp$ as the suspension $\susp\Sc$ of the circle.

% \subsection{Roadmap}
% \label{sec:roadmap}

% For the sphere $\Sp$, topological intuitions make very unlikely to
% construct a proof of $(\Sp + \Sp) \weq (\Sp = \Sp)$. We shall target a
% lesser goal: prove that $(\Sp = \Sp)$ has two equivalent connected
% components, and investigate some of the elements of both these
% components. Through univalence, it is enough to do so for the type
% $(\Sp \weq \Sp)$. The first step is to provide a dependent function
% \begin{equation}
%   \label{eq:14}
%   \prod_{f:\Sp \weq \Sp} \Truncprop{e_0=f} + \Truncprop{e_1=f}
% \end{equation}
% where $e_0$ and $e_1$ are two equivalences to be defined afterwards
% that embody the role of $\id_\Sc$ and $-\id_\Sc$ in the case of
% $\Sc$. One of the key property is that $e_0 \neq e_1$, so that for
% each $f: \Sp \weq \Sp$, the type
% $P(f) \defequi \Truncprop{e_0=f} + \Truncprop{e_1=f}$ is a
% proposition. Hence, for a given $f$, in proving $P(f)$ one might as
% well suppose a path $f_0: N = f(N)$ instead of the mere fact that
% $\Truncprop{N=f(N)}$. In other words, one can suppose that $f$ is a
% pointed equivalence. We shall show how to produce a map\footnote{This
%   map is {\tt \color{red} (should be)} related to the usual degree of
%   pointed endomaps of $\Sp$ as follows: for $f:\Sp \ptdto \Sp$, under
%   the equivalence $\pi_2(\Sp) \weq \zet \weq (\base = \base)$, the
%   induced group morphism $\pi_2(f)$ identifies with $\ap{\phi(f)}$.}
% \begin{equation}
%   \label{eq:18}
%   \phi: (\Sp \ptdto \Sp) \to (\Sc \to \Sc)
% \end{equation}
% such that $\phi(\id_{\Sp},\refl N) = \id_\Sc$ and such that
% $\phi(g\circ f) = \phi(g) \circ \phi(f)$. (In other words, $\phi$ is a
% morphism of monoids.) In particular $\phi$ sends pointed equivalences
% to equivalences $\Sc \weq \Sc$, which we already know to be merely
% equal either to $\id_\Sc$ or $-\id_\Sc$. From the definition of $e_0$
% and $e_1$, it will be obvious that $\phi(e_0) = \id_\Sc$ and
% $\phi(e_1) = -\id_\Sc$. The proposition (\ref{eq:14}) will be
% eventually proved once it is shown that
% $\Truncprop{\phi(f) = \phi(g)} \implies \Truncprop{f=g}$.

% The second step is to prove that the component
% $\sum_{f:\Sp \weq \Sp}\Truncprop{e_0=f}$ is equivalent to
% $\sum_{f:\Sp \weq \Sp}\Truncprop{e_1=f}$. For that purpose, we shall
% construct a map $\flip : (\Sp \to \Sp) \to (\Sp \to \Sp)$ where $\flip(f)$
% is defined by induction as follows:
% \begin{equation}
%   \label{eq:19}
%   \begin{aligned}
%     \flip(f)(N) &\defequi f(N)
%     \\
%     \flip(f)(S) &\defequi f(S)
%     \\
%     \flip(f)(\mrd(\base)) &\defis f(\mrd(\base))
%     \\
%     \flip(f)(\mrd(\Sloop)) &\defis \inv{f(\mrd(\Sloop))}
%   \end{aligned}
% \end{equation}
% This is an equivalence, as it is its own pseudoinverse. In addition,
% it is easily seen that $\flip(e_0) = \flip(e_1)$. Hence, $\flip$ is an
% equivalence that sends the connected component at $e_0$ into the
% connected component at $e_1$. The equivalence from
% $\conncomp{(\Sp\weq\Sp)}{e_0}$ to $\conncomp{(\Sp\weq\Sp)}{e_1}$ is
% obtained by restraining $\flip$ to $\conncomp{(\Sp\weq\Sp)}{e_0}$, and
% co-restraining it to $\conncomp{(\Sp\weq\Sp)}{e_1}$.

% \subsection{Degree}
% \label{sec:winding-numbers}
% \def\hopffam{\mathcal H}%

% In associating a degree in $\ZZ$ to every pointed map $\Sp \to \Sp$,
% the first step is usually to define a map
% $\loopspace 2 \Sp \to \loopspace 1 \Sc$. The construction of such a
% map in HoTT is known material that can be tracked back at least to
% Brunerie's thesis. It goes through the construction of a
% type-theoretical version of the Hopf fibration that we shall follow
% carefully.

% Consider the type family $\hopffam : \Sp \to \U$ defined by induction
% as follows:
% \begin{equation}
%   \label{eq:7}
%   \begin{aligned}
%     \hopffam (N) &\defequi \Sc
%     \\
%     \hopffam (S) &\defequi \Sc
%     \\
%     \hopffam (\mrd(x)) &\defis \iota_1(x) \quad\text{defined above}
%   \end{aligned}
% \end{equation}
% Denote $X$ for the total space of the type family $\hopffam$. One can
% show that $X$ is equivalent to the 3-sphere $\susp \Sp$, but this
% shall not prove useful here. Hence we keep the neutral notation $X$ to
% avoid confusion. By analogy with the topological situation, the
% projection $\fst : X\to \Sp$ is referred as the Hopf fibration in the
% literature.

% If one equip $\Sc$ with the point $\base$, $\Sp$ with the point $N$
% and $X$ with the point $(N,\base)$, then $\hopffam$ becomes a pointed
% type family and the Hopf fibration a pointed map whose fiber at the
% selected point $N$ is equivalent to $\Sc$.
% \begin{definition}
%   A {\em fiber sequence} is a sequence of pointed maps
%   $F \overset i \to B \overset f \to A$ together with an equivalence
%   $\varphi : F \weq \inv f(\pt_A)$ such that $i = \fst \varphi$.
% \end{definition}
% Given that definition, the Hopf fibration gives a fiber sequence as
% follows:
% \begin{equation}
%   \label{eq:9}
%   \begin{tikzcd}
%     \Sc \rar["i"] & X \rar["\fst"] & \Sp
%   \end{tikzcd}
% \end{equation}
% The first map $i:\Sc \to X$ is simply the inclusion $z\mapsto
% (N,z)$.

% The usual argument is to apply the following lemma:
% \begin{lemma}
%   \label{lemma:fiber-seq-omega}%
%   If $F\overset i \to B \overset f \to A$ is a fiber sequence, then
%   for every $n:\NN$, the following is a fiber sequence:
%   \begin{equation}
%     \label{eq:20}
%     \begin{tikzcd}
%       \loopspace n F \rar["\loopspace n i"] & \loopspace n B
%       \rar["\loopspace n \fst"] & \loopspace n A
%     \end{tikzcd}    
%   \end{equation}
%   And the fiber of $\loopspace n i$ identifies with
%   $\loopspace {n+1} A$.
% \end{lemma}
% Applying the lemma for the Hopf fiber sequence and $n=1$ will provide
% the wanted map $\loopspace 2 \Sp \to \loopspace 1 \Sc$.

% If we study the proof-term of lemma~\ref{lemma:fiber-seq-omega}
% closely in our instance of the Hopf fibration, the induced fiber
% sequence at the level of path spaces:
% \begin{equation}
%   \label{eq:12}
%   \begin{tikzcd}
%     \loopspace 1 \Sc \rar["\loopspace 1 i"] & \loopspace 1 X \rar["\loopspace 1 \fst"] & \loopspace 1 \Sp
%   \end{tikzcd}
% \end{equation}
% is obtained through the following equivalences:
% \begin{align}
%   \inv {(\loopspace 1 \fst)} {(\refl N)}
%   &\jdeq \sum_{(q,p):\loopspace 1 X}\refl N = (\loopspace 1 \fst)(q,p)
%   \\
%   &\weq \sum_{q:N=N}\sum_{p:\pathover \base {\hopffam} q \base}\refl N = q
%   \\
%   &\weq \sum_{q:N=N}(\refl N = q)\times(\trp[\hopffam]{q} (\base) =\base)
%   \\
%   &\weq (\base = \base) \jdeq \loopspace 1 \Sc
% \end{align}
% Following the equivalence from $\loopspace 1 \Sc$ back to the fiber is
% simply the function $p \mapsto ((\refl N,p),\refl{\refl N})$. Hence,
% by postcomposing with the first projection, one gets
% $p \mapsto (\refl N, p)$ which is indeed
% $\loopspace 1 i: \loopspace 1 \Sc \to \loopspace 1 X$. The type we are
% most interested in the fiber of $\loopspace 1 i$, for which the
% proof-term of lemma~\ref{lemma:fiber-seq-omega} is computed as follow:
% \begin{align}
%   \inv {(\loopspace 1 i)} {(\refl N,\refl\base)}
%   &\weq \sum_{p:\base = \base}(\refl N,\refl\base) = (\refl N,p)
%   \\
%   &\weq \sum_{\alpha:\refl N = \refl N}\sum_{p:\base=\base}\trp [T]{\alpha}(\refl \base)
%  = p \label{eq:type-fam-trp}
%   \\
%   &\weq (\refl N = \refl N) \jdeq \loopspace 2 \Sp
% \end{align}
% where the type family $T:\loopspace 1 \Sp \to \U$ in
% (\ref{eq:type-fam-trp}) is given by
% $e\mapsto {(\pathover \base \hopffam e \base)}$. The equivalence from
% $\loopspace 2 \Sp$ back to the actual fiber is given by:
% \begin{equation}
%   \begin{aligned}
%     \loopspace 2 \Sp &\to \sum_{p:\base = \base}(\refl N,\refl\base) =
%     (\refl N,p)
%     \\
%     \alpha &\mapsto \left(\trp[T]{\alpha}(\refl\base), \left(\alpha,
%         \refl{\trp[T]{\alpha}(\refl\base)}\right)\right)
%   \end{aligned}
% \end{equation}
% The map of interest is the composition of this equivalence with the
% first projection to $\loopspace 1 \Sc$, which is then simply
% $\alpha \mapsto \trp[T]{\alpha}(\refl \base)$. All that remains to do
% is to determine the transport in the type family $T$, so that one can
% make explicit further the map $\eta:\loopspace 2 \Sp \to \loopspace 1 \Sc$
% described above.

% For any paths $e,e':N=N$ and a path $\alpha: e=e'$ between them, the
% transport
% $\trp[T]{\alpha}:(\trp[\hopffam]{e}(\base) = \base) \to (\trp[\hopffam]{e'}(\base) =
% \base)$ identifies with the following function:
% \begin{equation}
%   \begin{aligned}
%     {t_\alpha}: (\trp[\hopffam]{e}(\base) = \base) &\to (\trp[\hopffam]{e'}(\base) = \base)
%     \\
%     p &\mapsto p\cdot \inv{\left(\ap{\trp[\hopffam]{\blank}(\base)}(\alpha)\right)}
%   \end{aligned}
% \end{equation}
% This is shown by induction on $\alpha$: indeed,
% $\ap{\trp[\hopffam]{\blank}(\base)}(\refl{e}) \jdeq \refl{\trp[\hopffam]{e}(\base)}$; hence
% path algebra provides a proof of $t_{\refl{e}}(p) = p$. So in
% particular when $\alpha:\refl N = \refl N$, one gets
% \begin{equation}\label{eq:trp-Ta(refl)}
%   \trp[T]{\alpha}(\refl \base) = \refl \base \cdot \inv{\left(\ap{\trp[\hopffam]{\blank}(\base)}(\alpha)\right)}
%   = \inv{\left(\ap{\trp[\hopffam]{\blank}(\base)}(\alpha)\right)}
% \end{equation}
% In the end, the map $\eta\from \loopspace 2 \Sp \to \loopspace 1 \Sc$
% is $\inv{(\blank)} \circ \ap \theta$ where $\theta : (N=N) \to \Sc$ is the function defined by
% \begin{equation}
%   \label{eq:17}
%   \theta(p) \defequi \trp[\hopffam]{p}(\base)
% \end{equation}
% For practical reason, we prefer to work directly with
% $\ap\theta\from \loopspace 2 \Sp \to \loopspace 1 \Sc$. This does not
% matter much as $\inv{(\blank)}$ is an equivalence
% $\loopspace 1 \Sc \weq \loopspace 1 \Sc$, hence every property that
% $\ap\theta$ will enjoy, $\eta$ will also.

% Recall now the adjoint pair $\susp{}\dashv\loopspace{}$, which
% provides an equivalence
% \begin{equation}
%   \label{eq:24}
%   \begin{tikzcd}[%
%     /tikz/column 1/.append style={anchor=base east},%
%     /tikz/column 2/.append style={anchor=base west}]
%     \blank^\natural : (\Sp \ptdto \Sp) \rar["\weq"] & (\Sc \ptdto
%     \loopspace 1 \Sp)
%     \\
%     (f,f_0) \rar[mapsto] & (\varphi_{f,f_0}, \pi_{f,f_0})
%   \end{tikzcd}
% \end{equation}
% where in the right hand side
% $\varphi_{f,f_0} \defequi\inv{f_0}\cdot \inv{f(\mrd(\base))}\cdot
% f(\mrd(\blank))\cdot f_0$ and the path $\pi_{f,f_0}$ is defined as the
% composition of the following elementary path algebra steps:
% \begin{equation}
%   \label{eq:16}
%   \refl N = \inv{f_0}f_0 = \inv{f_0}\refl {f(N)} f_0 =  \inv{f_0}\inv{f(\mrd(\base))}f(\mrd(\base)) f_0
% \end{equation}
% Remark that for two pointed functions $(f,f_0)$ and $(g,g_0)$, path
% algebra and functoriality of $g$ yields a path, call it
% $c_{f,f_0,g,g_0}$, in:
% \begin{equation}
%   \begin{aligned}
%     \varphi_{gf,g(f_0)g_0} %
%     &= \inv{(g(f_0)g_0)}\inv{gf(\mrd(\base))} gf(\mrd(\blank))
%     g(f_0)g_0
%     \\
%     &= \inv{g_0}\inv{g(f_0)}\inv{gf(\mrd(\base))} gf(\mrd(\blank))
%     g(f_0)g_0
%     \\
%     &= \inv{g_0} \cdot \ap g (\varphi_{f,f_0}(\blank)) \cdot g_0
%     \\
%     &\jdeq \loopspace 1 (g,g_0) \circ \varphi_{f,f_0}
%   \end{aligned}\label{eq:26}
% \end{equation}
% Now, the transport of $\pi_{gf,g(f_0)g_0}$ along $c_{f,f_0,g,g_0}$ in
% the type family $h\mapsto (\refl N = h(\base))$ is simply given by the
% composition $c_{f,f_0,g,g_0}\pi_{gf,g(f_0)g_0}$. Compare it to the
% path $\ap{\loopspace 1{(g,g_0)}}(\pi_{f,f_0})\varpi$ where $\varpi$ is
% the path $\refl N = \inv{g_0}g_0 = \inv{g_0}\refl{g(N)}g_0$ coming
% from path algebra. Both are equal from compatibility of path algebra
% with functoriality of function. In other words:
% \begin{equation}
%   \label{eq:27}
%   {\left((g,g_0)\circ (f,f_0)\right)}^\natural = \loopspace 1 (g,g_0) \circ (f,f_0)^\natural 
% \end{equation}
% And in particular for any pointed map $(f,f_0)$ one has an element of
% \begin{displaymath}
%   (f,f_0)^\natural = \loopspace 1 (f,f_0) \circ
%   (\id_{\Sp},\refl N)^\natural
% \end{displaymath}

% Now recall that the universal property of the circle establishes fro
% any type $A$ an equivalence:
% \begin{equation}
%   \label{eq:25}
%   \begin{tikzcd}[%
%     /tikz/column 1/.append style={anchor=base east},%
%     /tikz/column 2/.append style={anchor=base west}]
%     \lambda_{A}: (\Sc \ptdto A) \rar["\weq"] & \loopspace 1 A
%     \\
%     (h,h_0) \rar[mapsto] & \inv{h_0} \cdot h(\Sloop)\cdot h_0
%   \end{tikzcd}
% \end{equation}
% An inverse equivalence for $\lambda_A$ is simply mapping a loop
% $\ell : \pt_A = \pt_A$ to the map $\Sc \to A$ defined by induction as
% $\base \mapsto \pt_A$ and $\Sloop \mapsto \ell$ pointed by
% $\refl{\pt_A}$. Remark also that for any pointed maps
% $(f,f_0):\Sc \ptdto A$ and $(g,g_0):A \ptdto A$,
% \begin{equation}
%   \label{eq:32}
%   \lambda_A((g,g_0)\circ (f,f_0)) = \inv{g_0} \inv{g(f_0)} g(f(\Sloop)) g(f_0) g_0
%   = \loopspace 1 (g,g_0) ( \lambda_A(f,f_0) )
% \end{equation}
% In particular, when $A$ is $\loopspace 1 \Sp$, one can compose
% $\lambda_{\loopspace 1 \Sp}$ with $\blank^\natural$ to find an
% equivalence from $(\Sp \ptdto \Sp)$ to $\loopspace 2 \Sp$, which,
% through equations~(\ref{eq:27}) and~(\ref{eq:32}) maps a composition
% of the form $(g,g_0)\circ (f,f_0)$ to
% $\loopspace 2 (g,g_0)(\lambda_{\loopspace 1 \Sp}(f,f_0)^\natural)$. In
% particular, it follows that
% \begin{equation}
%   \label{eq:34}
%   \lambda_{\loopspace 1 \Sp}\left((f,f_0)^\natural\right) = \loopspace 2 {(f,f_0)} (\ap{\mrd(\base)\cdot \mrd(\blank)}(\Sloop))
% \end{equation}

% The HoTT-book proves that the map
% $\ap \theta:\loopspace 2 \Sp \to \loopspace 1 \Sc$ is (equivalent to)
% the set-truncation map of $\loopspace 2 \Sp$, or equivalently that
% $\Truncset {\ap \theta}$ is an equivalence. Denote $\Phi$ for the
% equivalence $\loopspace 1 \Sc \weq \ZZ$, and define the degree of a
% pointed function $(f,f_0)$ to be
% \begin{equation}
%   \label{eq:28}
%   d(f,f_0) \defequi \Phi\left(\ap\theta\left(
%       \lambda_{\loopspace 1 \Sp}\left(((f,f_0)^\natural\right))
%     \right)\right)
% \end{equation}
% Hence $\ZZ$ is equivalent to the set-truncation of $(\Sp \ptdto \Sp)$
% and the function $d: (\Sp \ptdto \Sp) \to \ZZ$ is the truncation of
% element under this equivalence. In other words, it means that two
% pointed functions are in the same connected component of
% $\Sp \ptdto \Sp$ if and only if they have the same degree.
% % To conclude that equivalences are merely equal to either $\id_\Sp$ or
% % $-\id_\Sp$, one still need to prove that equivalences have degree
% % either $1$ or $-1$. This will be achieved if we manage to prove that
% % $d$ maps compositions to products in $\ZZ$. For that we shall 

% Now remark that that there is a kind of naturality for the equivalence
% defined in (\ref{eq:25}): namely if $\ell:A \ptdto B$, then it defines
% a map $\ell^\ast : (\Sc\ptdto A) \to (\Sc \ptdto B)$ by
% postcomposition and
% $\ap \ell \circ \lambda_A = \lambda_B \circ \ell^\ast$ is
% inhabited. Define now $\bar d: (\Sp \ptdto \Sp) \to (\Sc \ptdto \Sc)$
% as follows:
% \begin{equation}
%   \label{eq:29}
%   \bar d(f,f_0) \defequi \theta\circ(f,f_0)^\natural
% \end{equation}
% The situation is better summarized in the following commutative
% diagram:
% \begin{equation}
%   \label{eq:30}
%   \begin{tikzcd}
%     {} & (\Sc \ptdto \Sc) \rar["\lambda_{\Sc}","\weq"swap] &
%     \loopspace 1 \Sc \rar["\Phi","\weq"swap] & \ZZ
%     \\
%     (\Sp \ptdto \Sp) \rar["\blank^\natural","\weq"swap] \ar[ur,bend
%     left,"\bar d"] \ar[urrr, bend
%     right=90,"d"swap] & (\Sc \ptdto \loopspace 1 \Sp)
%     \rar["\lambda_{\loopspace 1 \Sp}","\weq"swap] \uar["\theta^\ast"]
%     & \loopspace 2 \Sp \uar["\ap\theta"swap] & 
%   \end{tikzcd}
% \end{equation}

% Because $d$ is a set-truncation map of $\Sp\ptdto\Sp$, then so is
% $\bar d$. Recall from section~\ref{sec:circle-case} that pointed
% equivalences $\Sc \ptdto \Sc$ are equal to either $\id_\Sc$ or
% $-\id_\Sc$.\footnote{\color{red}This only appears in the proof of the
%   main result and not as an independent lemma. It should be
%   corrected.}  And because $\bar d(\id_\Sp) = \id_\Sc$ and
% $\bar d(-\id_\Sp) = -\id_\Sc$ (to be shown in full details later), it
% is enough to prove that $\bar d$ maps equivalences to equivalences to
% conclude. Let us shall that by proving $\bar d$ preserves composition
% of pointed maps. Any pointed map $(f,f_0): \Sp \ptdto \Sp$ induces a
% map
% \begin{equation}
%   \label{eq:33}
%   \pi_2(f,f_0)\defequi \Truncset {\loopspace 2 (f,f_0)}: \Truncset{\loopspace 2 \Sp} \to \Truncset{\loopspace 2 \Sp}
% \end{equation}
% which, by definition, has the property that the following diagram
% commutes:
% \begin{equation}
%   \label{eq:35}
%   \begin{tikzcd}
%     \loopspace 2 \Sp \rar["{\loopspace 2 (f,f_0)}"]
%     \dar["\truncset\blank"swap] & \loopspace 2 \Sp
%     \dar["\truncset\blank"]
%     \\
%     \Truncset{\loopspace 2 \Sp} \rar["{\pi_2 (f,f_0)}"swap] & \Truncset{\loopspace 2 \Sp}
%   \end{tikzcd}
% \end{equation}
% In particular, one has that
% $\ap \theta \circ \loopspace 2 (f,f_0) = \Truncset{\ap\theta} \circ
% \pi_2(f,f_0) \circ \truncset\blank$. Now by definition
% $\Truncset{\ap\theta} \circ \truncset\blank \jdeq \ap \theta$, and
% $\Truncset{\ap\theta}$ being an equivalence, it follows that
% $\truncset \blank = \inv{\Truncset{\ap\theta}}\circ \ap\theta$. Hence,
% if we denote $\bar \pi_2(f,f_0): \loopspace 1 \Sc \to \loopspace 1 \Sc$ for the
% composition
% $\Truncset{\ap\theta} \circ \pi_2(f,f_0) \circ
% \inv{\Truncset{\ap\theta}}$, then through equation~(\ref{eq:34}),
% \begin{equation}
%   \label{eq:36}%
%   (\inv\Phi \circ d) (f,f_0) = \bar\pi_2(f,f_0)\left(
%     \ap\theta \left( \ap {\mrd(\base) \cdot \mrd(\blank)} (\Sloop) \right)
%   \right)
% \end{equation}
% Remember that $\theta\circ (\mrd(\base)\cdot\mrd(\blank)) =
% \id_\Sc$. Then it follows that
% \begin{equation}
%   \label{eq:36}%
%   (\inv\Phi \circ d) (f,f_0) = \bar\pi_2(f,f_0)\left( \Sloop \right)
% \end{equation}
% Through the commutativity of diagram~(\ref{eq:30}), it holds that
% \begin{equation}
%   \bar d(f,f_0) = \inv{\lambda_\Sc}(\bar \pi_2(f,f_0)(\Sloop))
% \end{equation}
% In particular, if $(f,f_0)$ and $(g,g_0)$ are pointed maps
% $\Sp \ptdto \Sp$, then one has:
% \begin{equation}
%   \label{eq:37}
%   \begin{aligned}
%     \bar d ((g,g_0) \circ (f,f_0))
%     &= \inv{\lambda_\Sc} \left(
%       \bar\pi_2((g,g_0) \circ (f,f_0))\left( \Sloop \right)
%     \right)
%     \\
%     &= \inv{\lambda_\Sc} \left(
%       \bar\pi_2(g,g_0)\left( \bar\pi_2(f,f_0)\left(  \Sloop \right) \right)
%     \right)
%   \end{aligned}
% \end{equation}
% It is now easy to prove that the latter map is equal to
% $\inv{\lambda_\Sc}(\bar \pi_2(g,g_0)(\Sloop)) \circ
% \inv{\lambda_\Sc}(\bar \pi_2(f,f_0)(\Sloop))$ by circle induction: as
% both maps $\base$ definitionally to $\base$, one only need to check
% that their action on the loop is the same. By definition,
% $\inv{\lambda_\Sc}(\bar \pi_2(g,g_0)(\bar\pi_2(f,f_0)(\Sloop)))$ acts
% on the loop $\Sloop$ as $\bar
% \pi_2(g,g_0)(\bar\pi_2(f,f_0)(\Sloop))$. While the action of
% $\inv{\lambda_\Sc}(\bar \pi_2(g,g_0)(\Sloop)) \circ
% \inv{\lambda_\Sc}(\bar \pi_2(f,f_0)(\Sloop))$ on the loop leads to the
% application of
% $\loopspace 1 \left(\inv{\lambda_\Sc}(\bar
%   \pi_2(g,g_0)(\Sloop))\right)$ to $\bar\pi_2(f,f_0)(\Sloop)$. Hence,
% it suffices to prove that
% $\varpi_{g,g_0} \defequi\loopspace 1 \left(\inv{\lambda_\Sc}(\bar
%   \pi_2(g,g_0)(\Sloop))\right)$ is equal to $\bar \pi_2(g,g_0)$. They
% both are group morphisms that have the same image on $\Sloop$: for
% each
% $k:\ZZ$, one has
% \begin{equation}
%   \label{eq:39}
%   \varpi_{g,g_0} (\Sloop^k) = \varpi_{g,g_0} (\Sloop)^k
%   = \bar\pi_2(g,g_0)(\Sloop)^k = \bar\pi_2(g,g_0)(\Sloop^k)
% \end{equation}
% This concludes that $\bar d$ maps composition to composition. We have
% also seen that $\bar d (\id_\Sp,\refl N) = (\id_\Sc,\refl\base)$,
% hence $\bar d$ maps pointed equivalences to pointed equivalences.

% To conclude: given any equivalence $f:\Sp\simeq \Sp$, in order to
% prove the proposition
% $P(f)\defequi \Truncprop{\id_\Sp = f} + \Truncprop{\-id_\Sp = f}$, one
% can suppose that $f$ is pointed by a path $f_0$; then $\bar d(f,f_0)$
% is a pointed equivalence $\S \ptdto \Sc$ which is already known to be
% equal to either $(\id_\Sc,\refl\base)$ or
% $(-id_\Sc,\refl\base)$. Moreover $\bar d$ has the property of a
% set-truncation map, meaning that
% $\Truncprop{x=y} \weq (\bar d x = \bar d y)$: so
% $\bar d(f,f_0) = (\id_\Sc,\refl \base) \to \Truncprop{(f,f_0) =
%   (\id_\Sp,\refl N)}$ and
% $\bar d(f,f_0) = (-\id_\Sc,\refl \base) \to \Truncprop{(f,f_0) =
%   (-\id_\Sp,\refl N)}$. It remains to project on the first component
% and to use the universal property of a sum to find $P(f)$.

% %% -----------------------
% %% OLD STUFF, MAYBE USEFUL
% %% -----------------------
% %
% % Now remark that under the equivalence $\Phi\lambda_\Sc$,
% % multiplication in $\ZZ$ corresponds to composition in
% % $\Sc \ptdto \Sc$, as proved in the first section. {\tt \color{red}
% %   (Maybe we should make that a lemma somewhere.)} %
% % Hence, $d$ maps composition to product if and only if $\bar d$
% % preserves composition. Taking equation~(\ref{eq:27}) into account, one
% % needs to prove that for pointed functions $(f,f_0)$ and $(g,g_0)$, the
% % following type is inhabited:
% % \begin{equation}
% %   \label{eq:31}
% %   \theta \circ \loopspace 1 {(g,g_0)} \circ (f,f_0)^\natural
% %   = \theta \circ (g,g_0)^\natural \circ \theta \circ (f,f_0)^\natural  
% % \end{equation}

% % There is a simpler goal though: to prove that $\bar d$ maps pointed
% % equivalences to pointed equivalences. Take $(f,f_0): \Sp \ptdto \Sp$
% % such that $f$ is an equivalence, and write $g$ for its pseudo
% % inverse. We are targeting to prove that $\bar d(f,f_0)$ is an
% % equivalence, so we might assume that $g$ is pointed as well. One wants
% % to prove that for any $y\in \Sc$, the fiber
% % $\inv{\left( \theta \circ \varphi_{f,f_0} \right)} (y)$ is
% % contractible. The latter type depending on $y$ is a proposition, so we
% % take advantage of the connectedness of $\Sc$ and we verify it only on
% % the point $\base$. Because
% % $\pi_{f,f_0}: \refl N = \varphi_{f,f_0}(\base)$, the transport over
% % $\varphi_{f,f_0}(\base)$ in $\hopffam$ is trivial and the fiber
% % $\inv{\left( \theta \circ \varphi_{f,f_0} \right)} (\base)$ contains
% % the element $(\base, \ap \theta (\pi_{f,f_0}))$. For any other point
% % $(x,p)$ of this fiber, one gets
% % $p: \base = \theta(\varphi_{f,f_0}(x))$.
% %
% % Notice that for any $p,q:N=f(S)$,
% % \begin{equation}
% %   \inv{q} \cdot p = \inv{\mrd(\base)} \cdot (\inv{(q\circ \inv{\mrd(\base)})} \cdot p)
% % \end{equation}
% % Hence, if we denote
% % $c_{f,f_0}: r \mapsto \inv{(f(\mrd(\base))\cdot
% %   f_0\cdot\inv{\mrd(\base)})} \cdot r \cdot f_0$, then the equivalence
% % of (\ref{eq:24}) is equal to
% % $(f,f_0) \mapsto (\inv{\mrd(\base)}\cdot
% % c_{f,f_0}(f(\mrd(-))),\pi_{f,f_0})$ where
% % $\pi_{f,f_0} = \ap{\inv{\mrd(\base)}\cdot \blank}(\varpi_{f,f_0})
% % \cdot \varepsilon$ where
% % $\varepsilon: \refl N = \inv{\mrd(\base)}\cdot \mrd(\base)$ and
% % $\varpi_{f,f_0}:\mrd(\base) = c_{f,f_0}(f(\mrd(\base)))$ are both
% % paths from path algebra.

% % Then the composition of equivalences (\ref{eq:24}) and (\ref{eq:25})
% % is equal to the following equivalence:
% % \begin{equation}
% %   \begin{tikzcd}[%
% %     /tikz/column 1/.append style={anchor=base east},%
% %     /tikz/column 2/.append style={anchor=base west}]
% %     (\Sp \ptdto \Sp) \rar["\weq"] & \loopspace 2 \Sp
% %     \\
% %     (f,f_0) \rar[mapsto] & \inv{\pi_{f,f_0}} \cdot \left(
% %       \inv{\mrd(\base)} \cdot \blank \right)
% %     \left(c_{f,f_0}(f(\mrd(\Sloop)))\right) \cdot \pi_{f,f_0}
% %   \end{tikzcd}
% % \end{equation}

% % The degree function $d: (\Sp \ptdto \Sp) \to \loopspace 1 \Sc$ is then
% % defined as the composition of this equivalence with $\ap\theta$. It
% % holds that ({\tt \color{red} No it does not, need to correct up until
% %   equation (\ref{eq:15})}):
% % \begin{align}
% %   \label{eq:21}
% %   d(f,f_0)
% %   &=
% %     \ap\theta\left(
% %     \left(
% %     \inv{\mrd(\base)} \cdot \blank
% %     \right)
% %     \left(
% %     c_{f,f_0}(f(\mrd(\Sloop)))
% %     \right)
% %     \right)
% %   \\
% %   &= \ap{\theta(\inv{\mrd(\base)} \cdot \blank)}
% %     \left(
% %     c_{f,f_0}(f(\mrd(\Sloop)))
% %     \right)
% % \end{align}
% % Now remark that $\trp{H,\mrd(\base)} = \id_\Sc$, so that
% % $\theta(\inv{\mrd(\base)}\cdot \blank)$ is equal to the function:
% % \begin{equation}
% %   \label{eq:22}
% %   \tau : (N=S) \to \Sc, \quad p \mapsto \trp{\hopffam,p}(\base)
% % \end{equation}
% % Hence we can simplify the expression of the degree function as:
% % \begin{equation}
% %   \label{eq:23}
% %   d(f,f_0) = \ap\tau\left(c_{f,f_0}f(\mrd(\Sloop))\right)
% % \end{equation}

% % \begin{definition}
% %   Given a map $f\from \Sp \to \Sp$ and a path $f_0 : N = f(N)$, define
% %   $\phi(f,f_0)$ to be the composition
% %   \begin{equation}
% %     \label{eq:13}
% %     \begin{tikzcd}
% %       \Sc \rar["\mrd"] & N=S \rar["f"] & f(N) = f(S) \rar["c_f"] & N =
% %       S \rar["\tau"] & \Sc
% %     \end{tikzcd}
% %   \end{equation}
% % \end{definition}
% % One can sum up what is above as follows:
% % \begin{equation}
% %   \label{eq:15}
% %   d = \ap{\phi(\blank)}(\Sloop)
% % \end{equation}

% % Remark that the function $\tau$ is actually a retraction for $\mrd$:
% % indeed, $\trp{\hopffam, \mrd(x)} = \hopffam (\mrd(x)) = \iota_1(x)$
% % and $\iota_1(x)$ sends $\base$ to $x$ by definition, so
% % $\tau(\mrd(x)) = x$ for every $x:\Sc$. As $c_{\id,\refl N}=\id_{N=S}$,
% % it follows that:
% % \begin{equation}
% %   d(\id, \refl N) = \ap\tau(\ap\mrd(\Sloop)) = \Sloop
% % \end{equation}





% \subsection{Symmetries of the sphere}
% \label{sec:symmetries-sphere}

% Four equivalences $\Sp\to\Sp$ that will be used in the sequel are
% defined as follows. 
% \begin{align}
%   \label{eq:S2symmetries}
%   e_0(x)&\jdeq x \quad\text{the identity equivalence}\\
% \nonumber\\
%   e_1(N)&\jdeq N \quad\text{keeping the poles}\\
%   e_1(S)&\jdeq S\\
%   e_1(\mrd(\base))&= \mrd(\base)\\
%   e_1(\mrd(\Sloop))&= \mrd(\Sloop^{-1})\quad\text{reversing rotation of meridians}\\
% \nonumber\\
%   e_2(N)&\jdeq S \quad\text{flipping the poles}\\
%   e_2(S)&\jdeq N\\
%   e_2(\mrd(x))&= \mrd(x)^{-1}\quad\text{reversing the meridians}\\
% \nonumber\\
%   e_3(N)&\jdeq S \quad\text{flipping the poles}\\
%   e_3(S)&\jdeq N\\
%   e_3(\mrd(\base))&= \mrd(\base)^{-1}\quad\text{reversing the meridians, and}\\
%   e_3(\mrd(\Sloop))&= \ap{\mrd(\_)^{-1}}(\Sloop^{-1})\quad\text{reversing their rotation}
% \end{align}
% One would expect $e_1=e_2$ and $e_0=e_3$ and both will turn out to be true.

% In the sequel we will have to deal with the following situation.
% Let $A$ be a type with elements $a, b, c : A$ and 1- and 2-paths as follows:
% %
% \begin{align*}
%   p &: a = b       &       r &: b = c \\
%   q &: a = b       &       s &: b = c \\
%   \alpha &: p = q  &   \beta &: r = s
% \end{align*}
% %
% We define the \emph{horizontal composition} of $\alpha$ and $\beta$ as 
% a path $\beta\cdot_h\alpha: rp=sq$.
% First observe that $\ap{r\_}(\alpha):rp=rq$.
% Then observe that $\ap{\_q}(\beta):rq=sq$.
% Now define $\beta\cdot_h\alpha \defeq \ap{\_q}(\beta)\ap{r\_}(\alpha)$,
% the ordinary, \emph{vertical} composition of the previous two paths,
% which indeed gives a path $rp=sq$. Horizontal composition enjoys many of
% properties of ordinary composition, modulo some easy equivalences.
% For example, in the situation above, if $\beta\jdeq\refl{r}$,
% the type of $\alpha$ is $p=q$, whereas the type of $\refl{r}\cdot_h\alpha$
% is $rp=rq$. However, by induction on $r$ one easily gets an equivalence
% $e_r : (p=q) \to (rp=rq)$ and proves $e_r(\alpha)=\refl{r}\cdot_h\alpha$.
% Similar equivalences can be constructed for associativity and other laws
% of path algebra for horizontal composition. In order to stay well-typed
% we mention the equivalences, but leave their construction to the reader.

% We apply horizontal composition in the following situation:
% %
% \begin{align*}
%   p &: a = b       &   p^{-1}       &: b=a           & q     &: a = b \\
%   q &: a = b       &                &                & p     &: a = b \\
%   \alpha &: p = q, &   \refl{p^{-1}}&: p^{-1}=p^{-1} & \beta &: q = p
% \end{align*}
% % 
% In this situation, the composition $\alpha\beta$ has type $q=q$.
% The horizontal composition $(\beta\cdot_h\refl{p^{-1}})\cdot_h\alpha$
% has type $(qp^{-1})p = (p^{-1}p)q$. By induction on $p$ one constructs
% an equivalence 
% \[
% e_p: (q=q)\to((qp^{-1})p = (p^{-1}p)q)
% \]
% such that (note the change of order) 
% \[
% e_p(\alpha\beta)=(\beta\cdot_h\refl{p^{-1}})\cdot_h\alpha.
% \]
% The latter equality is easily proved by induction on $\beta$.

% Remark: taking $q^{-1}$ instead of $p^{-1}$ above yields a version
% in which the order of $\alpha$ and $\beta$ does  not change.



% \begin{lemma} We have $e_1 = e_2$ and $e_0 \neq e_2$ and $e_0 = e_3$.
% \end{lemma}

% \begin{proof}
% We construct $f(x):T(x)\defeq(e_1(x)=e_2(x))$ for all $x:\Sp$ by 
% suspension induction. We take $n\defeq\mrd(\base):(N=S)\jdeq T(N)$ and 
% $s\defeq\mrd(\base)^{-1}:(S=N)\jdeq T(S)$. 
% We have to give $m_f(z): n=^T_{\mrd(z)}s$
% for all $z:\Sc$. The latter type is equivalent to
% $P(z)\defeq(\mrd(z)^{-1}\cdot n\cdot e_1(\mrd(z))^{-1} = s)$.
% We apply circle induction:
% $m_f(\base): P(\base)$ can easily be given by path algebra,
% as $P(\base)$ is equivalent to 
% $\mrd(\base)^{-1}\cdot n\cdot\mrd(\base)^{-1} = s$.
% It remains to give $m_f(\Sloop): m_f(\base)=^P_{\Sloop}m_f(\base)$.

% Note first that $P$ is a identity type family with a function on the left
% and a constant on the right. This means that transport in $P$ along $p$
% is given by prefixing as follows:
% \[
% \trp{p}(q) = q \cdot (\mrd(\_)^{-1}\cdot n \cdot e_1(\mrd(\_))^{-1})(p)^{-1}.
% \]
% By induction on $p:\base=z$ one proves that
% \[
% (\mrd(\_)^{-1}\cdot n \cdot e_1(\mrd(\_))^{-1})(p) = 
% \_\!\!^{-1}(\mrd(p)) \cdot_h \refl{n} \cdot_h\  \_\!\!^{-1}(e_1(\mrd(p))).
% \]
% \begin{center}
%   \begin{tikzpicture}
%     \matrix (m) [matrix of nodes, column sep=9em] {
%       S & N & S & N \\
%     };%
%     \draw[->] (m-1-1.north) to[bend left=60] node[above, name=Afrom] {$\inv{\mrd(\base)}$} (m-1-2.north);%
%     \draw[->] (m-1-1.south) to[bend right=60] node[below, name=Ato] {$\inv{e_1(\mrd(z))}$} (m-1-2.south);%
%     \draw[cell] (Afrom) to node[fill=white] {$\inv{\_}(e_1(\mrd(p)))$} (Ato);%
%     \draw[->] (m-1-2) to node[above] {$\mrd(\base)$} (m-1-3);%
%     \draw[->] (m-1-3.north) to[bend left=60] node[above, name=Bfrom] {$\inv{\mrd(\base)}$} (m-1-4.north);%
%     \draw[->] (m-1-3.south) to[bend right=60] node[below, name=Bto] {$\inv{\mrd(z)}$} (m-1-4.south);%
%     \draw[cell] (Bfrom) to node[fill=white] {$\inv{\_}(\mrd(p))$} (Bto);%
%   \end{tikzpicture}
% \end{center}

% Taking $p\jdeq{\Sloop}$ gives
% $\_\!^{-1}(\mrd(\Sloop)) \cdot_h \refl{n} \cdot_h\  \_\!^{-1}(\mrd(\Sloop^{-1}))$,
% which is the special case of the horizontal composition that can be further
% simplified to
% \begin{align}
%   \label{eq:trp-for-e1=e2}
% e_{\mrd(\base)}(\_\!^{-1}(\mrd(\Sloop^{-1})) \cdot\_\!^{-1}(\mrd(\Sloop)) = 
% e_{\mrd(\base)}(\_\!^{-1}(\refl{\mrd(\base)})).
% \end{align}
% The latter path is a reflexivity path, so that $\trp{\Sloop}$ is actually homotopic 
% to the identity function. Hence we can take for 
% $m_f(\Sloop): m_f(\base)=^P_{\Sloop}m_f(\base)$ a simple transport of $\refl{m_f(\base)}$.

% For the second statement of the lemma, $(e_0 = e_2)\to\false$,
% assume we are given $f(x):R(x)\defeq(e_0(x)=e_2(x))$ for all $x:\Sp$.
% We are to prove $\false$. Since the goal is a proposition,
% and $N=S$ is connected (as $\pi_1 \Sp$ is contractible),
% we can assume $f(N)=n\jdeq\mrd(\base):(N=S)\jdeq R(N)$ and 
% $f(S)=s\jdeq\mrd(\base)^{-1}:(S=N)\jdeq R(S)$.
% This allows us to reuse parts of the proof of $e_1 = e_2$.
% Modulo some transport we have $f(\mrd(z)): n=^R_{\mrd(z)}s$
% for all $z:\Sc$. The latter type is equivalent to
% $Q(z)\defeq(\mrd(z)^{-1}\cdot n\cdot \mrd(z)^{-1} = s)$.
% Thus we get in particular $f'(\mrd(\base)): Q(\base)$ and 
% $f'(\mrd(\Sloop)): f'(\mrd(\base))=^Q_{\Sloop}f'(\mrd(\base))$
% for some transport $f'$ of $f$.
% Transport in the family $Q$ goes like transport in $P$ with $e_1$
% replaced by $e_0$, which is $\id$. 
% This means that the rhs of Eq.\ref{eq:trp-for-e1=e2} becomes
% $p_{02}\defeq e_{\mrd(\base)}(\_\!^{-1}(\mrd(\Sloop\cdot\Sloop)))$.
% Transport in $Q$ is equivalent to precompostion with the inverse of $p_{02}$,
% and hence $f'(\mrd(\Sloop)): f'(\mrd(\base))=^Q_{\Sloop}f'(\mrd(\base))$
% leads to a contradiction if we show
% $\mrd(\Sloop\cdot\Sloop) \neq \mrd(\refl{\base})$.
% Assume $\mrd(\Sloop\cdot\Sloop) = \mrd(\refl{\base})$.
% These are 2-paths of type $\mrd(\base)=\mrd(\base)$.
% Since $\hopffam(\mrd(x))(\base) = \iota_1(x)(\base) = x$ for all $x:\Sc$
% it follows that $\Sloop\cdot\Sloop = \refl{\base}$, which is absurd.

% The proof of $e_0 = e_3$ is similar to that of $e_1 = e_2$, with some
% technical adjustments for the different functions. We leave these to the
% reader. We come back to this when we discuss $\Trunc{e_0 = e_3}$.
% \end{proof}


% \begin{example}\label{exa:degree-k-function}
% We define a canonical pointed map and calculate its degree.
% \begin{align}
%   \label{eq:deg-k-func}
%   d_k(N)&\jdeq N \quad\text{keeping the poles, $\refl{N}: N=d_k(N)$}\\
%   d_k(S)&\jdeq S\\
%   d_k(\mrd(\base))&= \mrd(\base)\\
%   d_k(\mrd(\Sloop))&= \mrd(\Sloop^{k})\quad\text{$k$-fold rotation of meridians}
% \end{align}
% The corresponding function $\varphi_{d_k,\refl{N}}: \Sc \to (N=N)$ is:
% \[
% {\refl{N}}^{-1} \cdot d_k(\mrd(\base))^{-1}\cdot d_k(\mrd(\blank))\cdot \refl{N}
% \]
% We postcompose $\varphi_{d_k,\refl{N}}$ with $\trp[\hopffam]{\blank}(\base)$
% in order to get a function $d'_k :  \Sc \to \Sc$.
% Transport along the refexivity paths is the identity. 
% Transport in $\hopffam$ along $d_k(\mrd(\base))^{-1}$ is $\iota_{\base}$,
% which is also the identity. It remains to transport along $d_k(\mrd(\blank))$.
% Here we cannot say much in general. 
% However, we easily get $d'_k(\base)=\base$.
% We use again that
% $\hopffam(\mrd(x))(\base) = \iota_1(x)(\base) = x$ for all $x:\Sc$.
% Consequently, $\trp[\hopffam]{ \mrd(\Sloop^{k})} =  \Sloop^{k}$,
% and $d'_k(\Sloop)$ is a conjugate of $\Sloop^{k}$.
% This makes sense as a degree $k$ function, 
% and includes $e_0$ ($k=1$) and $e_1$ ($k=-1$).


% \end{example}


% \section*{Acknowledgements}
% We thank Nicolai Kraus for the insight that
% $\mrd(\Sloop\cdot\Sloop) \neq \mrd(\refl{\base})$.

\end{document}

% LocalWords: isomorphisms automorphisms morphisms
 
