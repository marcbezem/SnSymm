\documentclass[english,a4paper]{tufte-handout}

\usepackage{fontspec}
\usepackage{amsmath,amsthm,amssymb}
\usepackage{unicode-math}
\usepackage{polyglossia}
\usepackage{enumitem}
\usepackage{tikz,tikz-cd}
\usepackage{hyperref}
\usepackage{microtype}
%\usepackage[explicit]{titlesec}
\usepackage[mathscr]{euscript}
\usepackage{mathtools}

\input macros%
\input macros2%

\setmainlanguage{english}
\setotherlanguage{french}

\setmainfont{Linux Libertine O}[Ligatures={TeX,Rare,Historic,Common}]%
\setsansfont{Roboto Light}[Scale=MatchLowercase,BoldFont=Roboto
Medium]%
\setmonofont{Ubuntu Mono}[Scale=MatchLowercase]%
\setmathfont{Libertinus Math Regular}%[Scale=MatchLowercase]%
\setmathfont{TeX Gyre Pagella Math}[range={\mathfrak,\rightleftarrows,\rightsquigarrow}]%

\newfontfamily\custommathsffont{Lato Regular}[Scale=MatchLowercase,NFSSFamily=custommathsf]
\DeclareMathAlphabet{\mymathsf}{\encodingdefault}{custommathsf}{m}{n}

\hypersetup{
    colorlinks,
    linkcolor={red!50!black},
    citecolor={blue!50!black},
    urlcolor={blue!80!black}
}

\setlist[enumerate]{label=(\roman*)}

% Set up the spacing using fontspec features
\renewcommand\allcapsspacing[1]{{\addfontfeature{LetterSpace=15}#1}}
\renewcommand\smallcapsspacing[1]{{\addfontfeature{LetterSpace=10}#1}}

% \colorlet{custom}{black!90}
% \titleformat{\section}[hang]{}{}{0pt}{%
%   \vskip 3pt%
%   {\hskip -\marginparsep \color{custom!90}\Large\bfseries \thesection \hskip \marginparsep%
%     \tikz[remember picture, overlay]{%
%       \path (0,.15) coordinate (start) -| coordinate (stop) ([xshift=-(\marginparwidth+\marginparsep)]current page.east);%
%       \draw[custom!50, line width=3pt] (start) -- (stop);
%     }%
%   }%
%   \hskip 1em%
%   {\tikz[overlay,remember picture]{%
%       \node[fill=white,inner sep=1em,anchor=west,rounded corners] at ([xshift=\marginparsep]start) {%
%         \color{custom}\Large \sffamily \bfseries #1%
%       };%
%     }%
%   }
% }[]


\title{{\sffamily \bfseries The symmetries of the circle in univalent
    foundations}}%
\author{\normalfont Pierre Cagne} %
\date{\normalsize Last updated on \today}%

%% global tikz styles
\tikzset{cell/.style={%
    shorten <=1em,%
    shorten >=1em,
    /tikz/commutative diagrams/Rightarrow
  }%
}%

\def\U{\universe U}%

\begin{document}

\maketitle

\begin{abstract}
  The goal of these notes in to rigorously prove the equivalence of
  types
  \begin{displaymath}
    (\Sc = \Sc) \weq (\Sc + \Sc)
  \end{displaymath}
  in univalent mathematics.
\end{abstract}

\newthought{In this document,} we shall establish an equivalence
between the type of equalities of $\Sc$ with itself and the sum type
of $\Sc$ with itself. We work in intuitionistic Martin-Löf's type
theory with $\Sigma$-, $\Pi$- and $\mathrm{Id}$- types and with a
cumulative hierarchy of universes, simply written $\U$, for
which Voevodsky's univalence axiom hold. Moreover we introduce the
circle as a higher inductive type, meaning we postulate the existence
of a type $\Sc$ together with terms:
\begin{align}
  \label{eq:1}
  \base &: \Sc\\
  \Sloop &: \base =_{\Sc} \base
\end{align}
This type enjoys a inductive property: given a type family
$T : \Sc \to \U$, an element $t:T(\base)$ and a path
$p:t=_{T(\base)}t$, the $\Sc$-induction property provides a dependent
function $f : \prod_{x:\Sc}T(x)$ such that $f(\base) \jdeq t$ and an
element $q: f(\Sloop) = p$. In particular, $\Sc$-induction provides a
map
\begin{equation}
  \label{eq:2}
  \left(\sum_{t:T(\base)} \pathover t T \Sloop t\right) \to \prod_{x:\Sc}T(x)
\end{equation}
that is easily shown to be an equivalence (through $\Sc$-induction
again). In the case $T$ is the family constant in $A:\U$, it gives the
following universal property of the circle:
\begin{equation}
  \label{eq:3}
  \left(\sum_{t:A}t=t\right) \weq \left( \Sc \to A \right)
\end{equation}

\newthought{Using univalence}, one gets that $(\Sc = \Sc)$ is
equivalent to $(\Sc \weq \Sc)$ and we shall then concentrate on
proving
\begin{equation}
  (\Sc \weq \Sc) \weq (\Sc+\Sc)
\end{equation}
As usual, we shall denote $f:\Sc\weq\Sc$ for a map $f:\Sc\to \Sc $
such that the proposition $\isEq(f)$ is non-empty instead of the more
honest $(f,!):\Sc \weq \Sc$. This is harmful for that there is an
equivalence:
\begin{equation}
  ((f,!) =_{\Sc \weq \Sc} (g,!)) \simeq (f=_{\Sc \to \Sc}g)
\end{equation}
% First define a maps $\iota_1: \Sc \to (\Sc \weq \Sc)$ through
% equivalence~(\ref{eq:3}) by:
% \begin{enumerate}
% \item first, let $\iota_1 (\base) \defequi \id_{\Sc}$,
% \item now define $\iota_1(\Sloop) : \id_{\Sc} = \id_{\Sc}$ by function
%   extensionality as an element of $\prod_{x:\Sc}x = x$; in order to do
%   so, one can use $\Sc$-induction by providing
%   $\Sloop : \base = \base$ and then a pathover
%   $p:\pathover {\Sloop} {} \Sloop {\Sloop}$; the transport
%   over $\Sloop$ in the type family $x\mapsto (x=x)$ being simply
%   $\Sloop \blank {\inv\Sloop}$, one can take $p$ to be the witness of
%   path algebra that indeed $\Sloop \Sloop {\inv \Sloop} = \Sloop$.
% \end{enumerate}

Recall that for each $x:\Sc$ one has an equivalence:
\begin{equation}
  \label{eq:8}
  f_x : (\base = \base) \to (x=x)
\end{equation}
Indeed, such a map $x\mapsto f_x$ is defines by $\Sc$-induction by
setting $f_{\base} \defequi \id$ and $f_{\Sloop}$ to be the proof that
postcomposing with the conjugation $\Sloop \blank \inv\Sloop$ is the
identity. In particular one can easily prove that
$f_x(\Sloop^k) = f_x(\Sloop)^k$ for any $k:\zet$. Now define $\iota_1$
and $\iota_2$ on each $x:\Sc$ as follows:
\begin{align*}
  \iota_1(x) (\base) &\defequi x & \iota_2(x) (\base) &\defequi x,
  \\
  \iota_1(x)(\Sloop) &\defis f_x(\Sloop) &  \iota_2(x)(\Sloop) &\defis f_x(\Sloop)^{-1}
\end{align*}
Each $\iota_j(x)$ ($j\in\{1,2\}$) is indeed an equivalence: $\Sc$
being connected, the proposition $\isEq(\iota_j(x))$ has to be proved
for only one point, say $\base$; but $\iota_1(\base)$ is equal to
$\id_\Sc$ (hence an equivalence) and $\iota_2(\base)$, that we call
$-\id_\Sc$ in the following, is its own pseudo inverse.

The maps $\iota_1$ and $\iota_2$ induce a maps
$\iota \defequi \langle\iota_1,\iota_2\rangle : \Sc + \Sc \to (\Sc\weq
\Sc)$.
\begin{theorem*}
  The map $\iota$ is an equivalence.
\end{theorem*}
%
Before proving the theorem, here is a lemma that already explains that
the two components of $\Sc+\Sc$ are mapped to distinct parts of
$(\Sc \weq \Sc)$.
\begin{lemma}
  \label{lemma:id-not-equal-oppid}%
  The type $(\id_\Sc = -\id_\Sc)$ is empty.
\end{lemma}
\begin{proof}
  By function extensionality and $\Sc$-induction, an element
  $p : \id_\Sc = -\id_\Sc$ is given by the data of
  $p(\base) : \id_\Sc(\base) = -\id_\Sc(base)$, that is
  $p(\base):\base=\base$, together with a pathover
  $p(\Sloop): \pathover {p(\base)} {} \Sloop {p(\base)}$ in the type
  family $x\mapsto \id_\Sc (x)=-\id_\Sc (x)$. The transport over
  $\Sloop$ in this type family is given by
  $\inv\Sloop\blank\inv\Sloop$, so that $p(\Sloop)$ is equivalently
  given by an element of $\inv\Sloop p(\base) \inv\Sloop =
  p(\base)$. By expliciting $p(\base)$ as $\Sloop^k$ for some
  $k:\zet$, the last equation becomes
  $\inv \Sloop \Sloop^k \inv \Sloop = \Sloop^k$, leading to the
  contradiction $2=0$ in $\zet$.
\end{proof}

\begin{proof}[Proof of the theorem]
  Take an element $\phi : \Sc \weq \Sc$ and consider the fiber of
  $\iota$ at $\phi$:
  \begin{equation}
    \label{eq:fiber-iota-two-components}%
    \inv\iota(\phi) \jdeq \sum_{x:\Sc+\Sc} \phi = \iota(x)
    \weq \left(\sum_{x:\Sc}\phi = \iota_1(x)\right) +
    \left(\sum_{x:\Sc}\phi = \iota_2(x)\right)
  \end{equation}
  One wants to prove the proposition $\iscontr(\inv\iota(\phi))$,
  hence one may assume a path $q:\phi(\base) = \base$ instead of
  merely $\Trunc{\phi(\base) = \base}$. By the description of the
  symmetries of $\base:\Sc$, there is $k:\zet$ such that
  $q\phi(\Sloop)\inv q = \Sloop^k$. Denoting $\psi$ for a pseudo
  inverse of the equivalence $\phi$, one also gets
  $q':\psi(\base) = \base$ and subsequently $\ell:\zet$ such that
  $q'\psi(\Sloop)\inv{q'} = \Sloop^\ell$. Doing some path algebra it
  follows that:
  \begin{equation}
    \label{eq:4}
    \psi(\phi(\Sloop)) = \inv{(q' \psi(q))}\Sloop^{\ell k} (q' \psi(q))
  \end{equation}
  By hypothesis, one has also $\tau : \prod_{x:\Sc}\psi\phi(x) = x$,
  so there is a pathover of type
  $\pathover {\tau(\base)} {} \Sloop {\tau(\base)}$ in the type family
  $x \mapsto \psi(\phi(x)) = x$. The transport over $\Sloop$ in this
  family is given by $\Sloop \blank \inv{(\psi\phi(\Sloop))}$, so that
  \begin{equation}
    \label{eq:5}
    \Sloop \tau(\base) \inv{(\psi\phi(\Sloop))} = \tau(\base)
  \end{equation}
  Putting equations~(\ref{eq:5}) and~(\ref{eq:4}) together, one find
  \begin{equation}
    \label{eq:6}
    \inv{(q' \psi(q) \inv{\tau(\base)})} \Sloop^{\ell k} (q' \psi(q) \inv{\tau(\base)}) = \Sloop
  \end{equation}
  However, conjugation in $\base = \base$ is equal to the identity, so
  that in the end $\Sloop^{\ell k} = \Sloop$, or otherwise put
  $\ell k = 1$ in $\zet$. Using the decidability of equality on
  $\zet$, one can show that the only invertible elements of $\zet$ are
  $1$ and $-1$. So $k = 1$ or $k = -1$. We proceed by case:
  \begin{enumerate}
  \item If $k=1$, then $Q \defequi (q, !)$ is element of
    $\phi = \id_\Sc$, where the term $!$ of the proposition
    $q=\Sloop q \inv{\phi(\Sloop)}$ is given by path algebra from the
    hypothesis $q\phi(\Sloop)\inv q=\Sloop^k$ ($k$ being $1$ here). We
    can then only show $\iscontr(\inv\iota(\id_\Sc))$ to conclude (by
    transport over $\inv Q$) that $\iscontr(\inv\iota(\phi))$. 

    One can immediately see that the second component of the right
    hand-side in equation~(\ref{eq:fiber-iota-two-components}) is
    empty when instantiated with $\id_\Sc$ for $\phi$: indeed,
    $\false$ being a proposition and $\Sc$ being connected, one only
    need to check that $\id_\Sc = \iota_2(\base)$ is contradictory;
    and because $\iota_2(\base) \jdeq -\id_\Sc$, it follows from
    lemma~\ref{lemma:id-not-equal-oppid}.

    Hence it remains to prove that $\sum_{x:\Sc}\id_\Sc=\iota_1(x)$ is
    contractible. For $x:\Sc$, the type $\id_\Sc = \iota_1(x)$ is
    equivalent (through first function extensionality and then
    $\Sc$-induction) to
    $\sum_{p:\base=\iota_1(x)(\base)}\pathover p T \Sloop p$ where $T$
    is the type family $y\mapsto y=\iota_1(x)(y)$. Now
    $\iota_1(x)(\base)\jdeq x$ and the transport over the loop
    $\Sloop$ in the type family $T$ is given by
    $\iota_1(x)(\Sloop) \blank \inv \Sloop$. In other words,
    \begin{equation}
      \label{eq:10}
      (\id_\Sc = \iota_1(x)) \weq \sum_{p:\base = x}f_x(\Sloop)p = p {\Sloop}
    \end{equation}
    By induction on $p:\base=x$, one can see that the proposition
    $f_x(\Sloop)p=p\Sloop$ is always satisfied. In the end, one has
    \begin{equation}
      \label{eq:11}
      \left(\sum_{x:\Sc}\id_\Sc=\iota_1(x)\right)
      \weq
      \left(\sum_{x:\Sc}\base=x\right)
    \end{equation}
    The type on the right is a singleton, hence contractible. This
    conclude the case $k=1$.
    % We claim that $(\base, \refl {\id_\Sc})$ is a center of
    % contraction. That is one wants to prove that for all $x:\Sc$,
    % there is an element of
    % \begin{equation}
    %   \label{eq:7}
    %   T(x) \defequi \prod_{p:\id_\Sc=\iota_1(x)}(\base,\refl {\id_\Sc})=(x,p)
    % \end{equation}
    % We shall use $\Sc$-induction by first providing such an element
    % $\kappa$ for the case $x\jdeq \base$, and then providing a
    % pathover in $\pathover \kappa {} \Sloop \kappa$:
    % \begin{enumerate}[label=\arabic*.]
    % \item given $p:\id_\Sc=\iota_1(\base)\jdeq \id_\Sc$, consider the
    %   path $p(\base): \base = \base$; then the transport of
    %   $\refl {\Sc}$ above that path is the composite path
    %   $p \refl{\id_\Sc}$ which is equal to $p$ through path
    %   algebra. In other term, we found an element of
    %   $\kappa: (\base,\refl {\id_\Sc})=(\base,p)$, namely:
    %   \begin{displaymath}
    %     \kappa\defequi (p(\base), ! : p\refl{\id_\Sc} = p)
    %   \end{displaymath}
    % \item We need to find an element of
    %   $\trp {T,\Sloop} (\kappa) = \kappa$. But $T$ is a type family of
    %   the form $x\mapsto \prod_{y:B(x)}{C(x,y)}$ with $B: S^1 \to \U$
    %   and $C : (\sum_{x:S^1}B(x)) \to \U$. Hence one can use heavy
    %   transport and
    %   $\trp{T,\Sloop}\kappa = (p \mapsto \trp {\Sloop} \kappa
    %   (\trp{\Sloop} p) )$         
    % \end{enumerate}
  \item The case $k=-1$ is completely similar to the case $k=1$. We
    now have a proof that $\phi = -\id_\Sc$, hence we shall prove that
    $\iscontr(\inv\iota(-\id_\Sc))$. From there one can determine that
    the first component of in (\ref{eq:fiber-iota-two-components}) is
    empty and concentrate on proving that
    $\sum_{x:\Sc}-\id_\Sc = \iota_2(x)$. The type
    $-\id_\Sc = \iota_2(x)$ is equivalent to
    $\sum_{p:\base = x}\inv{f_x(\Sloop)}p=p\inv\Sloop$ which is again
    equivalent to $\base=x$. We conclude the proof in the same way by
    recognizing a singleton.
  \end{enumerate}
\end{proof}

\end{document}

% LocalWords: isomorphisms automorphisms morphisms
 