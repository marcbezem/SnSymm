\documentclass[11pt,a4paper,oneside,reqno]{amsart}

\pdfoutput=1

% \usepackage{todonotes}

\usepackage{amsmath,amssymb,amsfonts,amsthm}
\usepackage{mathtools}
% \usepackage{mathalfa}
\usepackage[T1]{fontenc}
% \usepackage{enumitem}
\usepackage{hyperref}
\usepackage[capitalise]{cleveref}
% \usepackage{upgreek}
% \usepackage{bbm}

\usepackage{tikz}
\usetikzlibrary{positioning,arrows}

\newtheorem{theorem}{Theorem}
\newtheorem{lemma}[theorem]{Lemma}
\newtheorem{corollary}[theorem]{Corollary}
\theoremstyle{definition}
\newtheorem{definition}[theorem]{Definition}
\newtheorem{example}[theorem]{Example}
\theoremstyle{remark}
\newtheorem{remark}[theorem]{Remark}
\newtheorem{conjecture}[theorem]{Conjecture}


\newcommand{\Nat}{\mathbb{N}}
\newcommand{\Z}{\mathbb Z}
\newcommand{\Prop}{\mathsf{Prop}}
\newcommand{\UU}{\mathcal{U}}
\newcommand{\sph}[1]{{\mathbb S}^{#1}}


\newcommand{\defeq}{\vcentcolon\equiv}  
\newcommand{\refl}{\mathsf{refl}}
\newcommand{\ct}{%
  \mathchoice{\mathbin{\raisebox{0.5ex}{$\displaystyle\centerdot$}}}%
             {\mathbin{\raisebox{0.5ex}{$\centerdot$}}}%
             {\mathbin{\raisebox{0.25ex}{$\scriptstyle\,\centerdot\,$}}}%
             {\mathbin{\raisebox{0.1ex}{$\scriptscriptstyle\,\centerdot\,$}}}
}

\newcommand{\istype}[1]{\mathsf{is}\mbox{-}{#1}\mbox{-}\mathsf{type}}
\newcommand{\trunc}[2]{\mathopen{}\left\Vert #2\right\Vert_{#1}\mathclose{}}
\newcommand{\brck}[1]{\trunc{}{#1}}

\newcommand{\tproj}[3][]{\mathopen{}\left|#3\right|_{#2}^{#1}\mathclose{}}
\newcommand{\bproj}[1]{\tproj{}{#1}}

\newcommand{\isequiv}{\mathsf{isequiv}}

\newcommand{\North}{\mathsf N}
\newcommand{\South}{\mathsf S}
\newcommand{\merid}{\mathsf{merid}}
\newcommand{\pointedm}{\rightarrow_\bullet}
\newcommand{\lo}{\mathsf{loop}}

\newcommand{\idf}{\mathsf{id}}


\begin{document}

\section{Based Symmetries on the Circle}

\begin{definition}[Suspension]
 Given a type $A$, we write $\Sigma A$ for the \emph{suspension} of $A$, defined to be the (h-) pushout of $1 \leftarrow A \rightarrow 1$.
 It can be implemented as the higher inductive type with points $\North, \South$ and a family $\merid : A \to (\North = \South)$.
\end{definition}

When we later regard $\Sigma A$ as a pointed type, we implicitly choose its base point to be $\North$.

\begin{definition}[Spheres $\sph n$]
 $\sph {-1}$ is defined to be the empty type, and $\sph {n+1}$ is $\Sigma \sph n$.
\end{definition}

Given pointed types $A$ and $B$ (with implicit base points $a_0$ and $b_0$), we write $A \pointedm B$ for the type of \emph{pointed maps}: $\Sigma (f : A \to B). f(a_0) = b_0$.

\begin{lemma}[{Spheres and loops \cite[check-lemma-number]{HoTT}}] \label{lem:sph-loops}
 Given $n \geq 0$ and a pointed type $A$, we have
 \begin{equation}
  (\sph n \pointedm A) \simeq \Omega^n(A).
 \end{equation}
 \qed
\end{lemma}
% \begin{proof}
%  (Todo: I am sure this is a lemma in the HoTT book, but I need to find the number.)
% \end{proof}

\begin{lemma}
 The map $\gamma : \Z \to (\sph 1 \pointedm \sph 1)$, given by $\gamma_n (\lo) \defeq \lo^n$, is an equivalence of types.
\end{lemma}
\begin{proof}
 By \cref{lem:sph-loops}, this is a reformulation of the usual property that the loop space of the circle is $\Z$ \cite[Chp 8.1]{HoTT}.
\end{proof}

We denote the standard multiplication on integers by $(\_ \cdot \_)$.
Note that $\Z$ with multiplication is a group\footnote{Note that the definition of a \emph{group} used here requires the carrier to be a set (a $0$-truncated type).}, and so is $(\sph 1 \pointedm \sph 1)$ with function composition.

\begin{theorem}[$\Z \simeq (\sph 1 \pointedm \sph 1)$]
 The function $\gamma$ is a group isomorphism between $(\Z, \_ \cdot \_)$ and $(\sph 1 \pointedm \sph 1, \_ \circ \_)$.
\end{theorem}
\begin{proof}
(Todo: I expect that this has already been recorded in the HoTT book?
If not, the following needs to be checked in detail.)

The only non-immediate property that needs to be checked is that multiplication of integers corresponds to composition of functions, in the sense that 
the following square commutes (up to homotopy):
\begin{equation}
\begin{tikzpicture}[x=7cm,y=-3cm,baseline=(current bounding box.center)]
 \tikzset{arrow/.style={shorten >=0.1cm,shorten <=.1cm,-latex}}
 \node (A) at (0,0) {$\Z \times \Z$}; 
 \node (B) at (0,1) {$(\sph 1 \pointedm \sph 1) \times (\sph 1 \pointedm \sph 1)$}; 
 \node (C) at (1,1) {$(\sph 1 \pointedm \sph 1)$}; 
 \node (D) at (1,0) {$\Z$}; 

 \draw[arrow] (A) to node [left] {$\gamma \times \gamma$} (B);
 \draw[arrow] (B) to node [above] {$(\_ \circ \_)$} (C);
 \draw[arrow] (D) to node [right] {$\gamma$} (C);
 \draw[arrow] (A) to node [above] {$(\_ \cdot \_)$} (D);
\end{tikzpicture}
\end{equation}
\end{proof}

Since the group of integers has exactly two invertible elements, we get the following
(note that we write $A \xrightarrow{\sim} B$ synonymously with $A \simeq B$):

\begin{corollary}
 The type $(\sph 1 \xrightarrow{\sim}_\bullet \sph 1)$ of pointed equivalences
%  , or equivalently the type $(\sph 1 =_\bullet \sph 1)$ of pointed equalities, 
 is isomorphic to $\mathsf{Bool}$.
\end{corollary}

\section{Based Symmetries of Higher Spheres}

The suspension is a ``functor'' in the somewhat naive sense of directly writing down the usual functor laws of ordinary category theory.
We first record this observation in \cref{lem:Susp-is-functor}
before clarifying why we call the concept a \emph{wild $1$-functor} in \cref{rem:wildness}.

\begin{lemma}[Funcotrial structure of the suspension] \label{lem:Susp-is-functor}
 The suspension can canonically be given the structure of a \emph{wild $1$-functor} in the following sense:
 \begin{enumerate}
  \item \emph{Objects:} for any type $A$, we get a type $\Sigma A$.
  \item \emph{Morphisms:} given a function $f : A \to B$, we get $\Sigma f : \Sigma A \to \Sigma B$.
  \item Given functions $A \xrightarrow f B \xrightarrow g C$, we have $\Sigma(g \circ f) = \Sigma g \circ \Sigma f$.
  \item For the identity function $\idf$, we have $\Sigma \idf_A = \idf_{\Sigma A}$.
 \end{enumerate}
 \qed
\end{lemma}

\begin{remark} \label{rem:wildness}
One should in general not expect that the direct translation of categorical concepts into type theory yields well-behaved concepts.
The reason is that equalities such as the functor laws used above can only be stated using the internal equality type.
As it is well-studied, equalities play the rule of $1$-morphism in the $\infty$-category (groupoid) given by the type.
Consequently, the functor laws as formulated in the lemma above are \emph{not} mere properties but carry structure, and require coherence in order to be well-behaved.
Morally, the suspension should be regarded as an $\infty$-endofunctor on the $\infty$-category $\UU$ of types and functions.\footnote{This can be made precise, cf.\ \cite{ann-cap-kra:two-level}.}
We cannot but fortunately do not need to spell out what this means in our setting.
What happens here (and in many situations in homotopy type theory) is that certain constructions only require a \emph{finite} amount of data of the $\infty$-category.
Recall (cf.\ \cite{capKra_semisegal}) that a \emph{wild $n$-category} is a structure which carries the operations of an $n$-category without being truncated, the intuition being that it ``morally'' is an $\infty$-category with all data from level $(n+1)$ onwards omitted.
We have obvious corresponding notions of wild functors, wild natural transformations, and so on.
In our case, $n \equiv 1$ will turn out to be sufficient.
In other words, the data for a \emph{wild $1$-functor} considered in \cref{lem:Susp-is-functor} is not well-behaved, but it is ``good enough'' for everything that we do in this note.
\end{remark}



While $(\sph 1 \pointedm \sph 1)$ is a group, the same is not the case in general for $(\sph n \pointedm \sph n)$.
The reason is that this type is in general not a set.
However,
$\trunc 0 {\sph n \pointedm \sph n}$ is a group in the canonical way.

\begin{lemma} \label{lem:grp-iso}
 The morphism map $\trunc 0 {\Sigma \_} : \trunc 0 {\sph n \pointedm \sph n} \to \trunc 0 {\sph {n+1} \pointedm \sph {n+1}}$ is an isomorphism of groups for all $n \geq 1$.
\end{lemma}
\begin{proof}
 (Todo: Details need to be worked out, but here is the sketch: The statement follows from the Freudenthal suspension theorem \cite[Thm 8.6.4]{HoTT}. The formulation in \cite[Cor 8.6.14]{HoTT} is useful, and \cite[Thm 8.6.17]{HoTT} \emph{nearly} says the correct thing; the latter ``forgets'' a bit too much, but I think the proof is essentially there.)
\end{proof}


\begin{corollary} \label{cor:Sn-bool}
%  The set-truncation of the type of pointed equivalences 
 The type
 $\trunc 0 {\sph n \xrightarrow{\sim}_\bullet \sph n}$
%  , or equivalently pointed equalities $(\sph n =_\bullet \sph n)$, 
is equivalent to $\mathsf{Bool}$ for all $n \geq 1$.
\end{corollary}

\section{Unbased Symmetries}

The next (and, at the moment, final) step is from the based symmetries considered above to unbased symmetries, i.e.\ from equivalences $\sph n \xrightarrow{\sim}_\bullet \sph n$ to $\sph n \xrightarrow{\sim} \sph n$ (and thus $\sph n = \sph n$).
We start with a lemma allowing us the removal of base points:
\begin{lemma} \label{lem:rm-base-points}
 For a type $B$ and $A$ with explicitly given point $a_0$, we have
 \begin{equation}
  (A \to B) \; \simeq  \; \Sigma (b : B). (A,a_0) \to_\bullet (B,b).
 \end{equation}
\end{lemma}
\begin{proof}
 ``Singleton-contraction.''
\end{proof}

\begin{lemma} \label{lem:eqv-to-bool}
 For all $x : \sph n$ with $n \geq 1$, we have the following equivalence:
 \begin{equation}
  \trunc 0 {(\sph n, \North) \to_\bullet (\sph n, x), \lambda \_ . x} \xrightarrow{\sim}_\bullet (\mathsf{Bool}, \mathsf{true}).
 \end{equation}
\end{lemma}
\begin{proof}
 By induction on $x$.
 Note that the goal is a proposition, i.e.\ it is enough to check the cases for the point constructors. For $\North$, this is given by 
 \cref{cor:Sn-bool}. Note that it is always true that $(B = C)$ implies $(A \to_\bullet B) \simeq (A \to_\bullet C)$;
 thus, a single equivalence on $\sph n$ swapping $\North$ and $\South$ is suffices to cover the case $\South$.
 There is a canonical such equivalence which reverses all meridians.
 (Remark/todo: I think this argument is fine, but one can also allude to a more general result about ``proving truncated properties over spheres'' here. Further, note that we could consider not-necessarily-pointed equivalences, which would make the goal a set, which should still be fine but maybe a bit more tedious.)
\end{proof}

\begin{lemma} \label{lem:sn-bool}
 For $n \geq 1$, we have
 \begin{equation}
  \big(\Sigma (x : \sph n). \trunc 0 {(\sph n , \North) \to_\bullet (\sph n, x)\big)} \; \simeq \; \sph n \times \mathsf{Bool}.
 \end{equation}
\end{lemma}
\begin{proof}
 By taking the total spaces of the result of \cref{lem:eqv-to-bool}.
\end{proof}

\begin{lemma} \label{lem:rm-truncs}
 For all types $A$, families $B$, and $k \geq -2$, we have
 \begin{equation}
  \trunc k {\Sigma (a : A). B(a)} \; \simeq \; \trunc k {\Sigma (a : A). \trunc k {B(a)}}.
 \end{equation}
 \qed
\end{lemma}

Curiously, the main theorem works for $n \equiv 0$, even though the above statements fail for this case (the requirement $n \geq 1$ trickles down from \cref{lem:grp-iso}).
\begin{theorem}
 For all $n \geq 0$, the type of symmetries of $\sph n$ has exactly two connected components.
 In other words,
 \begin{equation}
  \trunc 0 {\sph n = \sph n} = \mathsf{Bool}.
 \end{equation}
\end{theorem}
\begin{proof}
 The case $n \equiv 0$ is easily checked.
 For $n \geq 1$, we have the following chain of equivalences:
  \begin{alignat}{5}
  &&&\trunc 0 {\sph n = \sph n} \\
  \textit{by }\cref{lem:rm-base-points} \qquad &\simeq &\quad& \trunc 0 {\Sigma(x : \sph n).(\sph n, \North) \xrightarrow{\sim}_\bullet (\sph n, x)}  \\
  \textit{by }\cref{lem:rm-truncs} \qquad &\simeq && \trunc 0 {\Sigma(x : \sph n).\trunc 0 {(\sph n, \North) \xrightarrow{\sim}_\bullet (\sph n, x)}}  \\
  \textit{by }\cref{lem:sn-bool} \qquad &\simeq && \trunc 0 {\sph n \times \mathsf{Bool}}  \\
  \textit{ } \qquad &\simeq && \trunc 0 {\sph n} \times \trunc 0 {\mathsf{Bool}}    \\
  \textit{spheres are connected} \qquad &\simeq && \mathsf{Bool}
 \end{alignat}
\end{proof}






\bibliographystyle{plain}
\bibliography{bib}


\end{document}
