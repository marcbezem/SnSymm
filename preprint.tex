\documentclass[english,a4]{article}

\usepackage{amsmath,amsthm,amssymb}
\usepackage{enumitem}
\usepackage{tikz,tikz-cd}
\usepackage{microtype}
\usepackage[mathscr]{euscript}
\usepackage{mathtools}
\usepackage{hyperref}
\usepackage[capitalize]{cleveref}

\input macros%
\input macros2%

\hypersetup{
    colorlinks,
    linkcolor={red!50!black},
    citecolor={blue!50!black},
    urlcolor={blue!80!black}
}

\setlist[enumerate]{label=(\roman*)}

\title{{On symmetries of spheres in univalent foundations}}%
\author{Pierre Cagne\thanks{Universitetet i Bergen} \and Nicolai
  Kraus\thanks{University of Birmingham} \and Marc
  Bezem\thanks{Universitetet i Bergen}}%

\date{\normalsize Last updated on \today}%

%% global tikz styles
\tikzset{cell/.style={%
    shorten <=1em,%
    shorten >=1em,
    /tikz/commutative diagrams/Rightarrow
  }%
}%

\renewcommand{\ap}[1]{\left[{#1}\right]}
\newcommand{\ptdto}{\to_\ast}%
\newcommand{\setTrunc}[1]{\Trunc{#1}_0}
\newcommand{\settrunc}[1]{\trunc{#1}_0}

\begin{document}

\maketitle

\begin{abstract}
  In this paper, we investigate the symmetries of the spheres $\Sn n$
  ($n\geq 1)$ in univalent foundations.

  The case $n=1$ has a slick answer: $(\Sc = \Sc) \weq (\Sc +
  \Sc)$. Unfortunately, it is expected that the result does not
  generalize for $n>1$. However, we show that the type $\Sn n = \Sn n$
  has two equivalent connected components and we exhibit members for
  each component.
\end{abstract}


\section{Introduction}

In this paper, we are interested in the description of the type
$\Sn n = \Sn n$ ($n\geq 1$) where $\Sn n$ is the $n$-dimensional
sphere, as defined in the HoTT-book (cf.~\cite{HoTT}).

To be fully precise and to fix notations, we work inside an
intuitionistic Martin-Löf's type theory with $\Sigma$-, $\Pi$- and
$\mathrm{Id}$-types and with a cumulative hierarchy of universes,
simply written $\UU$, for which Voevodsky's univalence axiom hold. The
transport along $p$ in a type family $T:A\to \UU$ is denoted
$\trp [T] p$, and $\pathover u T p v$ denotes $\trp [T] p {(u)} = v$.
For any dependent function $f\from\prod_{x\from A}T(x)$,
the function
\begin{displaymath}
  \ap f:\prod_{x,y\from A}\prod_{p\from x=y} \pathover {f(x)} T p
  {f(y)}
\end{displaymath}
denotes the dependent application of $f$. We usually leave out the two
first arguments of $\ap f$ as they are inferable, and we write simply
$\ap f (p)$ for a path $p:x=y$. As in \cite{HoTT}, we call a type $A$
a {\em proposition} if the type $\prod_{x,y\from A} x=y$ contains an
element.

We postulate a type $\NN$ of natural numbers and a type $\ZZ$ of
integers with their respective inductive properties
(cf.~\cite[Ch.??]{HoTT}). We allow also several higher inductive types
that are defined in the HoTT-book: propositional truncation,
suspension, and the circle. Considering their importance in this
paper, we recall them in full\footnote{\color{red}Or maybe we should
  make an appendix of that? MB is for an appendix!}:
\begin{enumerate}
\item the {\em propositional truncation} of a type $A$ is a type
  $\Trunc A$ defined by
  \begin{itemize}
  \item a map $\trunc \blank \from A \to \Trunc A$, and
  \item a dependent function
    $\istrunc {-1} \from \prod_{x,y\from \Trunc A} x=y$
  \end{itemize}
  with the following induction rule:
  \begin{quote}
    given a type family $T\from \Trunc A \to \UU$ such that every
    $T(x)$ is a proposition for each $x\from \Trunc A$, every
    dependent function
    \begin{displaymath}
      s\from \prod_{a\from A}T(\trunc a)
    \end{displaymath}
    defines a dependent function
    \begin{displaymath}
      \ind(s)\from\prod_{x\from \Trunc A} T(x)
    \end{displaymath}
    such that $\ind(s)(\trunc a) \jdeq s(a)$ for all $a\from A$.
  \end{quote}
\item the {\em suspension} of a type $A$ is a type $\susp A$ defined
  by
  \begin{itemize}
  \item two elements $N,S\from \susp A$, and
  \item a  map $\mrd \from A \to (N=S)$
  \end{itemize}
  with the following induction rule:
  \begin{quote}
    given a type family $T\from \susp A \to \UU$, every dependent
    triple of elements
    \begin{displaymath}
      n\from T(N), \quad s\from T(S),\quad
      m\from \prod_{a\from A}\pathover n T {\mrd a} s
    \end{displaymath}
    defines a dependent function
    \begin{displaymath}
      \ind(n,s,m)\from\prod_{x\from \susp A} T(x)
    \end{displaymath}
    such that
    \begin{gather*}
      \ind(n,s,m)(N) \jdeq n,\quad \ind(n,s,m)(S) \jdeq s,\\
      \ap{\ind(n,s,m)}(\mrd a) = m(a)\ \text{for all}\ a\from A.
    \end{gather*}
  \end{quote}
\item the {\em circle} $\Sc$ is a type defined by
  \begin{itemize}
  \item an element $\base \from \Sc$, and
  \item a path $\Sloop: \base = \base$
  \end{itemize}
  with the following induction rule:
  \begin{quote}
    given a type family $T\from \Sc \to \UU$, every dependent pair of
    elements
    \begin{displaymath}
      b\from T(\base), \quad \ell\from \pathover b T \Sloop b
    \end{displaymath}
    defines a dependent function
    \begin{displaymath}
      \ind(b,\ell)\from\prod_{x\from \Sc} T(x)
    \end{displaymath}
    such that
    \begin{displaymath}
      \ind(b,\ell)(\base) \jdeq b,\quad \ap{\ind(b,\ell)}(\Sloop) = \ell \,.
    \end{displaymath}
  \end{quote}
\end{enumerate}
The spheres $\Sn n$ are then defined by induction on $n:\NN$ by
$\Sn n \defequi \susp {(\Sn {n-1})}$ for all $n\geq 1$.

???We shall frequently use, for all inductive and higher inductive types,
that functions defined by induction are unique. This is a well-known
result, see \cite{XXX}.

\paragraph{Plan of the paper.}%
In \cref{sec:circle-case}, we show carefully that the type $\Sc = \Sc$
is equivalent to $\Sc + \Sc$. Maybe surprising at first, this result
is quite straightforward once we unravel the details.

In \cref{sec:sphere}, we deal with the case $n=2$. Because the
equivalence of the case $n=1$ relies heavily on $\Omega(\Sc)\weq \ZZ$
being an abelian group, we can not expect to generalize the proof for
$n=2$. However, a careful analysis of Brunerie's proof of
$\pi_2(\Sp) \weq \ZZ$ allows us to prove that the type $\Sp=\Sp$ has
exactly two connected components, equivalent to one another. 
Determining a type with a simple description that is
equivalent to these components is a challenge in itself, still open,
to the authors' knowledge.

In \cref{sec:higher-sphere}, we propagate the result by induction on
$n\geq 2$: the type $\Sn n = \Sn n$ has exactly two connected
components, equivalent to one another. Each induction step relies on
Freudenthal's suspension theorem.

\section{Symmetries of the circle}
\label{sec:circle-case}%

In this section, we will provide an equivalence:
\begin{equation}
  \label{eq:symm-cricle}%
  (\Sc = \Sc) \weq \Sc+\Sc.
\end{equation}
We will treat univalence as transparent, so that equivalences
$f:\Sc\weq \Sc$ will be treated as elements of $\Sc=\Sc$ without any
warning. In particular, we shall write as if
``$\refl \Sc \jdeq \id_\Sc$''.

To provide an equivalence of type \cref{eq:symm-cricle}, we proceed
in several steps:
\begin{itemize}
\item first, we will describe two elements of $\Sc=\Sc$;
\item next, we will prove that these two elements are not equal (and
hence in different connected components);
\item then, we will prove that every equivalence in $\Sc=\Sc$ is
  merely equal to one of these two elements;
\item finally, we will conclude by exhibiting an equivalence between
  $\Sc$ and the connected component of each of these two elements.
\end{itemize}

The first element is simply the identity equivalence $\id_\Sc$. The
second element is the function
$-\id_\Sc \defequi \ind(\base,\inv{\Sloop})$ defined by
$\Sc$-induction in the constant type family at $\Sc$. In other words,
$-\id_\Sc$ is the (propositionally) unique function $\Sc \to \Sc$ such
that:
\begin{displaymath}
  -\id_\Sc(\base) \jdeq \base%
  \quad\text{and}\quad%
  \ap{-\id_\Sc}(\Sloop) = \inv{\Sloop}.
\end{displaymath}

Let us note that $-\id_\Sc$ is an equivalence. It is indeed its own inverse, as
it is shown in the following. In order to construct a proof of $-\id_\Sc \circ
-\id_\Sc = \id_\Sc$, we use function extensionality and $\Sc$-induction.
$\refl\base$ is an element of $-\id_\Sc\circ -\id_\Sc (\base) = \base$, and
we only need to provide an element of $\pathover {\refl\base} T \Sloop
{\refl\base}$ where $T$ is the type family $x \mapsto -\id_\Sc \circ -\id_\Sc (x)
= x$. The transport in the type family $T$ over $\Sloop$ is given by
\begin{displaymath}
  p \mapsto \Sloop \cdot p \cdot \inv{[-\id_\Sc \circ -\id_\Sc](\Sloop)}.
\end{displaymath}
By definition of $-\id_\Sc$, the formula can be rewritten as $\Sloop \cdot p
\cdot \inv \Sloop$.  Hence $\trp[T] \Sloop (\refl\base) = \refl\base$ by simple
path algebra, as wanted. 

\begin{lemma}
  \label{lemma:S1-id-neq-minusid}%
  The proposition $\id_\Sc \neq -\id_\Sc$ holds.
\end{lemma}
\begin{proof}
  Suppose a path $p\from\id_\Sc = -\id_\Sc$ and derive a
  contradiction. Through function extensionality, and because
  $\id_\Sc(\base) \jdeq \base \jdeq -\id_\Sc(\base)$, one gets a path
  $p(\base) \from \base = \base$, and a $2$-path
  $\ap p (\Sloop) \from \pathover {p(\base)} T \Sloop {p(\base)}$ where
  $T(x)\defequi (\id_\Sc(x) = -\id_\Sc(x))$ for all $x:\Sc$. One can
  easily find by induction on $q\from x=y$ a 2-path of type
  \begin{displaymath}
    \trp [T] q (\blank) = \ap{-\id_\Sc}(q) \blank \inv{\ap{\id_\Sc}(q)}.
  \end{displaymath}
  By composition, $\ap p (\Sloop)$ provides a 2-path of type
  $\inv\Sloop p(\base) \inv\Sloop = p(\base)$. Now recall that
  the function
  \begin{equation}
    \label{eq:loopspace-circle-Z}%
    \ZZ\to(\base=\base),\quad k \mapsto \Sloop^k 
  \end{equation}
  is an equivalence. In particular, it shows that path-composition in
  $\base=\base$ is commutative, and $\ap p (\Sloop)$ together with
  path algebra provide then a path of type $\Sloop = \inv\Sloop$. This
  is a contradiction, as it would yield a path of type $1=-1$ in $\ZZ$
  through the equivalence of \cref{eq:loopspace-circle-Z}.
\end{proof}
\cref{lemma:S1-id-neq-minusid} proves that $\id_\Sc$ and $-\id_\Sc$
lie in different connected component of $\Sc=\Sc$. We will now prove
that there are no more connected components in $\Sc=\Sc$. We will need
the following consequence of \cref{lemma:S1-id-neq-minusid}.
\begin{corollary}%
  \label{cor:S1-eq-either-isaprop}%
  For every equivalence $\varphi\from \Sc=\Sc$, the following type is
  a proposition:
  \begin{equation}
    \label{eq:S1-def-target-eq}%
    P(\varphi) \defequi \Trunc{\varphi = \id_\Sc} + \Trunc{\varphi = -\id_\Sc}.
  \end{equation}
\end{corollary}
\begin{proof}
  The type $P(\varphi)$ is the disjoint sum of two propositions. In
  order to be a proposition, it is enough (and in fact necessary) for
  the two summands to not overlap. More precisely, we need to show
  that the premise
  $\Trunc{\varphi = \id_\Sc}\times\Trunc{\varphi = -\id_\Sc}$ leads to
  absurdity. Absurdity is a proposition, so we can as well suppose the
  premise ${(\varphi = \id_\Sc)}\times{(\varphi = -\id_\Sc)}$. By
  composition of paths, it gives a path of type $\id_\Sc = -\id_\Sc$
  and \cref{lemma:S1-id-neq-minusid} allows us to derive a
  contradiction.
\end{proof}

\begin{proposition}
  \label{prop:S1-eq-either}%
  For every equivalence $\varphi\from\Sc=\Sc$, the proposition
  $P(\varphi)$ of \cref{eq:S1-def-target-eq} holds.
\end{proposition}
\begin{proof}
  Let $\varphi$ be a symmetry of the circle, and let $\psi$ denote a
  quasi-inverse of $\varphi$. The type $\Sc$ being connected, one
  has $\Trunc{\base=\varphi(\base)}$ and
  $\Trunc{\base=\psi(\base)}$. Because the goal $P(\varphi)$ is
  propositional, one can instead suppose actual paths
  $\varphi_0\from\base=\varphi(\base)$ and
  $\psi_0\from\base=\psi(\base)$. In other words, we can suppose that
  $\varphi$ and its quasi-inverse $\psi$ are pointed maps. Denote
  $\pi$ for a path of type $\id_\Sc=\varphi\psi$. Then one can craft a
  path of type
  $(\varphi,\varphi_0) \circ (\psi,\psi_0) =
  (\id_\Sc,\pi_{\base}\ap\varphi(\psi_0)\varphi_0)$. Now, consider the
  induced applications
  \begin{displaymath}
    \loopspace{} (\varphi,\varphi_0) \from \loopspace{} \Sc \to \loopspace{} \Sc
    \quad\text{and}\quad
    \loopspace{} (\psi,\psi_0) \from \loopspace{} \Sc \to \loopspace{} \Sc.
  \end{displaymath}
  The elements $\loopspace{} (\varphi,\varphi_0) (\Sloop)$ and
  $\loopspace{} (\psi,\psi_0) (\Sloop)$ of $(\base=\base)$ must be a
  power of $\Sloop$ by the equivalence
  of~\cref{eq:loopspace-circle-Z}. We denote them $\Sloop^k$ and
  $\Sloop^\ell$ respectively. Then the following chain of propositions
  holds:
  \begin{align*}
    \Sloop^{k\ell} &= (\Sloop^k)^\ell\\
    &= \loopspace{} (\varphi,\varphi_0) \left(
      \loopspace{} (\psi,\psi_0) (\Sloop)
    \right)\\
    &= \loopspace{} (\id_\Sc,\pi_{\base}\ap\varphi(\psi_0)\varphi_0) (\Sloop)\\
    &= \inv{\left(\pi_{\base}\ap\varphi(\psi_0)\varphi_0\right)} \Sloop
      \left(\pi_{\base}\ap\varphi(\psi_0)\varphi_0\right)\\
    &= \Sloop
  \end{align*}
  where the last equality exploits commutativity of path composition
  in the loop space $(\base = \base)$. Through the equivalence
  of~\cref{eq:loopspace-circle-Z}, we get that $k\ell=1$ in $\ZZ$,
  from which we get an element of $(k=1)+(k=-1)$. From $k=1$, and
  because
  $\loopspace{} (\varphi,\varphi_0) \jdeq \inv{\varphi_0}
  \ap\varphi(\Sloop) \varphi_0$, one gets an element
  \begin{displaymath}
    b\from \Sloop = \inv{\varphi_0} \ap\varphi(\Sloop)\varphi_0.
  \end{displaymath}
  One can then construct an element $\kappa\from \id_\Sc = \varphi$ by
  $\Sc$-induction by taking $\kappa(\base)$ to be
  $\varphi_0 \from (\base = \varphi(\base))$ and $\ap \kappa(\Sloop)$
  the element $b$ just defined, transported back over the equivalence
  \begin{displaymath}
    (\pathover {\varphi_0} {\id_\Sc(\blank)=\varphi(\blank)} \Sloop {\varphi_0}) \weq
    (\ap\varphi(\Sloop) \varphi_0 \inv \Sloop = \varphi_0)
    \weq (\Sloop = \inv{\varphi_0} \ap\varphi(\Sloop) \varphi_0)
  \end{displaymath}
  By taking $\trunc{\kappa}$, we get an element of
  $\Trunc{\id_\Sc = \varphi}$. Similarly, from $k=-1$, one gets an
  element of $\Trunc{-\id_\Sc =\varphi}$. So in both cases, one gets
  an element of $P(\varphi)$.
\end{proof}

For any type $A$, and an element $a\from A$, let us write $\conncomp A a$
for the connected component of $a$ in $A$. In other words,
\begin{displaymath}
  \conncomp A a \defequi \sum_{x\from A}\Trunc{a=x}.
\end{displaymath}
\cref{lemma:S1-id-neq-minusid} and \cref{prop:S1-eq-either} put
together show that the type $\Sc=\Sc$ has two connected components,
one is the connected component of $\id_\Sc$ and the other is the
connected component of $-\id_\Sc$. In summary, we have shown that
there is an element of the type:
\begin{displaymath}
  (\Sc=\Sc) \weq \conncomp{(\Sc=\Sc)}{\id_\Sc} + \conncomp{(\Sc=\Sc)}{-\id_\Sc}.
\end{displaymath}

In order to prove our goal (cf.~\cref{eq:symm-cricle}), it remains to
exhibit equivalences from $\Sc$ to both
$\conncomp{(\Sc=\Sc)}{\id_\Sc}$ and
$\conncomp{(\Sc=\Sc)}{-\id_\Sc}$. First note that because $\Sc=\Sc$ is
a subtype of $\Sc\to\Sc$, the connected component of some equivalence
$\varphi$ in $\Sc=\Sc$ is equivalent to the connected component of
$\varphi$ seen as a function in $\Sc \to \Sc$. In particular,
\begin{displaymath}
  \conncomp{\left(\Sc=\Sc\right)}{\id_\Sc} \weq \conncomp{\left(\Sc\to\Sc\right)}{\id_\Sc}
  \quad\text{and}\quad
  \conncomp{\left(\Sc=\Sc\right)}{-\id_\Sc} \weq \conncomp{\left(\Sc\to\Sc\right)}{-\id_\Sc}.
\end{displaymath}
Recall that there is an element
\begin{equation}
  \label{eq:S1-loopspace-Z-at-each-x}%
  f\from \prod_{x:\Sc} (\base=\base) \weq (x=x)
\end{equation}
defined by
$\Sc$-induction as the dependent function such that
$f(\base) \jdeq \id_{\base=\base}$ and such that the element
$\ap f (\Sloop) \from (\Sloop\blank\inv\Sloop = \id_{\base=\base})$ is
the reflexivity path of $\id_{\base=\base}$ transported using
commutativity in $\base=\base$ and path algebra. 
Because $(\base=\base)$ is equivalent
to $\ZZ$ (cf.~\cref{eq:loopspace-circle-Z}), it gives an equivalence
$\varepsilon_x\from (x=x) \weq \ZZ$ for any $x\from\Sc$. We can now
exhibit an equivalence:
\begin{displaymath}
  \begin{tikzcd}
    \left(\Sc \to \Sc\right) \rar["\weq"] & (\sum_{x\from\Sc}x=x)
    \rar["\weq"] & \Sc \times \ZZ
    \\
    \varphi \rar[mapsto] & (\varphi(\base),\ap \varphi (\Sloop))
    \rar[mapsto] & \left(%
      \varphi(\base), \varepsilon_{\varphi(\base)}(\ap\varphi(\Sloop))%
    \right)%
\end{tikzcd}
\end{displaymath}
In particular, one can verify that $\id_\Sc$ is mapped to $(\base,1)$
and $-\id_\Sc$ is mapped to $(\base,-1)$ through this
equivalence. Because $\ZZ$ is a set and $\Sc$ is connected, the
connected components of $\Sc\times\ZZ$ are easy to understand: for
every $x:\Sc$ and $k:\ZZ$, the connected component of $(x,k)$ is
equivalent to the type of all those pairs $(x',k)$ for
$x'\from\Sc$. In particular, every connected component of
$\Sc\times\ZZ$ is equivalent to $\Sc$. In particular, one gets
equivalences:
\begin{displaymath}
  \conncomp{\left(\Sc \to \Sc\right)}{\id_\Sc} \weq \Sc
  \quad\text{and}\quad
  \conncomp{\left(\Sc \to \Sc\right)}{-\id_\Sc} \weq \Sc
\end{displaymath}

Finally, one can compose all the equivalences that we exhibited:
\begin{align*}
  (\Sc=\Sc)
  & \weq \conncomp{\left(\Sc = \Sc\right)}{\id_\Sc}
    + \conncomp{\left(\Sc = \Sc\right)}{-\id_\Sc}
  \\
  & \weq \conncomp{\left(\Sc \to \Sc\right)}{\id_\Sc}
    + \conncomp{\left(\Sc \to \Sc\right)}{-\id_\Sc}
  \\
  & \weq \Sc + \Sc.
\end{align*}

\section{Symmetries of the $2$-sphere}
\label{sec:sphere}

In this section, we will prove that the canonical inclusion:
\begin{equation}
  \label{eq:goal-section-sphere}
  \conncomp{\left(\Sp = \Sp\right)}{\id_\Sp} +
  \conncomp{\left(\Sp = \Sp\right)}{-\id_\Sp}
  \to 
  \left(\Sp = \Sp\right)
\end{equation}
is an equivalence. The equivalence $-\id_\Sp \from \Sp \to \Sp$ is
define by $\Sp$-induction as the function such that
$-\id_\Sp(N) \jdeq S$, and $-\id_\Sp(S) \jdeq N$ and
$\ap{-\id_\Sp}(\mrd(x)) = \inv{(\mrd(x))}$ for all $a\from\Sc$. 

\begin{lemma}
  The function $-\id_\Sp$ is an equivalence.
  \label{lem:minus-id-equivalence}
\end{lemma}
\begin{proof}
  We are to produce an element of the type $\prod_{x:\Sp}T(x)$ where $T$ is the
  type family defined by $T(x)\defequi x= (-\id_\Sp \circ -\id_\Sp)(x)$. Let us
  use $\Sp$-induction for this purpose. By definition of $-\id_\Sp$, $T(N)
  \jdeq N$ and $T(S) \jdeq S$. Hence one can select $\refl N:T(N)$ and $\refl
  S:T(S)$. To complete $\Sp$-induction, one should now provide an element of
  type $\prod_{y:\Sc} \pathover{\refl N} T {\mrd y} {\refl S}$. However,
  transporting over a meridian in the family $T$ is conjugation by the
  meridian: indeed, the transport over any path $p:x=x'$ in $T$ is given by
  $q\mapsto \ap{-\id_\Sp \circ -\id_\Sp}(p) \cdot q \cdot \inv p$, and
  \begin{align*}
    \ap{-\id_\Sp \circ -\id_\Sp} (\mrd(y)) 
     &= \ap{-\id_\Sp}(\inv {\mrd(y)}) 
    \\ &= \inv {\left( \ap{-\id_\Sp}(\mrd(y)) \right)} 
    \\ &= \inv { \left( \inv{\mrd(y)} \right) }
    \\ &= \mrd(y).
  \end{align*}
  Hence $\pathover{\refl N} T {\mrd(y)} {\refl S}$ is equivalent to
  $\mrd(y)\refl N \inv {\mrd(y)} = \refl S$, which is indeed inhabited for any
  $y:\Sc$ by simple path algebra.
\end{proof}

We will follow more or less the same road map as
in~\cref{sec:circle-case}:
\begin{itemize}
\item first, we will prove that $\id_\Sp$ and $-\id_\Sp$ are not in
  the same connected component,
\item then, we will prove that every equivalence in $\Sp=\Sp$ is
  either in the connected component of $\id_\Sp$ or in the connected
  component of $-\id_\Sp$,
\item finally, we will prove that the connected component of $\id_\Sp$
  and $-\id_\Sp$ are equivalent to each other.
\end{itemize}
Notice that the last step is less ambitious than in the case of $\Sc$,
where the two connected components were proven equivalent to each
other but also each equivalent to $\Sc$ itself. Although we do not
have any counter-argument for it, we do not expect that the connected
components of $\id_\Sp$ and $-\id_\Sp$ are each equivalent to $\Sp$
itself. Indeed, the proof in the case of $\Sc$ relies heavily on two
facts: $\Sc$ is $1$-truncated and $\Sc$ is the classifying type of an
abelian group. In other words, the homotopy structure of $\Sc$ is very
well understood. This is not the case for $\Sp$: for example, it is
certainly not $2$-truncated (\cite{brunerie:thesis}), and it expected
to be provably not $n$-truncated for any $n$.

The main tool for this section is the Hopf family, defined by Brunerie
in~\cite{brunerie:thesis}, to get an analog in HoTT of the Hopf
fibration in topology. First, let us define, uniformly in $x\from\Sc$, a
function $\iota_x\from \Sc\to\Sc$ by $\Sc$-induction, putting
$\iota_x(\base) \jdeq x$ and $\ap{\iota_x}(\Sloop) = f_x(\Sloop)$.
Here $f\from \prod_{x\from\Sc} (\base=\base) \weq (x=x)$ is the
dependent function defined in~\cref{eq:S1-loopspace-Z-at-each-x}.
Obviously, $\iota_{\base}=\id_{\Sc}$ is an equivalence and hence,
since $\Sc$ is connected, every $\iota_x$ is an equivalence.
Consequently, we may view $\iota_x\from \Sc=\Sc$ as a path.
Notice furthermore that $\iota_x$ is the element
of $\conncomp{(\Sc=\Sc)}{\id_\Sc}$ that corresponds to $x:\Sc$
under the equivalence $\Sc \weq \conncomp{(\Sc=\Sc)}{\id_\Sc}$
exhibited in~\cref{sec:circle-case}. Now define the type family
$\hopf\from \Sp \to \UU$ by $\Sp$-induction as the family such that
\begin{displaymath}
  \hopf(N) \jdeq \Sc,\quad
  \hopf(S) \jdeq \Sc,\quad\text{and}\quad
  \ap\hopf (\mrd(x)) = \iota_x\ \text{for all}\ x:\Sc.
\end{displaymath}

We start by carrying out the first step of the road map.

\begin{lemma}
  \label{lemma:S2-id-neq-minusid}%
  The proposition $\id_\Sp \neq -\id_\Sp$ holds.
\end{lemma}
\begin{proof}
  Suppose $p:\id_\Sp = -\id_\Sp$ and derive a contradiction. Through
  function extensionality, it produces paths
  \begin{displaymath}
    p(N) \from N = S
    \quad\text{and}\quad
    p(S) \from S = N,
  \end{displaymath}
  and for all $x\from\Sc$ a path over
  $\ap p (\mrd(x))\from \pathover {p(N)} T {\mrd(x)} {p(S)}$ where
  $T\from \Sp \to \UU$ is the type family
  $T(a)\defequi (\id_\Sp(a) = -\id_\Sp(a))$. Because $\Sp$ is simply
  connected and we are targeting the empty type $\varnothing$, which
  is a proposition, we might as well assume paths of types $p(N)=\mrd(\base)$
  and $p(S) = \inv{\mrd(\base)}$. Transporting $\ap p(\mrd(x))$ over
  these two paths, one gets a path of type
  $\pathover {\mrd(\base)} T {\mrd(x)} {\inv{\mrd(\base)}}$. 
  Transport over $\mrd(x)$ in the type family $T$ is
  easily proven to be equal to the function
  $q \mapsto \inv{\mrd(x)} q \inv{\mrd(x)}$, so that one
  ends up with a path
  \begin{displaymath}
    \pi_x\from \inv{\mrd(x)} \mrd(\base) \inv{\mrd(x)} = \inv{\mrd(\base)}
    \quad
    \text{for all }x\from\Sc
  \end{displaymath}
  In particular, one has
  \begin{displaymath}
    \pi_{\base} \from \inv{\mrd(\base)} \mrd(\base) \inv{\mrd(\base)}
    = \inv{\mrd(\base)}
  \end{displaymath}
  and $\ap {\pi_{\blank}} (\Sloop)$ produces a path
  $\trp [T'] \Sloop (\pi_{\base}) = \pi_{\base}$, where $T'\from\Sc\to\UU$
  is the type family defined by
  \begin{displaymath}
    T'(x) \defequi \left(\inv{\mrd(x)}\mrd(\base)\inv{\mrd(x)}
      = \inv{\mrd(\base)} \right).
  \end{displaymath}
  The type family $T'$ is just a family of path-types with one fixed
  end, so the transport is easily computed:
  \begin{displaymath}
    \trp [T'] \Sloop (\pi_{\base}) = \pi_{\base} \cdot
    \ap{\inv{\mrd(\blank)} \mrd(\base) \inv{\mrd(\blank)}}(\Sloop).
  \end{displaymath}
  Ultimately, one gets a path of type
  \begin{displaymath}
    \ap{\inv{\mrd(\blank)} \mrd(\base) \inv{\mrd(\blank)}}(\Sloop) =
    \refl{\inv{\mrd(\base)} \mrd(\base) \inv{\mrd(\base)}}.
  \end{displaymath}
  However, by path induction on $p\from \base = x$, one can provide an
  element of type
  \begin{displaymath}
    \ap{\inv{\mrd(\blank)} \mrd(\base) \inv{\mrd(\blank)}}(p) =
    \ap{\inv{\mrd(\blank)}}(p) \hcomp \refl{\mrd(\base)}
    \hcomp \ap{\inv{\mrd(\blank)}}(p). 
  \end{displaymath}
  Indeed, for $p\jdeq \refl{\base}$, both sides reduce to
  $\refl{\inv{\mrd(\base)} \mrd(\base) \inv{\mrd(\base)}}$. One can
  apply that in the case $p\jdeq \Sloop$ and
  invoke~\cref{lemma:mustache-lemma} to get an element of type
  \begin{displaymath}
    {\ap{\inv{\mrd(\blank)}}(\Sloop)}^2 = \refl{\inv{\mrd(\base)}}
  \end{displaymath}
  Notice that the term on the left hand-side is equal to
  $\ap{\inv{\blank}}(\ap\mrd (\Sloop^2))$ and the right hand-side is
  equal to $\ap{\inv{\blank}}(\refl{\mrd(\base)})$. Because
  $\inv{\blank}$ is an equivalence, this is equivalently given by an
  element of type
  \begin{displaymath}
    \ap\mrd(\Sloop^2) = \refl{\mrd(\base)}.
  \end{displaymath}

  Now the Hopf family enters into play. Taking the action of $\ap\hopf$ on
  paths on both sides, one gets:
  
  \begin{displaymath}
    \ap{\ap\hopf\circ \mrd}(\Sloop^2) = \refl{\ap\hopf(\mrd(\base))}
  \end{displaymath}
  
  Recall that, by design, ${\ap\hopf}\circ{\mrd} = \iota$ is an equivalence
  from $\Sc$ to the connected component at $\id_\Sc$ of $\Sc=\Sc$. One ends up
  then with the equation $\Sloop^2 = \refl \base$, or in other words $2=0$ in
  $\zet$, which is absurd.
  
\end{proof}

This proves that $\id_\Sp$ and $-\id_\Sp$ belong to different connected
components. Now we will prove that every equivalence in $\Sp \simeq \Sp$ is
either in the component of $\id_\Sp$ or in the component of $-\id_\Sp$. First,
notice that from the previous lemma, one concludes that the type 
\begin{displaymath}
  \prod_{\varphi:\Sp\weq\Sp}\Trunc{\id_\Sp = \varphi} + \Trunc{-\id_\Sp = \varphi}
\end{displaymath}
is a proposition, following the same exact poof as for
\cref{cor:S1-eq-either-isaprop}. The goal is to prove that this proposition
holds indeed. To this end, take a symmetry $\varphi$ of the sphere and let us
prove that the proposition $\Trunc{\id_\Sp = \varphi} + \Trunc{-\id_\Sp =
\varphi}$ holds. We tackle the problem as follows: 
\begin{enumerate}
  \item Because we are targeting a proposition, and $\Sp$ is
    connected, one can as well suppose that $\varphi$ is pointed. Denote $p:N=\varphi(N)$ the given path.
  \item Recall the degree function $d: (\Sp \ptdto \Sp) \to \ZZ$, which is a
    monoid morphism. In particular, it maps pointed equivalences to invertible
    elements in $\ZZ$, that is $1$ or $-1$.
  \item Proves that $d$ is a set-truncation map, meaning that $d(f)=d(g)$ if and
    only if $\Trunc{f=g}$. This will prove either $\Trunc{(\id_\Sp,\refl
    N)=(\varphi,p)}$ or $\Trunc{(-\id_\Sp,\mrd\base)=(\varphi,p)}$, from which
    one can finally prove that either $\Trunc{\id_\Sp = \varphi}$ or
    $\Trunc{-\id_\Sp = \varphi}$.
\end{enumerate}

In order to perform these steps, one shall first recall a result from the
HoTT-book: the second homotopy group of $\Sp$ is $\ZZ$. More precisely, the
second homotopy group $\pi_2(\Sp)$ is defined as the set truncation
$\setTrunc{\loopspace 2 \Sp}$, and one can exhibit the map
\begin{displaymath}
  \tau: \loopspace\null\Sp \to \Sc, \, p \mapsto \ap\hopf(p)(\base).
\end{displaymath}
Then \cite[Theorem ??]{HoTT} proves that the map $\loopspace\null\tau :
\loopspace 2 \Sp \to \loopspace \null \Sc$ induces an isomorphism of groups
$\setTrunc{\loopspace\null\tau}: \pi_2(\Sp) \to \pi_1(\Sc)$ on the connected
components. Composing with the known isomorphism $\pi_1(\Sc) \weq \ZZ$, one
ends up with a group isomorphism $\zeta: \pi_2(\Sp) \to \ZZ$. The degree of a
pointed map is then defined as the image of $1$ through the induced morphism
between the second homotopy groups, viewed under this equivalence. Formally
speaking one defines: 
\begin{displaymath}
  d(f) \defequi \zeta\left(\pi_2(f)\left(\inv\zeta(1)\right)\right):\ZZ \qquad
  \text{for any}\ f:\Sp\ptdto\Sp.
\end{displaymath}

\begin{proposition}
  The degree function $d$ is a morphism of monoids from the monoid of pointed
  endofunction to the multiplicative monoid $\ZZ$.
  \label{prop:deg-monoid-morphism}
\end{proposition}
\begin{proof}
  First, let us prove that $\id_\Sp$, pointed by $\refl N$, has degree $1$.
  This is easy because $\pi_2(\id_\Sp) = \id_{\pi_2(\Sp)}$ so that $d(\id_\Sp)
  = \zeta(\inv\zeta(1)) = 1$.

  Now, let us prove that $d(g\circ f) = d(g)\times d(f)$ for any $f,g:\Sp
  \ptdto\Sp$. This again comes from the functoriality of $\pi_2$
  (\cite[??]{HoTT}), meaning that $\pi_2(g\circ f) = \pi_2(g)\circ\pi_2(f)$
  holds. Hence:
  \begin{align*}
    d(g\circ f) &= \zeta\left( \pi_2(g\circ f)\left( \inv\zeta(1) \right) \right) 
    \\
    &= \zeta\left( \pi_2(g) \left( \pi_2(f)\left( \inv\zeta(1) \right)
    \right) \right) 
    \\
    &= \zeta\left( \pi_2(g) \left( \inv\zeta \left( \zeta \left( \pi_2(f)\left( \inv\zeta(1) \right)
    \right) \right) \right) \right)
    \\
    &= \zeta \left( \pi_2(g) \left( \inv\zeta \left(d(f)\right) \right) \right)
    \\
    &= \zeta \left( \pi_2(g) \left( \inv\zeta(1)^{d(f)} \right) \right)
    \\
    &= \zeta \left( \pi_2(g) \left( \inv\zeta(1)\right)^{d(f)} \right)
    \\
    &= \zeta \left( \pi_2(g) \left( \inv\zeta(1)\right) \right)\times d(f)
    \\
    &= d(g)\times d(f)
  \end{align*}
  The chain of equality above uses the fact that $\zeta$ is not a mere
  equivalence but also a group morphism: in particular, for any $n:\ZZ$, one
  gets $\inv\zeta(n) = {\inv\zeta(1)}^n$ in $\pi_2(\Sp)$ (where the group
  operation is denoted multiplicatively).
\end{proof}

\begin{corollary}
  The degree of a pointed equivalence is either $1$ or $-1$.
  \label{cor:degree-equivalences}
\end{corollary}
\begin{proof}
  By the previous proposition, invertible elements in $\Sp \ptdto \Sp$ have
  invertible degrees in the multiplicative monoid $\ZZ$. However there is only
  two such elements in $\ZZ$: $1$ and $-1$. 
\end{proof}

The final step is where the complexity hides: one want to prove that the degree
map is an injection on connected components. To do so, we will explicitate
another map $\bar d: (\Sp \ptdto \Sp) \to \ZZ$ which is easily proven an
injection on connected components, and then we will prove that $d = \bar d$.
Recall that for any types $A,B$ respectively pointed by $a$and $b$, there is an
equivalence
\begin{displaymath}
  \Phi_{A,B}: (\susp A \ptdto B) \weq (A \ptdto \loopspace \null B) 
\end{displaymath}
that takes a function $f:\susp A \to B$ pointed by $f_0:b=f(N)$ and maps it to 
\begin{displaymath}
  \inv {f_0}\cdot (\ap f\circ (\inv{\mrd(a)}\mrd(\blank)))\cdot f_0 : A \to \loopspace\null B
\end{displaymath}
which is pointed by the path $\varpi_{f}:\refl b = \inv {f_0} \cdot (\ap f (\inv
\mrd(a)\mrd(a))) \cdot f_0$ given by path algebra operations.

In particular, for each type $A$, one gets a {\em unit} function 
\begin{displaymath}
  \eta_A \defequi \Phi_{A,\susp A}(\id_{\susp A}): A \ptdto \loopspace\null\susp A.
\end{displaymath}
This map is involved in the so-called triangle identity: for any type $B$,
there is an element of the following type
\begin{equation}
  \loopspace\null(\blank) \circ \eta_A = \Phi_{A,B}
  \label{eq:triangle-identity-unit}
\end{equation}
given by simple path algebra: for any map $f:\susp A \to B$ pointed by $f_0:b=f(N)$
\begin{displaymath}
  \loopspace\null(f) \circ \eta_A = \inv {f_0} \cdot \ap f (\eta_A(\blank)) \cdot f_0 
  = \inv {f_0} \cdot \ap f (\inv{\mrd(a)}\cdot\mrd(\blank)) \cdot f_0 = \Phi_{A,B}(f). 
\end{displaymath}

There is now an equivalence $\gamma:(\Sp \ptdto \Sp) \weq \loopspace 2 \Sp$ defined as
the composition:
\begin{displaymath}
  (\Sp \ptdto \Sp) \stackrel{\Phi_{\Sc,\Sp}}\weq (\Sc \ptdto\loopspace\null \Sp)
  \stackrel{\Phi_{\bn 2,\loopspace\null\Sp}}\weq (\bn 2\ptdto\loopspace 2 \Sp) \weq \loopspace 2 \Sp
\end{displaymath}
We now define $\bar d$ as the composition
\begin{displaymath}
  (\Sp \ptdto \Sp) \stackrel \gamma \to \loopspace 2 \Sp \stackrel {\settrunc\blank} \to 
  \pi_2(\Sp) \stackrel \zeta \to \ZZ.
\end{displaymath}

\begin{proposition}
  The equation $d=\bar d$ holds.
  \label{prop:alternative-description-degree}
\end{proposition}
\begin{proof}
  In other words, we want to prove that the outer diagram commutes in the following:
  \begin{displaymath}
    \begin{tikzcd}
      (\Sp \ptdto \Sp) \ar[r,"\loopspace\null"] \ar[dd,"\Phi_{\Sc,\Sp}"{swap}]
      \drar[phantom,"\textcircled{\small 1}"{description, near start}] 
      & (\loopspace \null \Sp \ptdto \loopspace\null\Sp) \ar[r,"\loopspace\null"] \ar[ddl,"\blank\circ\eta_{\Sc}"] 
      \dar["\textcircled{\small 2}", phantom]
      & (\loopspace 2 \Sp \ptdto \loopspace 2 \Sp) \ar[ddr,"\ev_\ell"] 
      \ar[r,"\setTrunc\blank"] \dar["\blank\circ \loopspace\null(\eta_{\Sc})"swap] 
      & (\pi_2(\Sp) \to \pi_2(\Sp)) \drar["\ev_{\inv\zeta(1)}"] \ar[dd,"\textcircled{\small 5}", phantom] &
      \\
      & {} \drar["\textcircled{\small 3}"{near end}, phantom] 
      & (\loopspace\null\Sc \ptdto \loopspace 2 \Sp) \dar["\blank\circ\eta_{\bn 2}"swap] 
      \drar["\textcircled{\small 4}", phantom] &
      & \pi_2(\Sp)
      \\
      (\Sc \ptdto \loopspace\null\Sp) \ar[rr,"\Phi_{\bn 2,\loopspace\null\Sp}"swap] \ar[urr,"\loopspace\null"] & 
      & (\bn 2 \ptdto \loopspace 2 \Sp) \rar["\ev_0"swap] & \loopspace 2 \Sp \urar["\settrunc\blank"swap] & 
    \end{tikzcd}
  \end{displaymath}
  We will do that by proving that each of the small inner diagrams denoted
  $\textcircled{\small 1},\dots,\textcircled{\small 5}$ commutes, for
  elementary reasons. Triangles $\textcircled{\small 1}$ and
  $\textcircled{\small 3}$ commute as instances of
  \cref{eq:triangle-identity-unit}. The commutativity of square
  $\textcircled{\small 2}$ simply expresses the functoriality of
  $\loopspace\null$. In order to prove that $\textcircled{\small 4}$ and
  $\textcircled{\small 5}$ commute, we first need to define what is denoted
  $\ell$: by definition, we set $\ell \defequi \loopspace\null (\eta_\Sc)(\Sloop)$.
  Then it is almost immediate that $\textcircled{\small 4}$ commutes because
  under the equivalence $\Sc = \susp {\bn 2}$, one has $\Sloop = \eta_2(0)$.
  Now the commutativity of $\textcircled{\small 5}$ will follow from the
  functoriality of $\setTrunc\blank$ once we have shown $\settrunc \ell =
  \inv\zeta(1)$. By definition of $\zeta$, this is equivalent to showing that
  $\loopspace\null(\tau) (\ell) = \Sloop$ in $\loopspace\null \Sc$. However, we
  have seen\footnote{\color{red}Actually we have not, let's make it a lemma
  soon after the definition of $\tau$} that $\loopspace\null (\eta_\Sc)$ is a
  section of $\loopspace\null(\tau)$. Hence it follows that:
  \begin{displaymath}
    \loopspace\null(\tau)(\ell) = \loopspace\null(\tau)(\loopspace\null(\eta_\Sc)(\Sloop)) = \Sloop.
  \end{displaymath}
\end{proof}

In the following, we make the abuse of taking $\id_\Sp$ pointed by the path
$\refl N$, and $-\id_\Sp$ pointed by the path $\mrd(\base)$.
\begin{corollary}
  A pointed equivalence is in the connect component of either $\id_\Sp$ or $-\id_\Sp$.
  \label{cor:equivalence-conn-component}
\end{corollary}
\begin{proof}
  The previous result shows that $d$ is equal to the composition of
  $\settrunc\blank:\loopspace\null\Sp \to\ZZ$ with two equivalences. As
  $\Trunc{x = y}$ if and only if $\settrunc x = \settrunc y$ for any
  $x,y:\loopspace\null \Sp$, the same property holds for $d$:
  \begin{displaymath}
    d(f) = d(g) \quad \text{if and only if}\quad
    \Trunc{f = g} \quad \text{for any}\ f,g:\Sp \ptdto\Sp.
  \end{displaymath}
  Given a pointed equivalencei $\varphi$, we know from
  \cref{cor:degree-equivalences} that its degree is either $1$ or $-1$.
  However, one has already computed that $d(\id_\Sp) = 1$, and it remains only
  to prove that $d(-\id_\Sp) = -1$. For that it is enough to show that the type
  $\loopspace 2 (-\id_\Sp)(\ell) = \inv\ell$ has an element (where $\ell$ is
  defined as $\loopspace\null(\eta_\Sc)(\Sloop)$ as in the previous proof), for
  then
  \begin{displaymath}
    d(-\id_\Sp) \jdeq \zeta(\pi_2(-\id_\Sp)(\settrunc{\ell})) = \zeta\left( \inv{\settrunc\ell} \right) = - \zeta(\inv\zeta(1)) = -1.
  \end{displaymath}
  Let us compute:
  \begin{align*}
    \loopspace 2 (-\id_\Sp)(\ell) &= \loopspace 2 (-\id_\Sp) \left( \loopspace\null(\eta_\Sc)(\Sloop) \right)
    \\
    &= \loopspace\null \left( \Phi_{\Sc,\Sp}(-\id_\Sp) \right)(\Sloop)
  \end{align*}
  However one can verify that $\Phi_{\Sc,\Sp}(-\id_\Sp) =
  \inv{\eta_\Sc(\blank)}$ has an element. Because $\loopspace\null(\inv\blank)
  = \inv\blank$ ({\color{red}abstract non sense from mustache style lemma, we
  might want to put something in appendix}), it follows that
  $\loopspace\null(\Phi_{\Sc,\Sp}(-\id_\Sp))(\Sloop) =
  \inv{\loopspace\null\eta_\Sc(\Sloop)}$. Hence, $\loopspace 2 (-\id_\Sp)(\ell)
  = \inv{\ell}$.
\end{proof}

\begin{proposition}
  The canonical inclusion
  \begin{displaymath}
    \conncomp{(\Sp = \Sp)}{\id_\Sp} + \conncomp{(\Sp = \Sp)}{-\id_\Sp} \to (\Sp = \Sp)
  \end{displaymath}
  is an equivalence.
  \label{prop:symm-S2-connected-components}
\end{proposition}
\begin{proof}
  In other words, one wants to prove that a symmetry of $\Sp$ is in the
  connected component of either $\id_\Sp$ or $-\id_\Sp$. Given an equivalence
  $\varphi:\Sp \weq \Sp$, the target 
  \begin{displaymath}
    \Trunc{\id_\Sp = \varphi} + \Trunc{-\id_\Sp = \varphi}
  \end{displaymath}
  is a proposition by \cref{lemma:S2-id-neq-minusid}. Hence, one can as well
  assume that $\varphi$ is pointed by a path $p:N=\varphi(N)$. Then from
  \cref{prop:symm-S2-connected-components}, it follows that:
  \begin{displaymath}
    \Trunc{(\id_\Sp,\refl N) = (\varphi, p)} + \Trunc{(-\id_\Sp,\mrd(\base)) = (\varphi, p)}.
  \end{displaymath}
  It is then enough to prove the two following propositions:
  \begin{align*}
    \Trunc{(\id_\Sp,\refl N) = (\varphi, p)} &\to \Trunc{\id_\Sp = \varphi},
    \\
    \Trunc{(-\id_\Sp,\mrd(\base)) = (\varphi, p)} &\to \Trunc{-\id_\Sp = \varphi}
  \end{align*}
  Both implications come from the general behaviour of truncations with
  projections. ({\color{red} Find an explicit reference in HoTT-Book if
  possible.})
\end{proof}

\section{Symmetries of the $n$-sphere}
\label{sec:higher-sphere}


\bibliographystyle{alpha}
\bibliography{bib}

\end{document}


%%% Local Variables:
%%% mode: latex
%%% TeX-master: t
%%% reftex-default-bibliography: ("bib.bib")
%%% End:
